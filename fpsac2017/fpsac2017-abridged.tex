%%% Version 2, second submission to FPSAC 2017.
%%% Built from ../dp-abstr.tex.
%%% FIXME: Address field is fucked up by SWP

\documentclass[submission]{FPSAC2017}%
\usepackage{amsmath,amssymb,url}
\usepackage{color}
\usepackage{xcolor}
\usepackage[all,cmtip,color]{xy}
\usepackage{amsfonts}
\usepackage{amssymb}
\usepackage{framed}
\usepackage{amsmath}
\usepackage{comment}
\usepackage{needspace}
\usepackage[latin1]{inputenc}
\usepackage{graphicx}%
\setcounter{MaxMatrixCols}{30}
%TCIDATA{OutputFilter=latex2.dll}
%TCIDATA{Version=5.50.0.2960}
%TCIDATA{LastRevised=Sunday, November 13, 2016 00:58:42}
%TCIDATA{<META NAME="GraphicsSave" CONTENT="32">}
%TCIDATA{<META NAME="SaveForMode" CONTENT="1">}
%TCIDATA{BibliographyScheme=BibTeX}
%BeginMSIPreambleData
\providecommand{\U}[1]{\protect\rule{.1in}{.1in}}
%EndMSIPreambleData
\newtheorem{thm}{Theorem}
\newtheorem{lem}{Lemma}
\newtheorem{theorem}{Theorem}
\newtheorem{proposition}[theorem]{Proposition}
\newtheorem{conjecture}[theorem]{Conjecture}
\newtheorem{corollary}[theorem]{Corollary}
\newtheorem{remark}[theorem]{Remark}
\newtheorem{lemma}[theorem]{Lemma}
\theoremstyle{definition}
\newtheorem{definition}[theorem]{Definition}
\newtheorem{example}[theorem]{Example}
\iffalse
\newenvironment{proof}[1][Proof]{\noindent\textbf{#1.} }{\ \rule{0.5em}{0.5em}}
\fi
\let\sumnonlimits\sum
\let\prodnonlimits\prod
\let\cupnonlimits\bigcup
\let\capnonlimits\bigcap
\renewcommand{\sum}{\sumnonlimits\limits}
\renewcommand{\prod}{\prodnonlimits\limits}
\renewcommand{\bigcup}{\cupnonlimits\limits}
\renewcommand{\bigcap}{\capnonlimits\limits}
\newcommand{\kk}{{\mathbf{k}}}
\newcommand{\xx}{{\mathbf{x}}}
\newcommand{\id}{{\operatorname{id}}}
\newcommand{\ev}{\operatorname{ev}}
\newcommand{\Adm}{\operatorname{Adm}}
\newcommand{\pack}{\operatorname{pack}}
\newcommand{\Comp}{{\operatorname{Comp}}}
\newcommand{\QSym}{{\operatorname{QSym}}}
\newcommand{\Par}{\operatorname{Par}}
\newcommand{\bD}{{\mathbf{D}}}
\newcommand{\EE}{{\mathbf{E}}}
\newcommand{\FF}{{\mathbf{F}}}
\newcommand{\bk}{{\mathbf{k}}}
\newcommand{\NN}{{\mathbb{N}}}
\newcommand{\ZZ}{{\mathbb{Z}}}
\newcommand{\QQ}{{\mathbb{Q}}}
\received{\today}
\abstract{A quasisymmetric function is assigned to every double poset
(that is, every finite set endowed with two partial orders) and any weight
function on its ground set. This generalizes well-known objects such as
monomial and fundamental quasisymmetric functions, (skew) Schur functions,
dual immaculate functions, and quasisymmetric $\left(  P, \omega\right)
$-partition enumerators. We prove a formula for the antipode of this
function that holds under certain conditions (which are satisfied when the
second order of the double poset is total, but also in some other cases); this
restates (in a way that to us seems more natural) a result by Malvenuto and
Reutenauer, but our proof is new and self-contained. We generalize it further
to an even more comprehensive setting, where a group acts on the double poset
by automorphisms.}
\keywords{
antipodes,
double posets,
Hopf algebras,
posets,
P-partitions,
quasisymmetric functions
}
\begin{document}

\title{Double posets and the antipode of ${\operatorname{QSym}}$ (extended abstract)}
\author{Darij Grinberg\thanks{darijgrinberg@gmail.com}\addressmark{1}}
\address{\addressmark{1}University of Minnesota, School of Mathematics, 206 Church St.
SE, Minneapolis, MN 55455, USA}
\maketitle

%% note that you DO NOT have to put your abstract here -- it is generated by \maketitle and the \abstract and \resume commands above


\section{Introduction}

\label{sec:in}

Double posets and ${\mathbf{E}}$-partitions (for ${\mathbf{E}}$ a double
poset) have been introduced by Claudia Malvenuto and Christophe Reutenauer
\cite{Mal-Reu-DP} in their definition of a ``Hopf algebra
of double posets''. We shall employ these same
notions to study a formula for the antipode in the Hopf
algebra ${\operatorname{QSym}}$ of quasisymmetric functions due to (the same)
Malvenuto and Reutenauer \cite[Theorem 3.1]{Mal-Reu}.
We shall restate this formula in a more natural form, outline a
new (and self-contained) proof, and extend it further to a
setting in which a group acts on the double poset.
 This
 latter extension, and with it the whole work,
 owes its inspiration to Katharina Jochemko's \cite{Joch}.
//
This extended abstract surveys the results in \cite{Grinbe16} and
sketches the main ideas of the proofs. For details, we refer to
\cite{Grinbe16}.

\section{Notations}

% Let us first briefly introduce the notations that will be used in the following.

We set ${\mathbb{N}}=\left\{  0,1,2,\ldots\right\}  $. A \textit{composition}
means a finite sequence of positive integers. We let ${\operatorname{Comp}}$
be the set of all compositions. For any composition $\alpha=\left(  \alpha
_{1},\alpha_{2},\ldots,\alpha_{k}\right)  $, set $\left\vert \alpha\right\vert
=\alpha_{1}+\alpha_{2}+\cdots+\alpha_{k}$.
%For $n \in{\mathbb{N}%
%}$, a \textit{composition of $n$} means a composition whose entries sum to $n$
%(that is, a composition $\left(  \alpha_{1}, \alpha_{2}, \ldots, \alpha
%_{k}\right)  $ satisfying $\alpha_{1} + \alpha_{2} + \cdots+ \alpha_{k} = n$).


Fix a commutative ring $\kk$. We consider the ${\mathbf{k}}%
$-algebra $\mathbf{k}\left[  \left[  x_{1},x_{2},x_{3},\ldots\right]  \right]
$ of formal power series in infinitely many (commuting) indeterminates
$x_{1},x_{2},x_{3},\ldots$ over ${\mathbf{k}}$. A \textit{monomial} shall
always mean a monomial (without coefficients) in the variables $x_{1}%
,x_{2},x_{3},\ldots$. The algebra $\mathbf{k}\left[  \left[  x_{1},x_{2}%
,x_{3},\ldots\right]  \right]  $ comes with a topology that provides meaning
to certain infinite sums; see \cite[�2]{Grinbe16} for details. A power series
$f\in\mathbf{k}\left[  \left[  x_{1},x_{2},x_{3},\ldots\right]  \right]  $ is
said to be \textit{bounded-degree} if there exists a $d\in{\mathbb{N}}$ such
that no monomial of degree $>d$ appears in $f$.

Two monomials $\mathfrak{m}$ and $\mathfrak{n}$ are said to be
\textit{pack-equivalent}
% \footnote{Pack-equivalence and the related notions of
% packed combinatorial objects that we will encounter below originate in work of
% Hivert, Novelli and Thibon. Simple as they are, they are of
% great help in dealing with quasisymmetric functions.}
if they have the forms
$x_{i_{1}}^{a_{1}}x_{i_{2}}^{a_{2}}\cdots x_{i_{\ell}}^{a_{\ell}}$ and
$x_{j_{1}}^{a_{1}}x_{j_{2}}^{a_{2}}\cdots x_{j_{\ell}}^{a_{\ell}}$ for two
strictly increasing sequences $\left(  i_{1}<i_{2}<\cdots<i_{\ell}\right)  $
and \newline$\left(  j_{1}<j_{2}<\cdots<j_{\ell}\right)  $ of positive
integers and one (common) sequence $\left(  a_{1},a_{2},\ldots,a_{\ell
}\right)  $ of positive integers.\footnote{For instance, $x_{2}^{2}x_{3}%
x_{4}^{2}$ is pack-equivalent to $x_{1}^{2}x_{4}x_{8}^{2}$ but not to
$x_{2}x_{3}^{2}x_{4}^{2}$.} A power series $f\in\mathbf{k}\left[  \left[
x_{1},x_{2},x_{3},\ldots\right]  \right]  $ is said to be
\textit{quasisymmetric} if every two pack-equivalent monomials have equal
coefficients in front of them in $f$. The set of quasisymmetric bounded-degree
power series in $\mathbf{k}\left[  \left[  x_{1},x_{2},x_{3},\ldots\right]
\right]  $ is a ${\mathbf{k}}$-subalgebra of $\mathbf{k}\left[  \left[
x_{1},x_{2},x_{3},\ldots\right]  \right]  $, and is known as the
\textit{${\mathbf{k}}$-algebra of quasisymmetric functions over ${\mathbf{k}}%
$}. It is denoted by ${\operatorname{QSym}}$. It is clear that the symmetric
bounded-degree power series in $\mathbf{k}\left[  \left[  x_{1},x_{2}%
,x_{3},\ldots\right]  \right]  $ (commonly known in combinatorics as the
\textit{symmetric functions}) form a ${\mathbf{k}}$-subalgebra of
${\operatorname{QSym}}$. The quasisymmetric functions have a rich theory which
is related to, and often sheds new light on, the classical theory of symmetric
functions; this theory goes back to Gessel \cite{Gessel} and
Malvenuto and Reutenauer \cite{Mal-Reu-dua}, and
expositions can be found in \cite[��\ 7.19, 7.23]{Stanley-EC2} and
\cite[��5-6]{Reiner} and other sources.

For every composition $\alpha = \left(
\alpha_{1},\alpha_{2},\ldots,\alpha_{\ell}\right) \in \Comp$, we set
\[
M_{\alpha}=\sum_{i_{1}<i_{2}<\cdots<i_{\ell}}x_{i_{1}}^{\alpha_{1}}x_{i_{2}%
}^{\alpha_{2}}\cdots x_{i_{\ell}}^{\alpha_{\ell}}=\sum_{\mathfrak{m}\text{ is
a monomial pack-equivalent to }x_{1}^{\alpha_{1}}x_{2}^{\alpha_{2}}\cdots
x_{\ell}^{\alpha_{\ell}}}\mathfrak{m}%
\]
(where the $i_{k}$ in the first sum are positive integers). Then,
$\left( M_{\alpha}\right)  _{\alpha\in{\operatorname{Comp}}}$ is known
to be a basis of the $\kk$-module $\QSym$; it is known as the
\textit{monomial basis} of ${\operatorname{QSym}}$.

The ${\mathbf{k}}$-algebra ${\operatorname{QSym}}$ can be endowed with a
structure of a ${\mathbf{k}}$-coalgebra which, combined with its ${\mathbf{k}%
}$-algebra structure, turns it into a Hopf algebra. We refer to the literature
both for the theory of coalgebras and Hopf algebras (see \cite{Montg-Hopf},
\cite[�1]{Reiner}, \cite[�1-�2]{Manchon-HA}, etc.) and for a deeper study of
the Hopf algebra ${\operatorname{QSym}}$ (see, e.g., \cite[�5]{Reiner}); we
shall need but the very basics of this structure, and so it is only them that
we introduce.

We define a ${\mathbf{k}}$-linear map $\Delta:{\operatorname{QSym}}%
\rightarrow{\operatorname{QSym}}\otimes{\operatorname{QSym}}$ (here and in the
following, all tensor products are over ${\mathbf{k}}$ by default) by
requiring that
\[
\Delta\left(  M_{\left(  \alpha_{1},\alpha_{2},\ldots,\alpha_{\ell}\right)
}\right)   =\sum_{k=0}^{\ell}M_{\left(  \alpha_{1},\alpha_{2},\ldots
,\alpha_{k}\right)  }\otimes M_{\left(  \alpha_{k+1},\alpha_{k+2}%
,\ldots,\alpha_{\ell}\right)  }
 \quad\text{ for every }\left(  \alpha_{1},\alpha_{2},\ldots,\alpha_{\ell
}\right)  \in{\operatorname{Comp}.}
\]
(By linearity, this defines $\Delta$ on all of $\operatorname*{QSym}$, since
$\left(  M_{\alpha}\right)  _{\alpha\in{\operatorname{Comp}}}$ is a basis of
the $\mathbf{k}$-module $\operatorname*{QSym}$.) We further define a
${\mathbf{k}}$-linear map $\varepsilon:{\operatorname{QSym}}\rightarrow
{\mathbf{k}}$ by requiring that
\[
\varepsilon\left(  M_{\left(  \alpha_{1},\alpha_{2},\ldots,\alpha_{\ell
}\right)  }\right)  =\delta_{\ell,0}\quad\text{ for every }\left(  \alpha
_{1},\alpha_{2},\ldots,\alpha_{\ell}\right)  \in{\operatorname{Comp}.}%
\]
(Equivalently, $\varepsilon$ sends every power series $f\in
{\operatorname{QSym}}$ to the result $f\left(  0,0,0,\ldots\right)  $ of
substituting zeroes for the variables $x_{1},x_{2},x_{3},\ldots$ in $f$. The
map $\Delta$ can also be described in such terms, but with greater difficulty
\cite[(5.3)]{Reiner}.) It is well-known that these maps $\Delta$ and
$\varepsilon$ satisfy the equalities
\[
\left(  \Delta\otimes\operatorname*{id}\nolimits_{\operatorname*{QSym}%
}\right)  \circ\Delta=\left(  \operatorname*{id}%
\nolimits_{\operatorname*{QSym}}\otimes\Delta\right)  \circ\Delta
,\quad \left(  \varepsilon\otimes\operatorname*{id}%
\nolimits_{\operatorname*{QSym}}\right)  \circ\Delta=\iota_{1}%
,\quad \left(  \operatorname*{id}\nolimits_{\operatorname*{QSym}%
}\otimes\varepsilon\right)  \circ\Delta=\iota_{2}%
\]
(where $\iota_{1}:\operatorname*{QSym}\rightarrow\mathbf{k}\otimes
\operatorname*{QSym}$ and $\iota_{2}:\operatorname*{QSym}\rightarrow
\operatorname*{QSym}\otimes\mathbf{k}$ are the canonical isomorphisms), and so
$\left(  {\operatorname{QSym}},\Delta,\varepsilon\right)  $ is what is
commonly called a \textit{${\mathbf{k}}$-coalgebra}. Furthermore, $\Delta$ and
$\varepsilon$ are ${\mathbf{k}}$-algebra homomorphisms, which is what makes
this ${\mathbf{k}}$-coalgebra ${\operatorname{QSym}}$ into a
\textit{${\mathbf{k}}$-bialgebra}. Finally, let $m:{\operatorname{QSym}%
}\otimes{\operatorname{QSym}}\rightarrow{\operatorname{QSym}}$ be the
${\mathbf{k}}$-linear map sending every pure tensor $a\otimes b$ to $ab$, and
let $u:{\mathbf{k}}\rightarrow{\operatorname{QSym}}$ be the ${\mathbf{k}}%
$-linear map sending $1\in{\mathbf{k}}$ to $1\in{\operatorname{QSym}}$. Then,
there exists a unique ${\mathbf{k}}$-linear map $S:{\operatorname{QSym}%
}\rightarrow{\operatorname{QSym}}$ satisfying%
\begin{equation}
m\circ\left(  S\otimes\operatorname*{id}\right)  \circ\Delta=u\circ
\varepsilon=m\circ\left(  \operatorname*{id}\otimes S\right)  \circ\Delta.
\label{eq.antipode}%
\end{equation}
This map $S$ is known as the \textit{antipode} of ${\operatorname{QSym}}$. It
is known to be an involution and an algebra automorphism of
${\operatorname{QSym}}$, and its action on the various quasisymmetric
functions defined combinatorially is the main topic of this note. The
existence of the antipode $S$ makes ${\operatorname{QSym}}$ into a
\textit{Hopf algebra}.

\section{Double posets}

Next, we shall introduce the notion of a double poset, following
\cite{Mal-Reu-DP}.

\begin{definition}
\label{def.double-poset}

\begin{itemize}
\item[(a)] We shall encode posets as pairs $\left(  P,<\right)  $, where $P$
is a set and $<$ is a strict partial order (i.e., an irreflexive, transitive
and antisymmetric binary relation) on the set $P$; this relation $<$ will be
regarded as the smaller relation of the poset.

\item[(b)] If $<$ is a strict partial order on a set $P$, and if $a \in P$
and $b \in P$, then we say that $a$ and $b$ are \textit{$<$%
-comparable} if either $a < b$ or $a = b$ or $b < a$. A strict partial
order $<$ on a set $P$ is said to be a \textit{total order} if and only if
every two elements of $P$ are $<$-comparable.

\item[(c)] If $<$ is a strict partial order on a set $P$, and if $a \in P$
and $b \in P$, then we say that $a$ is \textit{$<$-covered by $b$}
if we have $a<b$ and there exists no $c\in P$ satisfying $a<c<b$.

\item[(d)] A \textit{double poset} is defined as a triple $\left(  E, <_{1},
<_{2}\right)  $ where $E$ is a finite set and $<_{1}$ and $<_{2}$ are two
strict partial orders on $E$.

\item[(e)] A double poset $\left(  E, <_{1}, <_{2}\right)  $ is said to be
\textit{special} if the relation $<_{2}$ is a total order.

\item[(f)] A double poset $\left(  E,<_{1},<_{2}\right)  $ is said to be
\textit{tertispecial} if it satisfies the following condition: If $a$ and $b$
are two elements of $E$ such that $a$ is $<_{1}$-covered by $b$, then $a$ and
$b$ are $<_{2}$-comparable.\footnote{The notions of a double poset and of a
special double poset come from \cite{Mal-Reu-DP}. See \cite{Foissy13}
for more about the latter.
The notion of a
\textquotedblleft tertispecial double poset\textquotedblright\ (in hindsight,
\textquotedblleft locally special\textquotedblright\ would be better, but
other authors have already adopted this one) appears to be new and arguably
sounds artificial, but is the most suitable setting for the results
below (and appears in nature, beyond the particular case of special double
posets -- see Example~\ref{exam.dp}).}

\item[(g)] If $<$ is a binary relation on a set $P$, then the \textit{opposite
relation} of $<$ is defined to be the binary relation $>$ on the set $P$ defined
by the equivalence $\left(e > f\right) \Longleftrightarrow \left(f < e\right)$.
Notice that if $<$ is a strict partial order, then so is the
opposite relation $>$ of $<$.
\end{itemize}
\end{definition}

\begin{definition}
\label{def.E-partition} If ${\mathbf{E}} = \left(  E, <_{1}, <_{2}\right)  $
is a double poset, then an \textit{${\mathbf{E}}$-partition} shall mean a map
$\phi: E \to\left\{  1,2,3,\ldots\right\}  $ such that:

\begin{itemize}
\item every $e \in E$ and $f \in E$ satisfying $e <_{1} f$ satisfy
$\phi\left(  e\right)  \leq\phi\left(  f\right)  $;

\item every $e \in E$ and $f \in E$ satisfying $e <_{1} f$ and $f <_{2} e$
satisfy $\phi\left(  e\right)  < \phi\left(  f\right)  $.
\end{itemize}
\end{definition}

\begin{example}
\label{exam.dp} The notion of an ${\mathbf{E}}$-partition (which was inspired
by the earlier notions of $P$-partitions and $\left(  P,\omega\right)
$-partitions as studied by Gessel and Stanley\footnote{See \cite{Gessel-Ppar}
for the history of these notions, and \cite{Gessel} and
\cite[�7.19]{Stanley-EC2} for some of their
theory. Mind that these sources use different and sometimes incompatible
notations -- e.g., the $P$-partitions of
\cite{Gessel-Ppar} differ from those of \cite{Gessel} by a sign reversal.})
generalizes various well-known combinatorial concepts. For example:

\begin{itemize}
\item If $<_{2}$ is the same order as $<_{1}$ (or any extension of this
order), then ${\mathbf{E}}$-partitions are weakly increasing maps from the
poset $\left(  E, <_{1}\right)  $ to the totally ordered set $\left\{  1, 2,
3, \ldots\right\}  $.

\item If $<_{2}$ is the opposite relation of $<_{1}$ (or any extension of this
opposite relation), then ${\mathbf{E}}$-partitions are strictly increasing
maps from the poset $\left(  E, <_{1}\right)  $ to the totally ordered set
$\left\{  1, 2, 3, \ldots\right\}  $.
\end{itemize}

For a more interesting example, let $\mu= \left(  \mu_{1}, \mu_{2}, \mu_{3},
\ldots\right)  $ and $\lambda= \left(  \lambda_{1}, \lambda_{2}, \lambda_{3},
\ldots\right)  $ be two partitions such that $\mu\subseteq\lambda$. (See
\cite[�2]{Reiner} for the notations we are using here.) The skew Young diagram
$Y\left(  \lambda/ \mu\right)  $ is then defined as the set of all $\left(  i,
j\right)  \in\left\{  1, 2, 3, \ldots\right\}  ^{2}$ satisfying $\mu_{i} < j
\leq\lambda_{i}$. On this set $Y\left(  \lambda/ \mu\right)  $, we define two
strict partial orders $<_{1}$ and $<_{2}$ by
\begin{align*}
\left(  i,j\right)  <_{1} \left(  i^{\prime},j^{\prime}\right)
&\Longleftrightarrow\left(  i \leq i^{\prime}\text{ and } j \leq j^{\prime
}\text{ and } \left(  i,j\right)  \neq\left(  i^{\prime},j^{\prime}\right)
\right) \qquad \text{ and} \\
\left(  i,j\right)  <_{2} \left(  i^{\prime},j^{\prime}\right)
&\Longleftrightarrow\left(  i \geq i^{\prime}\text{ and } j \leq j^{\prime
}\text{ and } \left(  i,j\right)  \neq\left(  i^{\prime},j^{\prime}\right)
\right)  .
\end{align*}
The resulting double poset $\mathbf{Y}\left(  \lambda/ \mu\right)  = \left(
Y\left(  \lambda/ \mu\right)  , <_{1}, <_{2}\right)  $ has the property that
the $\mathbf{Y}\left(  \lambda/ \mu\right)  $-partitions are precisely the
semistandard tableaux of shape $\lambda/ \mu$. (Again, see \cite[�2]{Reiner}
for the meaning of these words.)

This double poset $\mathbf{Y}\left(  \lambda/\mu\right)  $ is not special (in
general), but it is tertispecial. Some authors prefer to use a special double
poset instead, which is defined as follows: We define a total order $<_{h}$ on
$Y\left(  \lambda/\mu\right)  $ by
\[
\left(  i,j\right)  <_{h}\left(  i^{\prime},j^{\prime}\right)
\Longleftrightarrow\left(  i>i^{\prime}\text{ or }\left(  i=i^{\prime}\text{
and }j<j^{\prime}\right)  \right)  .
\]
Then, $\mathbf{Y}_{h}\left(  \lambda/\mu\right)  =\left(  Y\left(  \lambda
/\mu\right)  ,<_{1},<_{h}\right)  $ is a special double poset, and the
$\mathbf{Y}_{h}\left(  \lambda/\mu\right)  $-partitions are precisely the
semistandard tableaux of shape $\lambda/\mu$.
\end{example}

We now assign a certain formal power series to every double poset:

\begin{definition}
\label{def.Gammaw} If ${\mathbf{E}} = \left(  E, <_{1}, <_{2}\right)  $ is a
double poset, and $w : E \to\left\{  1, 2, 3, \ldots\right\}  $ is a map, then
we define a power series $\Gamma\left(  {\mathbf{E}} , w\right)  \in
\mathbf{k}\left[  \left[  x_{1},x_{2},x_{3},\ldots\right]  \right]  $ by
\[
\Gamma\left(  {\mathbf{E}} , w\right)  = \sum_{\pi\text{ is an }{\mathbf{E}%
}\text{-partition}} {\mathbf{x}}_{\pi, w} , \qquad\text{where } {\mathbf{x}%
}_{\pi, w} = \prod_{e \in E} x_{\pi\left(  e\right)  }^{w\left(  e\right)  }
.
\]

\end{definition}

The following fact is easy to see:

\begin{proposition}
\label{prop.Gammaw.qsym} Let ${\mathbf{E}} = \left(  E, <_{1}, <_{2}\right)  $
be a double poset, and $w : E \to\left\{  1, 2, 3, \ldots\right\}  $ be a map.
Then, $\Gamma\left(  {\mathbf{E}} , w\right)  \in{\operatorname{QSym}}$.
\end{proposition}

\begin{example}
\label{exam.Gamma} Various well-known quasisymmetric functions
can be written as $\Gamma\left(  {\mathbf{E}} , w\right)  $:

\begin{enumerate}
\item[(a)] If ${\mathbf{E}} = \left(  E, <_{1}, <_{2}\right)  $ is a double
poset, and $w : E \to\left\{  1, 2, 3, \ldots\right\}  $ is the constant
function sending everything to $1$, then $\Gamma\left(  {\mathbf{E}} ,
w\right)  = \sum_{\pi\text{ is an }{\mathbf{E}}\text{-partition}} {\mathbf{x}%
}_{\pi}$, where ${\mathbf{x}}_{\pi} = \prod_{e \in E} x_{\pi\left(  e\right)
}$. We shall denote this power series $\Gamma\left(  {\mathbf{E}} , w\right)
$ by $\Gamma\left(  {\mathbf{E}}\right)  $; it is exactly what has been called
$\Gamma\left(  {\mathbf{E}}\right)  $ in \cite[�2.2]{Mal-Reu-DP}. All results
proven below for $\Gamma\left(  {\mathbf{E}} , w\right)  $ can be applied to
$\Gamma\left(  {\mathbf{E}}\right)  $, yielding simpler (but less general) statements.

\item[(b)] If $E = \left\{  1, 2, \ldots, \ell\right\}  $ for some $\ell
\in{\mathbb{N}}$, if $<_{1}$ is the usual total order inherited from
${\mathbb{Z}}$, and if $<_{2}$ is the opposite relation of $<_{1}$, then the
special double poset ${\mathbf{E}} = \left(  E, <_{1}, <_{2}\right)  $
satisfies $\Gamma\left(  {\mathbf{E}}, w\right)  = M_{\alpha}$, where $\alpha$
is the composition $\left(  w\left(  1\right)  , w\left(  2\right)  , \ldots,
w\left(  \ell\right)  \right)  $.

\item[(c)] Let $\alpha=\left(  \alpha_{1},\alpha_{2},\ldots,\alpha_{\ell
}\right)  $ be a composition, and set $n=\left|  \alpha\right|  $. Let
$D\left(  \alpha\right)  $ be the set $\left\{  \alpha_{1},\alpha_{1}%
+\alpha_{2},\alpha_{1}+\alpha_{2}+\alpha_{3},\ldots,\alpha_{1}+\alpha
_{2}+\cdots+\alpha_{\ell-1}\right\}  $. Let $E$ be the set $\left\{
1,2,\ldots,n\right\}  $, and let $<_{1}$ be the total order inherited on $E$
from ${\mathbb{Z}}$. Let $<_{2}$ be some partial order on $E$ with the
property that
$\left(
i+1<_{2}i\text{ for every }i\in D\left(  \alpha\right)
\right)$
and \newline
$\left(
i<_{2}i+1\text{ for every }i\in\left\{  1,2,\ldots,n-1\right\}
\setminus D\left(  \alpha\right)
\right)$.
(There are several choices for such an order; in particular, we can find one
which is a total order.) Then,
\[
\Gamma\left(  \left(  E,<_{1},<_{2}\right)  \right)  =\sum_{\substack{i_{1}%
\leq i_{2}\leq\cdots\leq i_{n};\\i_{j}<i_{j+1}\text{ whenever }j\in D\left(
\alpha\right)  }}x_{i_{1}}x_{i_{2}}\cdots x_{i_{n}}=\sum_{\substack{\beta
\text{ is a composition};\\\left\vert \beta\right\vert =n;\ D\left(
\beta\right)  \supseteq D\left(  \alpha\right)  }}M_{\beta}.
\]
This power series is known as the $\alpha$-th \textit{fundamental
quasisymmetric function}, usually called $F_{\alpha}$ (in
\cite{Gessel}, \cite[\S 2]{Mal-Reu-dua}, \cite[�2.4]{BBSSZ}
and \cite[�2]{Gri-dimm}) or $L_{\alpha}$ (in \cite[�7.19]{Stanley-EC2} or
\cite[Def. 5.15]{Reiner}).

\item[(d)] Let ${\mathbf{E}}$ be one of the two double posets $\mathbf{Y}%
\left(  \lambda/ \mu\right)  $ and $\mathbf{Y}_{h}\left(  \lambda/ \mu\right)
$ defined as in Example \ref{exam.dp} for two partitions $\mu$ and $\lambda$.
Then, $\Gamma\left(  {\mathbf{E}}\right)  $ is the skew Schur function
$s_{\lambda/ \mu}$.

\item[(e)] Similarly, \textit{dual immaculate functions} as defined in
\cite[�3.7]{BBSSZ} can be realized as $\Gamma\left(  {\mathbf{E}}\right)  $
for appropriate ${\mathbf{E}}$ (see \cite[Proposition 4.4]{Gri-dimm}), which
helped the author prove one of their properties \cite{Gri-dimm}. (The
${\mathbf{E}}$-partitions here are the so-called \textit{immaculate tableaux}.)

\item[(f)] When the relation $<_{2}$ of a double poset ${\mathbf{E}}=\left(
E,<_{1},<_{2}\right)  $ is a total order (i.e., when the double poset
${\mathbf{E}}$ is special), the ${\mathbf{E}}$-partitions are precisely the
reverse $\left(  P,\omega\right)  $-partitions (for $P=\left(  E,<_{1}\right)
$ and $\omega$ being a labelling of $P$ dictated by $<_{2}$) in the
terminology of \cite[�7.19]{Stanley-EC2}, and the power series $\Gamma\left(
{\mathbf{E}}\right)  $ is the $K_{P,\omega}$ of \cite[�7.19]{Stanley-EC2}.
\end{enumerate}
\end{example}

\section{The antipode theorem}

We are now ready for the main results. We first state a theorem and a
corollary which are not new, but will be reproven in a novel and
self-contained way.

\begin{theorem}
\label{thm.antipode.Gammaw} Let $\left(  E, <_{1}, <_{2}\right)  $ be a
tertispecial double poset. Let $w : E \to\left\{  1, 2, 3, \ldots\right\}  $.
Then, $S\left(  \Gamma\left(  \left(  E, <_{1}, <_{2}\right)  , w\right)
\right)  = \left(  -1\right)  ^{\left|  E\right|  } \Gamma\left(  \left(  E,
>_{1}, <_{2}\right)  , w\right)  $, where $>_{1}$ denotes the opposite
relation of $<_{1}$.
\end{theorem}

\begin{corollary}
\label{cor.antipode.Gamma} Let $\left(  E, <_{1}, <_{2}\right)  $ be a
tertispecial double poset. Then, $S\left(  \Gamma\left(  \left(  E, <_{1},
<_{2}\right)  \right)  \right)  = \left(  -1\right)  ^{\left|  E\right|  }
\Gamma\left(  \left(  E, >_{1}, <_{2}\right)  \right)  $, where $>_{1}$
denotes the opposite relation of $<_{1}$.
\end{corollary}

We shall give examples for consequences of these facts shortly
(Example~\ref{exam.antipode.Gammaw}), but let us first explain where they have
already appeared. Corollary~\ref{cor.antipode.Gamma} is equivalent to
\cite[Corollary 5.27]{Reiner} (a result found by Malvenuto and Reutenauer
\cite[Lemma 3.2]{Mal-Reu}). Theorem~\ref{thm.antipode.Gammaw}
is equivalent to Malvenuto's and Reutenauer's \cite[Theorem 3.1]%
{Mal-Reu}.\footnote{These equivalences are not totally obvious. See
\cite[�4]{Grinbe16} for a few more details on them.} We believe
that our versions of these facts are more natural and simpler than the ones
appearing in existing literature (and if not, at least our proofs are).

To these known results, we add another, which seems to be unknown so far
(probably because it is far harder to state in the terminologies of $\left(
P, \omega\right)  $-partitions or equality-and-inequality conditions appearing
in literature). First, we need to introduce some notation:

\begin{definition}
\label{def.G-sets.terminology} Let $G$ be a group, and let $E$ be a $G$-set.

\begin{itemize}
\item[(a)] Let $<$ be a strict partial order on $E$. We say that $G$
\textit{preserves the relation $<$} if every $g\in G$, $a\in E$ and $b\in E$
satisfying $a<b$ satisfy $ga<gb$.

\item[(b)] Let $w : E \to\left\{  1, 2, 3, \ldots\right\}  $. We say that $G$
\textit{preserves $w$} if every $g \in G$ and $e \in E$ satisfy $w\left(
ge\right)  = w\left(  e\right)  $.

\item[(c)] Let $g \in G$. Assume that the set $E$ is finite. We say that $g$
is \textit{$E$-even} if the action of $g$ on $E$ (that is, the permutation of
$E$ that sends every $e \in E$ to $ge$) is an even permutation of $E$.

\item[(d)] If $X$ is any set, then the set $X^{E}$ of all maps $E\rightarrow
X$ becomes a $G$-set as follows: For any $\pi\in X^{E}$ and $g\in G$, we let
$g\pi\in X^{E}$ be the map sending each $e\in E$ to $\pi\left(  g^{-1}%
e\right)  $.

\item[(e)] Let $F$ be a further $G$-set. Assume that the set $E$ is finite. An
element $\pi\in F$ is said to be \textit{$E$-coeven} if every $g\in G$
satisfying $g\pi=\pi$ is $E$-even. A $G$-orbit $O$ on $F$ is said to be
\textit{$E$-coeven} if all elements of $O$ are $E$%
-coeven.\footnote{Equivalently, $O$ is $E$-coeven if and only if at least one
element of $O$ is $E$-coeven. (This is easy to check.)}
\end{itemize}
\end{definition}

\begin{theorem}
\label{thm.antipode.GammawG} Let ${\mathbf{E}} = \left(  E, <_{1},
<_{2}\right)  $ be a tertispecial double poset. Let $\operatorname{Par}
{\mathbf{E}}$ denote the set of all ${\mathbf{E}}$-partitions. Let $w : E
\to\left\{  1, 2, 3, \ldots\right\}  $. Let $G$ be a finite group which acts
on $E$. Assume that $G$ preserves both relations $<_{1}$ and $<_{2}$, and also
preserves $w$. Then, $G$ acts also on the set $\operatorname{Par} {\mathbf{E}%
}$ of all ${\mathbf{E}}$-partitions; namely, $\operatorname{Par} {\mathbf{E}}$
is a $G$-subset of the $G$-set $\left\{  1, 2, 3, \ldots\right\}  ^{E}$ (see
Definition~\ref{def.G-sets.terminology} (d) for the definition of the
latter).
%We say that an $\EE$-partition $\pi$ is
%\textit{even} if every $g \in G$ satisfying $g \pi = \pi$
%is $E$-even. We say that a $G$-orbit $O$ on
%$\Par \EE$ is \textit{even} if its elements are
%even (or, equivalently, one of its elements is even).
For any $G$-orbit $O$ on $\operatorname{Par}{\mathbf{E}}$, we define a
monomial ${\mathbf{x}}_{O, w}$ by
\[
{\mathbf{x}}_{O, w} = {\mathbf{x}}_{\pi, w} \qquad\text{ for some element }
\pi\text{ of } O
\]
(this does not depend on the choice of $\pi$). Let
\[
\Gamma\left(  {\mathbf{E}}, w, G\right)  = \sum_{O\text{ is a } G\text{-orbit
on } \operatorname{Par}{\mathbf{E}}} {\mathbf{x}}_{O, w}
\]
and
\[
\Gamma^{+}\left(  {\mathbf{E}}, w, G\right)  = \sum_{O\text{ is an
}E\text{-coeven } G\text{-orbit on } \operatorname{Par}{\mathbf{E}}}
{\mathbf{x}}_{O, w} .
\]
Then, $\Gamma\left(  {\mathbf{E}}, w, G\right)  $ and $\Gamma^{+}\left(
{\mathbf{E}}, w, G\right)  $ belong to ${\operatorname{QSym}}$ and satisfy
\[
S\left(  \Gamma\left(  {\mathbf{E}}, w, G\right)  \right)  = \left(
-1\right)  ^{\left|  E\right|  } \Gamma^{+}\left(  \left(  E, >_{1},
<_{2}\right)  , w, G\right)  .
\]

\end{theorem}

This theorem, which combines Theorem~\ref{thm.antipode.Gammaw} with the ideas
of P\'olya enumeration, is inspired by Jochemko's reciprocity result for order
polynomials \cite[Theorem 2.8]{Joch}, which can be obtained from it by
specializations (see \cite[�8]{Grinbe16} for the derivation).

We shall now review a number of particular cases of
Theorem~\ref{thm.antipode.Gammaw}.

\begin{example}
\label{exam.antipode.Gammaw}

\begin{enumerate}
\item[(a)] Corollary~\ref{cor.antipode.Gamma} follows from
Theorem~\ref{thm.antipode.Gammaw} by letting $w$ be the function which is
constantly $1$.

\item[(b)] Let $\alpha=\left(  \alpha_{1},\alpha_{2},\ldots,\alpha_{\ell
}\right)  $ be a composition, let $n=\left\vert \alpha\right\vert $, and let
${\mathbf{E}}=\left(  E,<_{1},<_{2}\right)  $ be the double poset defined in
Example~\ref{exam.Gamma} (b). Let $w:\left\{  1,2,\ldots,\ell\right\}
\rightarrow\left\{  1,2,3,\ldots\right\}  $ be the map sending every $i$ to
$\alpha_{i}$. As Example~\ref{exam.Gamma} (b) shows, we have $\Gamma\left(
{\mathbf{E}},w\right)  =M_{\alpha}$. Thus, applying
Theorem~\ref{thm.antipode.Gammaw} to these ${\mathbf{E}}$ and $w$ yields
\begin{align*}
S\left(  M_{\alpha}\right)   &  =\left(  -1\right)  ^{\ell}\Gamma\left(
\left(  E,>_{1},<_{2}\right)  ,w\right)  =\left(  -1\right)  ^{\ell}%
\sum_{i_{1}\geq i_{2}\geq\cdots\geq i_{\ell}}x_{i_{1}}^{\alpha_{1}}x_{i_{2}%
}^{\alpha_{2}}\cdots x_{i_{\ell}}^{\alpha_{\ell}}\\
&  =\left(  -1\right)  ^{\ell}\sum_{i_{1}\leq i_{2}\leq\cdots\leq i_{\ell}%
}x_{i_{1}}^{\alpha_{\ell}}x_{i_{2}}^{\alpha_{\ell-1}}\cdots x_{i_{\ell}%
}^{\alpha_{1}}=\left(  -1\right)  ^{\ell}\sum_{\substack{\gamma\text{ is a
composition};\ \left\vert \gamma\right\vert =n;\\D\left(  \gamma\right)
\subseteq D\left(  \left(  \alpha_{\ell},\alpha_{\ell-1},\ldots,\alpha
_{1}\right)  \right)  }}M_{\gamma}.
\end{align*}
This is the formula for $S\left(  M_{\alpha}\right)  $ given in
\cite[Proposition 3.4]{Ehrenb96}, in \cite[(4.26)]{Malve-Thesis},
in \cite[Theorem 5.11]{Reiner}, and in \cite[Theorem
4.1]{BenSag} (originally due to Ehrenborg and to Malvenuto and Reutenauer).

\item[(c)] Applying Corollary~\ref{cor.antipode.Gamma} to the double poset of
Example~\ref{exam.Gamma} (c) (where the relation $<_{2}$ is chosen to be a
total order) yields a classical formula for the antipode of a fundamental
quasisymmetric function (\cite[(4.27)]{Malve-Thesis}, \cite[(5.9)]{Reiner},
\cite[Theorem 5.1]{BenSag}).

\item[(d)] By applying Corollary~\ref{cor.antipode.Gamma} to any of
the two tertispecial double posets
$\mathbf{Y}\left(  \lambda/ \mu\right)  $ and
$\mathbf{Y}_h\left(  \lambda/ \mu\right)  $
from Example~\ref{exam.dp}, we can obtain the
well-known formula $S\left(  s_{\lambda/ \mu}\right)  = \left(  -1\right)  ^{\left|  \lambda/
\mu\right|  } s_{\lambda^{t} / \mu^{t}}$ for the antipode of a skew
Schur function (where $\nu^t$ denotes the conjugate of a partition
$\nu$). See, e.g., \cite[Example 4.8 (d)]{Grinbe16} for the
details. (This is not a new argument; it appeared, e.g., in \cite[proof of
Corollary 5.29]{Reiner} in the language of $P$-partitions. It makes
use of the fact that the antipode of the symmetric functions is a
restriction of the antipode of $\QSym$.)
A more general antipode formula for ``Schur functions
with cell weights'' (no longer symmetric, at least in general)
can be obtained using Theorem~\ref{thm.antipode.Gammaw}.
% Let us use the notations of Example~\ref{exam.dp}. For any
% partition $\lambda$, let $\lambda^{t}$ denote the conjugate partition of
% $\lambda$. Let $\mu$ and $\lambda$ be two partitions satisfying $\mu
% \subseteq\lambda$. Then, there is a bijection $\tau: Y\left(  \lambda/
% \mu\right)  \to Y\left(  \lambda^{t} / \mu^{t}\right)  $ sending each $\left(
% i, j\right)  \in Y\left(  \lambda/ \mu\right)  $ to $\left(  j, i\right)  $.
% This bijection is an isomorphism of double posets from $\left(  Y\left(
% \lambda/ \mu\right)  , >_{1}, <_{2}\right)  $ to $\left(  Y\left(  \lambda^{t}
% / \mu^{t}\right)  , >_{1}, >_{2}\right)  $. Thus, applying
% Corollary~\ref{cor.antipode.Gamma} to the tertispecial double poset
% $\mathbf{Y}\left(  \lambda/ \mu\right)  $, we obtain
% \begin{align}
% S\left(  \Gamma\left(  \mathbf{Y}\left(  \lambda/ \mu\right)  \right)
% \right)   &  = \left(  -1\right)  ^{\left|  \lambda/ \mu\right|  }
% \Gamma\left(  \left(  Y\left(  \lambda/ \mu\right)  , >_{1}, <_{2}\right)
% \right) \nonumber\\
% &  = \left(  -1\right)  ^{\left|  \lambda/ \mu\right|  } \Gamma\left(  \left(
% Y\left(  \lambda^{t} / \mu^{t}\right)  , >_{1}, >_{2}\right)  \right)  .
% \label{eq.exam.antipode.Gammaw.schur.1}%
% \end{align}
% But from Example~\ref{exam.Gamma} (d), we know that $\Gamma\left(
% \mathbf{Y}\left(  \lambda/ \mu\right)  \right)  = s_{\lambda/ \mu}$. Moreover,
% a similar argument using \cite[Remark 2.12]{Reiner} shows that $\Gamma\left(
% \left(  Y\left(  \lambda^{t} / \mu^{t}\right)  , >_{1}, >_{2}\right)  \right)
% = s_{\lambda^{t} / \mu^{t}}$. Hence, \eqref{eq.exam.antipode.Gammaw.schur.1}
% rewrites as
% \begin{equation}
% S\left(  s_{\lambda/ \mu}\right)  = \left(  -1\right)  ^{\left|  \lambda/
% \mu\right|  } s_{\lambda^{t} / \mu^{t}} .
% \label{eq.exam.antipode.Gammaw.schur.2}%
% \end{equation}
% This is a well-known formula, and is usually stated for $S$ being the antipode
% of the Hopf algebra of symmetric (rather than quasisymmetric) functions, but
% the latter antipode is a restriction of the antipode of ${\operatorname{QSym}%
% }$.

% It is also possible (but more difficult) to derive
% \eqref{eq.exam.antipode.Gammaw.schur.2} by using the double poset
% $\mathbf{Y}_{h}\left(  \lambda/ \mu\right)  $ instead of $\mathbf{Y}\left(
% \lambda/ \mu\right)  $. (This boils down to what was done in \cite[proof of
% Corollary 5.29]{Reiner}.)

\item[(e)] A result of Benedetti and Sagan \cite[Theorem 8.2]{BenSag} on the
antipodes of immaculate functions can be obtained from
Corollary~\ref{cor.antipode.Gamma} using dualization.
\end{enumerate}
\end{example}

\section{An outline of the proofs}

In preparation for the proofs of the above results, we
shall now introduce the notion of a \textit{packed map}, and state some
simple lemmas. Proofs can be found in \cite[�5]{Grinbe16}.

\begin{definition}
If $E$ is a set and $\pi:E\rightarrow\left\{  1,2,3,\ldots\right\}  $ is a
map, then $\pi$ is said to be \textit{packed} if $\pi\left(  E\right)
=\left\{  1,2,\ldots,k\right\}  $ for some $k\in\mathbb{N}$.
\end{definition}

\begin{definition}
Let $E$ be a set. Let $\pi:E\rightarrow\left\{  1,2,3,\ldots\right\}  $ be a
packed map. Let $w:E\rightarrow\left\{  1,2,3,\ldots\right\}  $ be a map.
Then, a composition $\operatorname{ev}_{w}\pi$ is defined as follows: Let
$\ell=\left\vert \pi\left(  E\right)  \right\vert $. Set $\operatorname{ev}%
_{w}\pi=\left(  \alpha_{1},\alpha_{2},\ldots,\alpha_{\ell}\right)  $, where
each $\alpha_{i}=\sum_{e\in\pi^{-1}\left(  i\right)  }w\left(  e\right)  $.
\end{definition}

\begin{proposition}
\label{prop.Gammaw.packed} Let ${\mathbf{E}}=\left(  E,<_{1},<_{2}\right)  $
be a double poset. Let $w:E\rightarrow\left\{  1,2,3,\ldots\right\}  $ be a
map. Then,
\begin{equation}
\Gamma\left(  {\mathbf{E}},w\right)  =\sum_{\varphi\text{ is a packed
}{\mathbf{E}}\text{-partition}}M_{\operatorname{ev}_{w}\varphi}.
\label{eq.prop.Gammaw.packed}%
\end{equation}

\end{proposition}

We shall now describe the coproduct of $\Gamma\left(  {\mathbf{E}},w\right)
$, following \cite[Theorem 2.2]{Mal-Reu-DP}.

\begin{definition}
Let ${\mathbf{E}} = \left(  E, <_{1}, <_{2}\right)  $ be a double poset.

\begin{itemize}
\item[(a)] Then, $\operatorname{Adm}{\mathbf{E}}$ will mean the set of all
pairs $\left(  P,Q\right)  $, where $P$ and $Q$ are subsets of $E$ satisfying
$P\cap Q=\varnothing$ and $P\cup Q=E$ and having the property that no $p\in P$
and $q\in Q$ satisfy $q<_{1}p$. These pairs $\left(  P,Q\right)  $ are called
the \textit{admissible partitions} of ${\mathbf{E}}$.

\item[(b)] For any subset $T$ of $E$, we let ${\mathbf{E}}\mid_{T}$ denote the
double poset $\left(  T, <_{1}, <_{2}\right)  $, where $<_{1}$ and $<_{2}$ (by
abuse of notation) denote the restrictions of the relations $<_{1}$ and
$<_{2}$ to $T$.
\end{itemize}
\end{definition}

\begin{proposition}
\label{prop.Gammaw.coprod} Let ${\mathbf{E}} = \left(  E, <_{1}, <_{2}\right)
$ be a double poset. Let $w : E \to\left\{  1, 2, 3, \ldots\right\}  $ be a
map. Then,
\begin{equation}
\label{eq.prop.Gammaw.coprod}\Delta\left(  \Gamma\left(  {\mathbf{E}},
w\right)  \right)  = \sum_{\left(  P, Q\right)  \in\operatorname{Adm}
{\mathbf{E}}} \Gamma\left(  {\mathbf{E}}\mid_{P}, w\mid_{P}\right)
\otimes\Gamma\left(  {\mathbf{E}}\mid_{Q}, w\mid_{Q}\right)  .
\end{equation}

\end{proposition}

\begin{proof}
[Proof outline for Theorem~\ref{thm.antipode.Gammaw}.]We shall only
demonstrate the cornerstones of this proof. See \cite[�6]{Grinbe16} for the details.

We use strong induction over $\left\vert E\right\vert $. The induction base
($\left\vert E\right\vert =0$) is straightforward. Now, consider a
tertispecial double poset ${\mathbf{E}}=\left(  E,<_{1},<_{2}\right)  $ with
$\left\vert E\right\vert >0$ and a map $w:E\rightarrow\left\{  1,2,3,\ldots
\right\}  $, and assume that Theorem~\ref{thm.antipode.Gammaw} is proven for
all tertispecial double posets of smaller size.

From $\left\vert E\right\vert >0$, it is easy to see that $\varepsilon\left(
\Gamma\left(  {\mathbf{E}},w\right)  \right)  =0$, so that $\left(
u\circ\varepsilon\right)  \left(  \Gamma\left(  {\mathbf{E}},w\right)
\right)  =0$.

But \eqref{eq.antipode} yields $\left(  m\circ\left(  S\otimes
{\operatorname{id}}\right)  \circ\Delta\right)  \left(  \Gamma\left(
{\mathbf{E}},w\right)  \right)  =\left(  u\circ\varepsilon\right)  \left(
\Gamma\left(  {\mathbf{E}},w\right)  \right)  =0$, so that%
\begin{align}
0  &  =\left(  m\circ\left(  S\otimes{\operatorname{id}}\right)  \circ
\Delta\right)  \left(  \Gamma\left(  {\mathbf{E}},w\right)  \right)  =m\left(
\left(  S\otimes{\operatorname{id}}\right)  \left(  \Delta\left(
\Gamma\left(  {\mathbf{E}},w\right)  \right)  \right)  \right) \nonumber\\
&  =m\left(  \left(  S\otimes{\operatorname{id}}\right)  \left(  \sum_{\left(
P,Q\right)  \in\operatorname{Adm}{\mathbf{E}}}\Gamma\left(  {\mathbf{E}}%
\mid_{P},w\mid_{P}\right)  \otimes\Gamma\left(  {\mathbf{E}}\mid_{Q},w\mid
_{Q}\right)  \right)  \right)  \qquad\left(  \text{by
\eqref{eq.prop.Gammaw.coprod}}\right) \nonumber\\
&  =\sum_{\left(  P,Q\right)  \in\operatorname{Adm}{\mathbf{E}}}S\left(
\Gamma\left(  {\mathbf{E}}\mid_{P},w\mid_{P}\right)  \right)  \Gamma\left(
{\mathbf{E}}\mid_{Q},w\mid_{Q}\right)  . \label{pf.thm.antipode.Gammaw.Req.1a}%
\end{align}
In order to prove Theorem~\ref{thm.antipode.Gammaw}, it now suffices to
verify
\begin{equation}
0=\sum_{\left(  P,Q\right)  \in\operatorname{Adm}{\mathbf{E}}}\left(
-1\right)  ^{\left\vert P\right\vert }\Gamma\left(  \left(  P,>_{1}%
,<_{2}\right)  ,w\mid_{P}\right)  \Gamma\left(  {\mathbf{E}}\mid_{Q},w\mid
_{Q}\right)  . \label{pf.thm.antipode.Gammaw.Req}%
\end{equation}
Indeed, each addend on the right hand side of
(\ref{pf.thm.antipode.Gammaw.Req.1a}) equals the corresponding addend on the
right hand side of (\ref{pf.thm.antipode.Gammaw.Req}) except maybe the addend
for $\left(  P,Q\right)  =\left(  E,\varnothing\right)  $%
\ \ \ \ \footnote{because if $\left(  P,Q\right)  \neq\left(  E,\varnothing
\right)  $, then $\left|P\right| < \left|E\right|$, and thus
the induction hypothesis (applied to the double poset
${\mathbf{E}}\mid_{P}$, which is easily seen to be tertispecial) yields
$S\left(  \Gamma\left(  {\mathbf{E}}\mid_{P},w\mid_{P}\right)  \right)
=\left(  -1\right)  ^{\left\vert P\right\vert }\Gamma\left(  \left(
P,>_{1},<_{2}\right)  ,w\mid_{P}\right)  $}. Therefore, once
(\ref{pf.thm.antipode.Gammaw.Req}) is proven, it will follow that the addends
for $\left(  P,Q\right)  =\left(  E,\varnothing\right)  $ are also equal; but
this is precisely the claim $S\left(  \Gamma\left(  \mathbf{E},w\right)
\right)  =\left(  -1\right)  ^{\left\vert E\right\vert }\Gamma\left(  \left(
E,>_{1},<_{2}\right)  ,w\right)  $ that needs to be proven. Hence, proving
(\ref{pf.thm.antipode.Gammaw.Req}) suffices.

Using the definitions of $\Gamma\left(  \left(  P,>_{1},<_{2}\right)
,w\mid_{P}\right)  $ and $\Gamma\left(  {\mathbf{E}}\mid_{Q},w\mid_{Q}\right)
$, we observe that each $\left(  P,Q\right)  \in\operatorname*{Adm}\mathbf{E}$
satisfies%
\begin{align*}
&  \Gamma\left(  \left(  P,>_{1},<_{2}\right)  ,w\mid_{P}\right)
\Gamma\left(  {\mathbf{E}}\mid_{Q},w\mid_{Q}\right) \\
&  =\left(  \sum_{\sigma\text{ is a }\left(  P,>_{1},<_{2}\right)
\text{-partition}}{\mathbf{x}}_{\sigma,w\mid_{P}}\right)  \left(  \sum
_{\tau\text{ is a }\left(  Q,<_{1},<_{2}\right)  \text{-partition}}%
{\mathbf{x}}_{\tau,w\mid_{Q}}\right) \\
&  =\sum_{\substack{\pi:E\rightarrow\left\{  1,2,3,\ldots\right\}  ;\\\pi
\mid_{P}\text{ is a }\left(  P,>_{1},<_{2}\right)  \text{-partition;}\\\pi
\mid_{Q}\text{ is a }\left(  Q,<_{1},<_{2}\right)  \text{-partition}%
}}{\mathbf{x}}_{\pi,w}.
\end{align*}
Therefore, in order to prove (\ref{pf.thm.antipode.Gammaw.Req}), it will be
enough to show that for every map $\pi:E\rightarrow\left\{  1,2,3,\ldots
\right\}  $, we have
\begin{equation}
\sum_{\substack{\left(  P,Q\right)  \in\operatorname{Adm}{\mathbf{E}}%
;\\\pi\mid_{P}\text{ is a }\left(  P,>_{1},<_{2}\right)  \text{-partition;}%
\\\pi\mid_{Q}\text{ is a }\left(  Q,<_{1},<_{2}\right)  \text{-partition}%
}}\left(  -1\right)  ^{\left\vert P\right\vert }=0.
\label{pf.thm.antipode.Gammaw.signrev}%
\end{equation}


Hence, let us fix a map $\pi:E\rightarrow\left\{  1,2,3,\ldots\right\}  $. Our
goal is now to prove \eqref{pf.thm.antipode.Gammaw.signrev}. We denote by $Z$
the set of all $\left(  P,Q\right)  \in\operatorname{Adm}{\mathbf{E}}$ such
that $\pi\mid_{P}$ is a $\left(  P,>_{1},<_{2}\right)  $-partition and
$\pi\mid_{Q}$ is a $\left(  Q,<_{1},<_{2}\right)  $-partition. We are going to
define an involution $T:Z\rightarrow Z$ of the set $Z$ having the property
that, for any $\left(  P,Q\right)  \in Z$, if we write $T\left(  \left(
P,Q\right)  \right)  $ in the form $\left(  P^{\prime},Q^{\prime}\right)  $,
then $\left(  -1\right)  ^{\left\vert P^{\prime}\right\vert }=-\left(
-1\right)  ^{\left\vert P\right\vert }$. Once such an involution $T$ is found,
it will clearly partition the addends on the left hand side of
\eqref{pf.thm.antipode.Gammaw.signrev} into pairs of mutually cancelling
addends, and so \eqref{pf.thm.antipode.Gammaw.signrev} will follow and we will
be done. It thus remains to find $T$.

The definition of $T$ is simple: Let $F$ be the subset of $E$ consisting of
those $e\in E$ which have minimum $\pi\left(  e\right)  $. Then, $F$ is a
nonempty subposet of the poset $\left(  E,<_{2}\right)  $, and hence has a
minimal element $f$ (that is, an element $f$ such that no $g\in F$ satisfies
$g<_{2}f$). Fix such an $f$. Now, the map $T$ sends a $\left(  P,Q\right)  \in
Z$ to $%
\begin{cases}
\left(  P\cup\left\{  f\right\}  ,Q\setminus\left\{  f\right\}  \right)  , &
\text{if }f\notin P;\\
\left(  P\setminus\left\{  f\right\}  ,Q\cup\left\{  f\right\}  \right)  , &
\text{if }f\in P
\end{cases}
$.

Less simple is the proof that $T$ is well-defined. See \cite[�6]{Grinbe16} for this argument.
\end{proof}

We shall be particularly brief about the proof of
Theorem~\ref{thm.antipode.GammawG}; the full proof can be found in
\cite[�7]{Grinbe16}. We merely state the two main observations used in the proof:

\begin{proposition}
\label{prop.G-poset.quot.Phi} Let ${\mathbf{E}}=\left(  E,<_{1},<_{2}\right)
$ be a tertispecial double poset. Let $G$ be a finite group which acts on $E$.
Assume that $G$ preserves both relations $<_{1}$ and $<_{2}$.

Let $g\in G$. Let $E^{g}$ be the set of all orbits under the action of $g$ on
$E$. Define a binary relation $<_{1}^{g}$ on $E^{g}$ by
\[
\left(  u<_{1}^{g}v\right)  \Longleftrightarrow\left(  \text{there exist }a\in
u\text{ and }b\in v\text{ with }a<_{1}b\right)  .
\]
Define a binary relation $<_{2}^{g}$ similarly. Set ${\mathbf{E}}^{g}=\left(
E^{g},<_{1}^{g},<_{2}^{g}\right)  $.

\begin{enumerate}
\item[(a)] Then, ${\mathbf{E}}^{g}$ is a tertispecial double poset.
\end{enumerate}

There is a bijection $\Phi:\left\{  \pi:E\rightarrow\left\{  1,2,3,\ldots
\right\}  \ \mid\ g\pi=\pi\right\}  \rightarrow\left\{  \overline{\pi}%
:E^{g}\rightarrow\left\{  1,2,3,\ldots\right\}  \right\}  $. Namely, this
bijection $\Phi$ sends any
map $\pi:E\rightarrow\left\{  1,2,3,\ldots\right\}
$ satisfying $g\pi=\pi$
to the map $\overline{\pi}:E^{g}\rightarrow\left\{
1,2,3,\ldots\right\}  $ defined by
\[
\overline{\pi}\left(  u\right)  =\pi\left(  a\right)  \qquad\text{for every
}u\in E^{g}\text{ and }a\in u.
\]
Consider this bijection $\Phi$. Let $\pi:E\rightarrow\left\{  1,2,3,\ldots
\right\}  $ be a map satisfying $g\pi=\pi$.

\begin{enumerate}
\item[(b)] The map $\pi$ is an ${\mathbf{E}}$-partition if and only if the map
$\Phi\left(  \pi\right)  $ is an ${\mathbf{E}}^{g}$-partition.

\item[(c)] Let $w:E\rightarrow\left\{  1,2,3,\ldots\right\}  $ be map. Define
a map $w^{g}:E^{g}\rightarrow\left\{  1,2,3,\ldots\right\}  $ by
\[
w^{g}\left(  u\right)  =\sum\limits_{a\in u}w\left(  a\right)  \qquad
\qquad\text{ for every }u\in E^{g}.
\]
Then, $\mathbf{x}_{\Phi\left(  \pi\right)  ,w^{g}}=\mathbf{x}_{\pi,w}$.
\end{enumerate}
\end{proposition}

\begin{lemma}
\label{lem.burnside.sums} Let $G$ be a finite group. Let $F$ be a
$G$-set. Let $O$ be a $G$-orbit on $F$, and let $\pi\in O$.

\begin{enumerate}
\item[(a)] We have $\dfrac{1}{\left\vert O\right\vert }=\dfrac{1}{\left\vert
G\right\vert }\sum_{\substack{g\in G;\\g\pi=\pi}}1$.

\item[(b)] Let $E$ be a further finite $G$-set. For every $g\in G$, let
$\operatorname{sign}_{E}g$ denote the sign of the permutation of $E$ that
sends every $e\in E$ to $ge$. (Thus, $g\in G$ is $E$-even if and only if
$\operatorname*{sign}\nolimits_{E}g=1$.) Then, $%
\begin{cases}
\dfrac{1}{\left\vert O\right\vert }, & \text{if }O\text{ is }E\text{-coeven}%
;\\
0, & \text{if }O\text{ is not }E\text{-coeven}%
\end{cases}
=\dfrac{1}{\left\vert G\right\vert }\sum_{\substack{g\in G;\\g\pi=\pi
}}\operatorname*{sign}\nolimits_{E}g$.
\end{enumerate}
\end{lemma}

Theorem~\ref{thm.antipode.GammawG} can be derived from
Theorem~\ref{thm.antipode.Gammaw} using the above observations and some
standard manipulations of sums, akin to the proof of the P�lya enumeration formula.

\nocite{*}
%use the following instead if you encounter problems
%\bibliographystyle{alpha}
\bibliographystyle{abbrvnat}
\bibliography{sample}
\label{sec:biblio}

\begin{thebibliography}{99}                                                                                               %


\bibitem {BenSag}Carolina Benedetti, Bruce Sagan, \textit{Antipodes and
involutions}, \href{http://arxiv.org/abs/1410.5023v4}{arXiv:1410.5023v4}.

\bibitem {BBSSZ}Chris Berg, Nantel Bergeron, Franco Saliola, Luis Serrano,
Mike Zabrocki, \href{http://dx.doi.org/10.4153/CJM-2013-013-0}{\textit{A lift of the Schur and Hall-Littlewood bases to
non-commutative symmetric functions}}, Canadian Journal of Mathematics 66
(2014), pp. 525--565. (Preprint:
\href{https://arxiv.org/abs/1208.5191v3}{arXiv:1208.5191v3}.)

\bibitem{Ehrenb96}Richard Ehrenborg,
\href{http://dx.doi.org/10.1006/aima.1996.0026}{\textit{On Posets and Hopf Algebras}},
Adv. in Maths. \textbf{119} (1996), pp. 1--25.
%\newline\url{http://dx.doi.org/10.1006/aima.1996.0026}

\bibitem{Foissy13}
Lo\"ic Foissy,
\textit{Plane posets, special posets, and permutations},
Advances in Mathematics 240 (2013), pp. 24--60.
(Preprint: \href{https://arxiv.org/abs/1109.1101v3}{arXiv:1109.1101v3}.)

\bibitem{Gessel}Ira M. Gessel, \href{http://people.brandeis.edu/~gessel/homepage/papers/multipartite.pdf}{\textit{Multipartite P-partitions and Inner
Products of Skew Schur Functions}}, Contemporary Mathematics, vol. 34, 1984,
pp. 289--301.

\bibitem {Gessel-Ppar}Ira M. Gessel, \textit{A Historical Survey of
$P$-Partitions},
% to be published in Richard Stanley's 70th Birthday
% Festschrift,
\href{http://arxiv.org/abs/1506.03508v1}{arXiv:1506.03508v1}.

\bibitem {Gri-dimm}Darij Grinberg, \textit{Dual immaculate creation operators
and a dendriform algebra structure on the quasisymmetric functions},
\href{http://arxiv.org/abs/1410.0079v6}{arXiv:1410.0079v6}.

\bibitem {Reiner}Darij Grinberg, Victor Reiner, \href{http://web.mit.edu/~darij/www/algebra/HopfComb.pdf}{\textit{Hopf algebras in
Combinatorics}},
% November 12, 2016.
% (Stable version:
\href{https://arxiv.org/abs/1409.8356v4}{arXiv:1409.8356v4}.

\bibitem {Grinbe16}Darij Grinberg, \href{http://www.cip.ifi.lmu.de/~grinberg/algebra/dp-abstr.pdf}{\textit{Double posets and the antipode of
}$\operatorname*{QSym}$}, %version 2.5 (February 19, 2017). (Stable version:
\href{https://arxiv.org/abs/1509.08355v3}{arXiv:1509.08355v3}.

\bibitem {Joch}Katharina Jochemko, \href{http://www.combinatorics.org/ojs/index.php/eljc/article/view/v21i2p52}{\textit{Order polynomials and P�lya's
enumeration theorem}}, The Electronic Journal of Combinatorics 21(2) (2014),
P2.52.

\bibitem {Malve-Thesis}Claudia Malvenuto, \href{http://www1.mat.uniroma1.it/people/malvenuto/Thesis.pdf}{\textit{Produits et coproduits des
fonctions quasi-sym�triques et de l'alg\`ebre des descentes}}, thesis, defended November 1993.

\bibitem{Mal-Reu-dua}
Claudia Malvenuto, Christophe Reutenauer,
\href{http://dx.doi.org/10.1006/jabr.1995.1336}{\textit{Duality between Quasi-Symmetric Functions and the Solomon
Descent Algebra}},
Journal of Algebra \textbf{177} (1995), pp. 967--982.

\bibitem {Mal-Reu}Claudia Malvenuto, Christophe Reutenauer, \href{http://www.sciencedirect.com/science/article/pii/S0012365X98001423}{\textit{Plethysm
and conjugation of quasi-symmetric functions}}, Discrete Mathematics,
193 (1998), pp. 225--233.

\bibitem {Mal-Reu-DP}Claudia Malvenuto, Christophe Reutenauer, \href{http://www.sciencedirect.com/science/article/pii/S0097316510001652}{\textit{A self
paired Hopf algebra on double posets and a Littlewood-Richardson rule}},
JCTA 118 (2011), pp. 1322--1333.

\bibitem {Manchon-HA}Dominique Manchon, \textit{Hopf algebras, from basics to
applications to renormalization}, \href{http://arxiv.org/abs/math/0408405v2}{arXiv:math/0408405v2}.

\bibitem {Montg-Hopf}Susan Montgomery, \textit{Hopf Algebras and their Actions
on Rings}, Regional Conference Series in Mathematics Nr. 82, AMS 1993.

\bibitem {Stanley-EC2}Richard P. Stanley, \textit{Enumerative Combinatorics,
volume 2}, CUP, 1999.

\end{thebibliography}

\end{document}