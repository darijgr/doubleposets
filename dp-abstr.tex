% -------------------------------------------------------------
% NOTE ON THE DETAILED AND SHORT VERSIONS:
% -------------------------------------------------------------
% This paper comes in two versions, a detailed and a short one.
% The short version should be more than sufficient for any
% reasonable use; the detailed one was written purely to
% convince the author of its correctness.
% To switch between the two versions, find the line containing
% "\newenvironment{noncompile}{}{}" in this LaTeX file.
% Look at the two lines right beneath this line.
% To compile the detailed version, they should be as follows:
%   \includecomment{verlong}
%   \excludecomment{vershort}
% To compile the short version, they should be as follows:
%   \excludecomment{verlong}
%   \includecomment{vershort}
% As a rule, the line
%   \excludecomment{noncompile}
% should stay as it is.
% -------------------------------------------------------------
% NOTES ON SOME HACKS USED IN THIS FILE:
% -------------------------------------------------------------
% One of my pet peeves with amsthm is its use of italics in the theorem and
% proposition environments; this makes math and text indistinguishable in said
% enviroments. To avoid this, I redefine the enviroments to use the standard
% font and to use a hanging indent, along with a bold vertical bar to its
% left, to distinguish these environments from surrounding text. (Along with
% the advantage of distinguishing math from text, this also allows nesting
% several such environments inside each other, like a definition inside a
% remark. I'm not sure how good of an idea this is, though. There are also
% downsides related to the hanging indentation, such as footnotes out of it
% being painful to do right.) This is done starting from the line
%   \theoremstyle{definition}
% and until the line
%   {\end{leftbar}\end{exmp}}

\documentclass[numbers=enddot,12pt,final,onecolumn,notitlepage,abstracton]{scrartcl}%
\usepackage[headsepline,footsepline,manualmark]{scrlayer-scrpage}
\usepackage[all,cmtip]{xy}
\usepackage{amssymb}
\usepackage{amsmath}
\usepackage{amsthm}
\usepackage{framed}
\usepackage{comment}
\usepackage{color}
\usepackage{tabu}
\usepackage[sc]{mathpazo}
\usepackage[T1]{fontenc}
\usepackage{needspace}
\usepackage[breaklinks]{hyperref}
%TCIDATA{OutputFilter=latex2.dll}
%TCIDATA{Version=5.50.0.2960}
%TCIDATA{LastRevised=Monday, September 21, 2015 16:27:34}
%TCIDATA{SuppressPackageManagement}
%TCIDATA{<META NAME="GraphicsSave" CONTENT="32">}
%TCIDATA{<META NAME="SaveForMode" CONTENT="1">}
%TCIDATA{BibliographyScheme=Manual}
%TCIDATA{Language=American English}
%BeginMSIPreambleData
\providecommand{\U}[1]{\protect\rule{.1in}{.1in}}
%EndMSIPreambleData
\newcounter{exer}
\theoremstyle{definition}
\newtheorem{theo}{Theorem}[section]
\newenvironment{theorem}[1][]
{\begin{theo}[#1]\begin{leftbar}}
{\end{leftbar}\end{theo}}
\newtheorem{lem}[theo]{Lemma}
\newenvironment{lemma}[1][]
{\begin{lem}[#1]\begin{leftbar}}
{\end{leftbar}\end{lem}}
\newtheorem{prop}[theo]{Proposition}
\newenvironment{proposition}[1][]
{\begin{prop}[#1]\begin{leftbar}}
{\end{leftbar}\end{prop}}
\newtheorem{defi}[theo]{Definition}
\newenvironment{definition}[1][]
{\begin{defi}[#1]\begin{leftbar}}
{\end{leftbar}\end{defi}}
\newtheorem{remk}[theo]{Remark}
\newenvironment{remark}[1][]
{\begin{remk}[#1]\begin{leftbar}}
{\end{leftbar}\end{remk}}
\newtheorem{coro}[theo]{Corollary}
\newenvironment{corollary}[1][]
{\begin{coro}[#1]\begin{leftbar}}
{\end{leftbar}\end{coro}}
\newtheorem{conv}[theo]{Convention}
\newenvironment{condition}[1][]
{\begin{conv}[#1]\begin{leftbar}}
{\end{leftbar}\end{conv}}
\newtheorem{quest}[theo]{Question}
\newenvironment{algorithm}[1][]
{\begin{quest}[#1]\begin{leftbar}}
{\end{leftbar}\end{quest}}
\newtheorem{warn}[theo]{Warning}
\newenvironment{conclusion}[1][]
{\begin{warn}[#1]\begin{leftbar}}
{\end{leftbar}\end{warn}}
\newtheorem{soln}{Solution}
\newenvironment{solution}[1][]
{\begin{soln}[#1]}
{\end{soln}}
\newtheorem{conj}[theo]{Conjecture}
\newenvironment{conjecture}[1][]
{\begin{conj}[#1]\begin{leftbar}}
{\end{leftbar}\end{conj}}
\newtheorem{exam}[theo]{Example}
\newenvironment{example}[1][]
{\begin{exam}[#1]\begin{leftbar}}
{\end{leftbar}\end{exam}}
\newtheorem{exmp}[exer]{Exercise}
\newenvironment{exercise}[1][]
{\begin{exmp}[#1]\begin{leftbar}}
{\end{leftbar}\end{exmp}}
\newenvironment{statement}{\begin{quote}}{\end{quote}}
\iffalse
\newenvironment{proof}[1][Proof]{\noindent\textbf{#1.} }{\ \rule{0.5em}{0.5em}}
\fi
\let\sumnonlimits\sum
\let\prodnonlimits\prod
\let\cupnonlimits\bigcup
\let\capnonlimits\bigcap
\renewcommand{\sum}{\sumnonlimits\limits}
\renewcommand{\prod}{\prodnonlimits\limits}
\renewcommand{\bigcup}{\cupnonlimits\limits}
\renewcommand{\bigcap}{\capnonlimits\limits}
\voffset=0cm
\hoffset=-0.7cm
\setlength\textheight{22.5cm}
\setlength\textwidth{15.5cm}
\newenvironment{verlong}{}{}
\newenvironment{vershort}{}{}
\newenvironment{noncompile}{}{}
\excludecomment{verlong}
\includecomment{vershort}
\excludecomment{noncompile}
\newcommand{\kk}{{\mathbf{k}}}
\newcommand{\xx}{{\mathbf{x}}}
\newcommand{\id}{{\operatorname{id}}}
\newcommand{\ev}{\operatorname{ev}}
\newcommand{\Adm}{\operatorname{Adm}}
\newcommand{\pack}{\operatorname{pack}}
\newcommand{\Comp}{{\operatorname{Comp}}}
\newcommand{\QSym}{{\operatorname{QSym}}}
\newcommand{\QSYM}{{\operatorname{QSYM}}}
\newcommand{\Powser}{\mathbf{k}\left[\left[x_1,x_2,x_3,\ldots\right]\right]}
\newcommand{\Par}{\operatorname{Par}}
\newcommand{\bdd}{\operatorname{bdd}}
\newcommand{\bD}{{\mathbf{D}}}
\newcommand{\EE}{{\mathbf{E}}}
\newcommand{\FF}{{\mathbf{F}}}
\newcommand{\bk}{{\mathbf{k}}}
%\newcommand{\Nplus}{{\mathbb{N}_{+}}}
\newcommand{\NN}{{\mathbb{N}}}
\newcommand{\ZZ}{{\mathbb{Z}}}
\newcommand{\QQ}{{\mathbb{Q}}}
%\newcommand{\arr}{\ar@{=>}}
%\newcommand{\bluenum}[1]{{\color{blue} #1 }}
\ihead{Double posets and the antipode of $\QSym$}
\ohead{page \thepage}
\cfoot{}
\begin{document}

\author{Darij Grinberg}

\title{Double posets and the antipode of $\QSym$}

\date{version 2.0 (\today)}

\begin{comment}
\keywords{
antipodes,
double posets,
Hopf algebras,
posets,
P-partitions,
quasisymmetric functions
}
\end{comment}

\maketitle

\begin{abstract}
We assign a quasisymmetric function to every double poset (that is,
every finite set endowed with two partial orders) and any weight function
on its ground set. This generalizes well-known objects such as monomial
and fundamental quasisymmetric functions, (skew) Schur functions, dual
immaculate functions, and quasisymmetric
$\left(P, \omega\right)$-partition enumerators.
We then prove a formula for the antipode of this function that
holds under certain conditions (which are satisfied when the second order
of the double poset is total, but also in some other cases); this
restates (in a way that to us seems more natural) a
result by Malvenuto and Reutenauer, but our proof is new and
self-contained. We generalize it further to an even more comprehensive
setting, where a group acts on the double poset by automorphisms.
\end{abstract}

\begin{comment}
% Here goes the French abstract (``R\'esum\'e''):

\begin{abstract}
Nous attribuons une fonction quasi-sym�trique � chaque ``double
poset'' (c'est-�-dire, ensemble fini dot� de deux ordres
partiels) et chaque fonction de poids sur son ensemble
sous-jacent. Cela g�n�ralise des objets bien connus tels que les
fonctions quasi-sym�triques fondamentales et monomiales, les
fonctions de Schur (obliques), les fonctions immacul�es duales,
et les $\left(P, \omega \right)$-partition enumerateurs
quasi-sym�triques.
Nous montrons ensuite une formule pour l'antipode de cette fonction
valable dans certaines conditions (qui sont satisfaits lorsque le second ordre
de la double poset est totale, mais aussi dans d'autres cas); ce
r�affirme (d'une mani�re qui nous semble plus naturelle) un
r�sultat par Malvenuto et Reutenauer, mais notre preuve est nouvelle et
autonome. Nous g�n�ralisons cet resultat plus loin � une situation
plus complexe,
o� un groupe agit sur le ``double poset'' par automorphismes.
\end{abstract}
\end{comment}

\section{Introduction}
\label{sec:in}

Double posets and $\EE$-partitions (for $\EE$ a double poset)
have been introduced by Claudia Malvenuto and Christophe
Reutenauer \cite{Mal-Reu-DP} in order to construct a combinatorial
Hopf algebra which harbors a noticeable amount of
structure, including an analogue of the Littlewood-Richardson
rule and a lift of the internal product operation of the
Malvenuto-Reutenauer Hopf algebra of permutations. In this note,
we shall employ these same notions to restate in a simpler form,
and reprove in a more elementary fashion, a formula for the
antipode in the Hopf algebra $\QSym$ of quasisymmetric functions
due to (the same) Malvenuto and Reutenauer (generalizing
an earlier result by Gessel), and extend it further to a case in
which a group acts on the double poset.

In the present version of the paper, some (classical and/or
straightforward) proofs are missing or sketched. A more detailed
version exists, in which at least a few of these proofs are
elaborated on more\footnote{It can be downloaded from
\url{http://web.mit.edu/~darij/www/algebra/dp-abstr-long.pdf}}.

\subsection*{Acknowledgments}

Katharina Jochemko's work \cite{Joch} provoked this research.
I learnt a lot about $\QSym$ from Victor Reiner.

\section{Quasisymmetric functions}
\label{sect.qsym-intro}

Let us first briefly introduce the notations that will be used in the
following.

We set $\NN = \left\{0, 1, 2, \ldots\right\}$. A \textit{composition}
means a finite sequence of positive integers. We let $\Comp$ be the set
of all compositions. For $n \in \NN$, a \textit{composition of $n$}
means a composition whose entries sum to $n$ (that is, a composition
$\left(\alpha_1, \alpha_2, \ldots, \alpha_k\right)$ satisfying
$\alpha_1 + \alpha_2 + \cdots + \alpha_k = n$).

Let $\kk$ be an arbitrary commutative ring. We shall keep $\kk$ fixed
throughout this paper.
We consider the $\kk$-algebra $\Powser$ of formal
power series in infinitely many (commuting) indeterminates
$x_1, x_2, x_3, \ldots$ over $\kk$. A \textit{monomial} shall always
mean a monomial (without coefficients) in the variables
$x_1, x_2, x_3, \ldots$.\ \ \ \ \footnote{For the sake of completeness,
let us give a detailed definition of monomials and of the topology
on $\Powser$. (This definition has been copied from
\cite[\S 2]{Gri-dimm}, essentially unchanged.)

Let $x_{1},x_{2},x_{3},\ldots$ be countably many distinct symbols. We
let $\operatorname*{Mon}$ be the free abelian monoid on the set $\left\{
x_{1},x_{2},x_{3},\ldots\right\}  $ (written multiplicatively); it consists of
elements of the form $x_{1}^{a_{1}}x_{2}^{a_{2}}x_{3}^{a_{3}}\cdots$ for
finitely supported $\left(  a_{1},a_{2},a_{3},\ldots\right)  \in
\mathbb{N}^{\infty}$ (where \textquotedblleft finitely
supported\textquotedblright\ means that all but finitely many positive
integers $i$ satisfy $a_{i}=0$). A \textit{monomial} will mean an element of
$\operatorname*{Mon}$. Thus, a monomial is a combinatorial
object, independent of $\mathbf{k}$; it does not carry a coefficient.

We consider the $\mathbf{k}$-algebra $\Powser$ of (commutative) power
series in
countably many distinct indeterminates $x_{1},x_{2},x_{3},\ldots$ over
$\mathbf{k}$. By abuse of notation, we shall identify every monomial
$x_{1}^{a_{1}}x_{2}^{a_{2}}x_{3}^{a_{3}}\cdots\in\operatorname*{Mon}$ with the
corresponding element $x_{1}^{a_{1}}\cdot x_{2}^{a_{2}}\cdot x_{3}^{a_{3}%
}\cdot\cdots$ of $\Powser$ when necessary
(e.g., when we speak of the sum of two monomials or
when we multiply a monomial with an element of $\mathbf{k}$). (To be
very pedantic, this identification is slightly dangerous, because
it can happen that two distinct monomials in $\operatorname{Mon}$ get
identified with two identical elements of $\Powser$. However, this
can only happen when the ring $\kk$ is trivial, and even then it is
not a real problem unless we infer the equality of monomials from the
equality of their counterparts in $\Powser$, which we are not going to
do.)

We furthermore endow the ring $\Powser$ with the following topology
(as in \cite[Section 2.6]{Reiner}):

We endow the ring $\mathbf{k}$ with the discrete topology. To define a
topology on the $\mathbf{k}$-algebra $\Powser$, we (temporarily) regard every power
series in $\Powser$ as the family of its coefficients (indexed by the
set $\operatorname{Mon}$). More precisely, we have a
$\kk$-module isomorphism
\[
\prod_{\mathfrak{m} \in \operatorname{Mon}} \kk \to \Powser,
\qquad \qquad \left(\lambda_{\mathfrak{m}}\right)_{\mathfrak{m} \in \operatorname{Mon}}
\mapsto \sum_{\mathfrak{m} \in \operatorname{Mon}} \lambda_{\mathfrak{m}} \mathfrak{m} .
\]
We use this isomorphism to transport the product topology on
$\prod_{\mathfrak{m} \in \operatorname{Mon}} \kk$ to $\Powser$. The
resulting topology on $\Powser$ turns $\Powser$ into a topological
$\kk$-algebra; this is the topology that we will
be using whenever we make statements about convergence in $\Powser$
or write down infinite sums of power series.
A sequence $\left( a_n \right)_{n \in \NN}$ of power series converges
to a power series $a$ with respect to this topology if
and only if for every monomial $\mathfrak{m}$, all sufficiently high
$n \in \NN$ satisfy
\[
\left(  \text{the coefficient of } \mathfrak{m}\text{ in }a_{n}\right)
=\left(  \text{the coefficient of } \mathfrak{m}\text{ in }a\right)  .
\]

Note that this topological $\kk$-algebra $\Powser$ is \textbf{not}
the completion of $\mathbf{k}\left[  x_{1},x_{2},x_{3},\ldots\right]$
with respect to the standard grading (in which all $x_{i}$ have degree $1$).
(They are distinct even as sets.)
}

Inside the $\kk$-algebra $\Powser$ is a
subalgebra $\Powser_{\bdd}$ consisting of the \textit{bounded-degree}
formal power series; these are the power series $f$ for which there
exists a $d \in \NN$ such that no monomial of degree $> d$ appears in
$f$\ \ \ \ \footnote{The \textit{degree} of a monomial
$x_1^{a_1} x_2^{a_2} x_3^{a_3} \cdots$ is defined to be the nonnegative
integer $a_1 + a_2 + a_3 + \cdots$. A monomial $\mathfrak{m}$ is said
to \textit{appear} in a power series $f \in \Powser$ if and only if
the coefficient of $\mathfrak{m}$ in $f$ is nonzero.}.
This $\kk$-subalgebra $\Powser_{\bdd}$ becomes a topological
$\kk$-algebra, by inheriting the topology from $\Powser$.

Two monomials $\mathfrak{m}$ and $\mathfrak{n}$ are said to be
\textit{pack-equivalent}\footnote{Pack-equivalence and the related
notions of packed combinatorial objects that we will encounter below
originate in work of Hivert, Novelli and Thibon
\cite{Nov-Thi}. Simple as they are, they are of great help in dealing
with quasisymmetric functions.} if they have the forms
$x_{i_1}^{a_1} x_{i_2}^{a_2} \cdots x_{i_\ell}^{a_\ell}$ and
$x_{j_1}^{a_1} x_{j_2}^{a_2} \cdots x_{j_\ell}^{a_\ell}$ for two
strictly increasing sequences
$\left(i_1 < i_2 < \cdots < i_\ell\right)$
and $\left(j_1 < j_2 < \cdots < j_\ell\right)$ of positive integers and
one (common) sequence $\left(a_1, a_2, \ldots, a_\ell\right)$ of
positive integers.\footnote{For instance, $x_2^2 x_3 x_4^2$ is
pack-equivalent to $x_1^2 x_4 x_8^2$ but not to $x_2 x_3^2 x_4^2$.}
A power series $f \in \Powser$ is said to be \textit{quasisymmetric}
if every two pack-equivalent monomials have equal coefficients in
front of them in $f$. It is easy to see that the quasisymmetric
power series form a $\kk$-subalgebra of $\Powser$; but usually one
is interested in the set of quasisymmetric bounded-degree power
series in $\Powser$.
This latter set is a $\kk$-subalgebra of $\Powser_{\bdd}$, and
is known as the \textit{$\kk$-algebra of quasisymmetric functions
over $\kk$}. It is denoted by $\QSym$.
It is clear that symmetric functions (in the
usual sense of this word in combinatorics -- so, really, symmetric
bounded-degree power series in $\Powser$) form a $\kk$-subalgebra
of $\QSym$. The quasisymmetric functions have a rich theory which
is related to, and often sheds new light on, the classical theory of
symmetric functions; expositions can be found in
\cite[\S\S 7.19, 7.23]{Stanley-EC2} and \cite[\S\S 5-6]{Reiner} and
other sources.

As a $\kk$-module, $\QSym$ has a basis
$\left(M_\alpha\right)_{\alpha \in \Comp}$ indexed by all
compositions, where the quasisymmetric function $M_\alpha$ for a
given composition $\alpha$ is defined as follows: Writing $\alpha$
as $\left(\alpha_1, \alpha_2, \ldots, \alpha_\ell\right)$, we set
\[
M_\alpha
= \sum_{i_1 < i_2 < \cdots < i_\ell}
 x_{i_1}^{\alpha_1} x_{i_2}^{\alpha_2} \cdots x_{i_\ell}^{\alpha_\ell}
= \sum_{\substack{\mathfrak{m}\text{ is a monomial pack-equivalent} \\
                  \text{to }
                  x_1^{\alpha_1} x_2^{\alpha_2} \cdots x_\ell^{\alpha_\ell}}}
  \mathfrak{m}
\]
(where the $i_k$ in the first sum are positive integers). This
basis $\left(M_\alpha\right)_{\alpha \in \Comp}$ is known as the
\textit{monomial basis} of $\QSym$, and is the simplest to
define among many. (We shall briefly encounter another basis in
Example~\ref{exam.Gamma}.)

The $\kk$-algebra $\QSym$ can be endowed with a structure of a
$\kk$-coalgebra which, combined with its $\kk$-algebra structure,
turns it into a Hopf algebra. We refer to the literature both for
the theory of coalgebras and Hopf algebras
(see
\cite{Montg-Hopf}, \cite[\S 1]{Reiner}, \cite[\S 1-\S 2]{Manchon-HA},
\cite{Abe-HA}, \cite{Sweedler-HA}, \cite{Dasca-HA} or
\cite[Chapter 7]{Fresse-Op})
and for a deeper study of the Hopf algebra $\QSym$ (see
\cite{Malve-Thesis}, \cite[Chp. 6]{HGK} or
\cite[\S 5]{Reiner}); in this note we shall need but the very
basics of this structure, and so it is only them that we introduce.

We define a $\kk$-linear map $\Delta : \QSym \to \QSym \otimes \QSym$
(here and in the following, all tensor products are over $\kk$ by
default)
by requiring that
\begin{align}
\label{eq.coproduct.M}
\Delta \left( M_{\left( \alpha_1, \alpha_2, \ldots, \alpha_\ell
\right) }\right)
&= \sum_{k=0}^{\ell} M_{\left( \alpha_1, \alpha_2, \ldots,
\alpha_k \right) } \otimes M_{\left( \alpha_{k+1}, \alpha_{k+2},
\ldots, \alpha_\ell \right) } \\
& \qquad \text{ for every } \left(\alpha_1, \alpha_2,
\ldots, \alpha_\ell\right) \in \Comp \nonumber
\end{align}
\footnote{This definition relies on the fact that
$\left(M_\alpha\right)_{\alpha \in \Comp}$ is a basis of the
$\kk$-module $\QSym$.}. We further define a $\kk$-linear map
$\varepsilon : \QSym \to \kk$ by requiring that
\[
\varepsilon\left(  M_{\left(
\alpha_1, \alpha_2, \ldots, \alpha_\ell \right) }\right)
= \delta_{\ell, 0}
\qquad \text{ for every } \left(\alpha_1, \alpha_2,
\ldots, \alpha_\ell\right) \in \Comp
\]
\footnote{Here, $\delta_{u,v}$ is defined to be
$\begin{cases}
1, & \text{if }u = v \text{;}\\
0, & \text{if }u \neq v
\end{cases}$
whenever $u$ and $v$ are two objects.}.
(Equivalently, $\varepsilon$ sends every power series $f \in \QSym$
to the result $f\left(0,0,0,\ldots\right)$ of substituting zeroes
for the variables $x_1, x_2, x_3, \ldots$ in $f$. The map $\Delta$
can also be described in such terms, but with greater difficulty
\cite[(5.3)]{Reiner}.) It is well-known that these maps $\Delta$ and
$\varepsilon$ make the three diagrams
\begin{align*}
& \xymatrixcolsep{5pc}\xymatrix{
\QSym\ar[r]^-{\Delta} \ar[d]_{\Delta} & \QSym\otimes\QSym\ar[d]^{\Delta
\otimes\id} \\
\QSym\otimes\QSym\ar[r]_-{\id\otimes\Delta} & \QSym\otimes\QSym\otimes\QSym}
, \\
& \xymatrixcolsep{3pc}
\xymatrix{
\QSym\ar[dr]_{\cong} \ar[r]^-{\Delta} & \QSym\otimes\QSym\ar[d]^-{\varepsilon
\otimes\id} \\
& \bk\otimes\QSym}
,
\qquad
\xymatrixcolsep{3pc}
\xymatrix{
\QSym\ar[dr]_{\cong} \ar[r]^-{\Delta} & \QSym\otimes\QSym\ar[d]^-{\id
\otimes\varepsilon} \\
& \QSym\otimes\bk}
\end{align*}
(where the $\cong$ arrows are the canonical isomorphisms)
commutative, and so $\left(\QSym, \Delta, \varepsilon\right)$ is what
is commonly called a \textit{$\kk$-coalgebra}. Furthermore, $\Delta$
and $\varepsilon$ are $\kk$-algebra homomorphisms, which is what makes
this $\kk$-coalgebra $\QSym$ into a \textit{$\kk$-bialgebra}. Finally,
let $m : \QSym \otimes \QSym \to \QSym$ be the $\kk$-linear map sending
every pure tensor $a \otimes b$ to $ab$, and let $u : \kk \to \QSym$ be
the $\kk$-linear map sending $1 \in \kk$ to $1 \in \QSym$. Then, there
exists a unique $\kk$-linear map $S : \QSym \to \QSym$ making the
diagram
\begin{equation}
\xymatrix{
& \QSym \otimes \QSym \ar[rr]^{S \otimes \id} & & \QSym \otimes \QSym \ar[dr]^{m} & \\
\QSym \ar[ur]^{\Delta} \ar[rr]^{\varepsilon} \ar[dr]_{\Delta} & & \kk \ar[rr]^{u} & & \QSym \\
& \QSym \otimes \QSym \ar[rr]^{\id \otimes S} & & \QSym \otimes \QSym \ar[ur]^{m} & 
}
\label{eq.antipode}
\end{equation}
commutative. This map $S$ is known as the \textit{antipode} of $\QSym$.
It is known to be an involution and an algebra automorphism of $\QSym$,
and its action on the various quasisymmetric functions defined
combinatorially is the main topic of this note.
The existence of the antipode $S$ makes $\QSym$ into a
\textit{Hopf algebra}.

\section{Double posets}
\label{sect.double-posets}

Next, we shall introduce the notion of a double poset, following
Malvenuto and Reutenauer \cite{Mal-Reu-DP}.

\begin{definition}
\label{def.double-poset}
\begin{itemize}

\item[(a)] We shall encode posets as pairs $\left(P, <\right)$,
where $P$ is a set and $<$ is a strict partial order
relation (i.e., an irreflexive, transitive and antisymmetric
binary relation) on the set $P$; this relation $<$ will be regarded
as the smaller relation of the poset. (All binary relations will be
written in infix notation: i.e., we write ``$a < b$'' for ``$a$ is
related to $b$ by the relation $<$''.)

\item[(b)] If $<$ is a strict partial order relation on a set $P$,
and if $a$ and $b$ are two elements of $P$, then we say that
$a$ and $b$ are \textit{$<$-comparable} if we have either $a < b$
or $a = b$ or $b < a$. A strict partial order relation $<$ on a
set $P$ is said to be a \textit{total order} if and only
if every two elements of $P$ are $<$-comparable.

\item[(c)] If $<$ is a strict partial order relation on a set $P$,
and if $a$ and $b$ are two elements of $P$, then we say that
$a$ is \textit{$<$-covered by $b$} if we have $a < b$ and there
exists no $c \in P$ satisfying $a < c < b$. (For instance, if $<$
is the standard smaller relation on $\ZZ$, then each
$i \in \ZZ$ is $<$-covered by $i+1$.)

\item[(d)] A \textit{double poset} is defined as a triple
$\left(E, <_1, <_2\right)$ where $E$ is a finite set and $<_1$ and
$<_2$ are two strict partial order relations on $E$.

\item[(e)] A double poset
$\left(E, <_1, <_2\right)$ is said to be \textit{special} if
the relation $<_2$ is a total order.

\item[(f)] A double poset
$\left(E, <_1, <_2\right)$ is said to be \textit{semispecial} if
every two $<_1$-comparable elements of $E$ are $<_2$-comparable.

\item[(g)] A double poset
$\left(E, <_1, <_2\right)$ is said to be \textit{tertispecial} if
it satisfies the following condition: If $a$ and $b$ are two
elements of $E$ such that $a$ is $<_1$-covered by $b$, then $a$
and $b$ are $<_2$-comparable.

\item[(h)] If $<$ is a binary relation on a set $P$, then the
\textit{opposite relation} of $<$ is defined to be the binary
relation $>$ on the set $P$ which is defined as follows: For any
$e \in P$ and $f \in P$, we have $e > f$ if and only if $f < e$.
Notice that if $<$ is a strict partial order relation, then so
is the opposite relation $>$ of $<$.
\end{itemize}
\end{definition}

Clearly, every special double poset is semispecial, and every
semispecial double poset is tertispecial.\footnote{The notions of a
double poset and of a special double poset come from
\cite{Mal-Reu-DP}. The notion of a ``tertispecial double poset''
(Dog Latin for ``slightly less special than semispecial'')
appears to be new and arguably sounds artificial, but is the
most suitable setting for some of the results below (and appears
in nature, beyond the particular case of special double posets
-- see Example~\ref{exam.dp}). We shall not use semispecial
double posets in the following; they were only introduced as a
middle ground between special and tertispecial double posets
with a less daunting definition.}

\begin{definition}
\label{def.E-partition}
If $\EE = \left(E, <_1, <_2\right)$ is a double poset, then
an \textit{$\EE$-partition} shall mean a map
$\phi : E \to \left\{ 1,2,3,\ldots\right\}$ such that:
\begin{itemize}
\item every $e \in E$ and $f \in E$ satisfying $e <_1 f$ satisfy
$\phi\left(e\right) \leq \phi\left(f\right)$;
\item every $e \in E$ and $f \in E$ satisfying $e <_1 f$ and
$f <_2 e$ satisfy $\phi\left(e\right) < \phi\left(f\right)$.
\end{itemize}
\end{definition}

\begin{example}
\label{exam.dp}
The notion of an $\EE$-partition (which was inspired by the earlier
notions of $P$-partitions and $\left(P,\omega\right)$-partitions
as studied by Gessel and Stanley\footnotemark)
generalizes various well-known
combinatorial concepts. For example:
\begin{itemize}
\item If $<_2$ is the same order
as $<_1$ (or any extension of this order), then
$\EE$-partitions are weakly increasing maps from the poset
$\left(E, <_1\right)$ to the totally ordered set
$\left\{1, 2, 3, \ldots\right\}$.
\item If $<_2$ is the opposite relation of
$<_1$ (or any extension of this opposite relation), then
$\EE$-partitions are strictly increasing maps from the
poset $\left(E, <_1\right)$ to the totally ordered set
$\left\{1, 2, 3, \ldots\right\}$.
\end{itemize}

For a more interesting example,
let $\mu = \left(\mu_1, \mu_2, \mu_3, \ldots\right)$ and
$\lambda = \left(\lambda_1, \lambda_2, \lambda_3, \ldots\right)$ be
two partitions such that $\mu \subseteq \lambda$.
(See \cite[\S 2]{Reiner} for the notations we are using
here.)
The skew Young
diagram $Y\left(\lambda / \mu\right)$ is then defined as the set of all
$\left(i, j\right) \in \left\{ 1, 2, 3, \ldots \right\}^2$ satisfying
$\mu_i < j \leq \lambda_i$. On this set $Y\left(\lambda / \mu\right)$,
we define two partial order relations $<_1$ and $<_2$ by
\[
\left(i,j\right) <_1 \left(i',j'\right) \Longleftrightarrow
\left( i \leq i' \text{ and } j \leq j' \text{ and }
\left(i,j\right) \neq \left(i',j'\right) \right)
\]
and
\[
\left(i,j\right) <_2 \left(i',j'\right) \Longleftrightarrow
\left( i \geq i' \text{ and } j \leq j' \text{ and }
\left(i,j\right) \neq \left(i',j'\right) \right) .
\]
The resulting double poset
$\mathbf{Y}\left(\lambda / \mu\right)
= \left(Y\left(\lambda / \mu\right), <_1, <_2\right)$ has the
property that the $\mathbf{Y}\left(\lambda / \mu\right)$-partitions
are precisely the semistandard tableaux of shape
$\lambda / \mu$. (Again, see \cite[\S 2]{Reiner} for the meaning
of these words.)

This double poset $\mathbf{Y}\left(\lambda / \mu\right)$
is not special (in general), but it is tertispecial. (Indeed,
if $a$ and $b$ are two elements of $Y\left(\lambda / \mu\right)$
such that $a$ is $<_1$-covered by $b$, then $a$ is either the left
neighbor of $b$ or the top neighbor of $b$, and thus we have
either $a <_2 b$ (in the former case) or $b <_2 a$ (in the latter
case).) Some authors prefer to use a special double poset instead,
which is defined as follows: We define a total
order $<_h$ on $Y\left(\lambda / \mu\right)$ by
\[
\left(i,j\right) <_h \left(i',j'\right) \Longleftrightarrow
\left( i > i' \text{ or } \left( i = i' \text{ and }
j < j' \right) \right) .
\]
Then, $\mathbf{Y}_h\left(\lambda / \mu\right)
= \left(Y\left(\lambda / \mu\right), <_1, <_h\right)$ is a special
double poset, and the
$\mathbf{Y}_h\left(\lambda / \mu\right)$-partitions
are precisely the semistandard tableaux of shape
$\lambda / \mu$.
\end{example}
\footnotetext{See \cite{Gessel-Ppar}
for the history of these notions, and see \cite{Gessel},
\cite{Stanley-Thes}, \cite[\S 3.15]{Stanley-EC1} and
\cite[\S 7.19]{Stanley-EC2} for
some of their theory. Mind that these sources use different and
sometimes incompatible notations -- e.g., the $P$-partitions of
\cite[\S 3.15]{Stanley-EC1} and \cite{Gessel-Ppar} differ from
those of \cite{Gessel} by a sign reversal.}

We now assign a certain formal power series to every double poset:

\begin{definition}
\label{def.Gammaw}
If $\EE = \left(E, <_1, <_2\right)$ is a double poset, and
$w : E \to \left\{1, 2, 3, \ldots\right\}$ is a map, then we define
a power series $\Gamma\left(\EE , w\right) \in \Powser$ by
\[
\Gamma\left(\EE , w\right)
= \sum_{\pi\text{ is an }\EE\text{-partition}}
  \xx_{\pi, w} ,
\qquad
\text{where } \xx_{\pi, w}
= \prod_{e \in E} x_{\pi\left(e\right)}^{w\left(e\right)} .
\]
\end{definition}

The following fact is easy to see (but will be reproven below):

\begin{proposition}
\label{prop.Gammaw.qsym}
Let $\EE = \left(E, <_1, <_2\right)$ be a double poset, and
$w : E \to \left\{1, 2, 3, \ldots\right\}$ be a map. Then,
$\Gamma\left(\EE , w\right) \in \QSym$.
\end{proposition}

\begin{example}
\label{exam.Gamma}
The power series $\Gamma\left(\EE , w\right)$ generalize various
well-known quasisymmetric functions.

\begin{enumerate}
\item[(a)] If $\EE = \left(E, <_1, <_2\right)$ is a double poset, and
$w : E \to \left\{1, 2, 3, \ldots\right\}$ is the constant
function sending everything to $1$, then
$\Gamma\left(\EE , w\right)
= \sum_{\pi\text{ is an }\EE\text{-partition}} \xx_{\pi}$,
where $\xx_{\pi} = \prod_{e \in E} x_{\pi\left(e\right)}$.
We shall denote this power series $\Gamma\left(\EE , w\right)$
by $\Gamma\left(\EE\right)$; it is exactly what has been called
$\Gamma\left(\EE\right)$ in \cite[\S 2.2]{Mal-Reu-DP}. All results
proven below for $\Gamma\left(\EE , w\right)$ can be applied to
$\Gamma\left(\EE\right)$, yielding simpler (but less general)
statements.

\item[(b)] If $E = \left\{1, 2, \ldots, \ell\right\}$ for some
$\ell \in \NN$, if $<_1$ is the usual total order inherited from
$\ZZ$, and if $<_2$ is the opposite relation of $<_1$, then the
special double poset $\EE = \left(E, <_1, <_2\right)$ satisfies
$\Gamma\left(\EE, w\right) = M_\alpha$, where $\alpha$ is the
composition $\left(w\left(1\right), w\left(2\right), \ldots,
w\left(\ell\right)\right)$. Thus, the elements of the monomial
basis $\left(M_\alpha\right)_{\alpha \in \Comp}$ are special
cases of the functions $\Gamma\left(\EE, w\right)$.

\item[(c)] Let
$\alpha = \left(\alpha_1, \alpha_2, \ldots, \alpha_\ell\right)$
be a composition of a nonnegative integer $n$. Let
$D\left(\alpha\right)$ be the set
$\left\{\alpha_1, \alpha_1 + \alpha_2, \alpha_1 + \alpha_2
+ \alpha_3, \ldots, \alpha_1 + \alpha_2 + \cdots + \alpha_{\ell-1}
\right\}$.
Let $E$ be the set $\left\{1, 2, \ldots, n\right\}$, and let
$<_1$ be the total order inherited on $E$ from $\ZZ$. Let $<_2$ be some
partial order on $E$ with the property that
\[
i+1 <_2 i \qquad \text{ for every } i \in D\left(\alpha\right)
\]
and
\[
i <_2 i+1 \qquad \text{ for every }
i \in \left\{1, 2, \ldots, n-1\right\}
\setminus D\left(\alpha\right) .
\]
(There are several choices for such an order; in particular, we
can find one which is a total order.) Then,
\begin{align*}
\Gamma\left(\left(E, <_1, <_2\right)\right)
&= \sum_{\substack{i_1 \leq i_2 \leq \cdots \leq i_n; \\
                  i_j < i_{j+1} \text{ whenever } j \in D\left(\alpha\right)}}
  x_{i_1} x_{i_2} \cdots x_{i_n} \\
&= \sum_{\beta\text{ is a composition of }n;
        \ D\left(\beta\right) \supseteq D\left(\alpha\right)}
  M_\beta .
\end{align*}
This power series is known as the $\alpha$-th
\textit{fundamental quasisymmetric function}, usually called
$F_\alpha$ (in \cite[\S 2.4]{BBSSZ} and \cite[\S 2]{Gri-dimm})
or $L_\alpha$ (in \cite[\S 7.19]{Stanley-EC2} or
\cite[Def. 5.15]{Reiner}).

\item[(d)] Let $\EE$ be one of the two double posets
$\mathbf{Y}\left(\lambda / \mu\right)$ and
$\mathbf{Y}_h\left(\lambda / \mu\right)$
defined as in Example \ref{exam.dp} for two partitions $\mu$
and $\lambda$. Then, $\Gamma\left(\EE\right)$ is the skew
Schur function $s_{\lambda / \mu}$.

\item[(e)] Similarly, \textit{dual immaculate functions} as defined in
\cite[\S 3.7]{BBSSZ} can be realized as $\Gamma\left(\EE\right)$
for conveniently chosen $\EE$ (see \cite[Proposition 4.4]{Gri-dimm}), which
helped the author to prove one of their properties \cite{Gri-dimm}.
(The $\EE$-partitions here are the so-called
\textit{immaculate tableaux}.)

\item[(f)] When the relation $<_2$ of a double poset
$\EE = \left(E, <_1, <_2\right)$ is a total order (i.e.,
when the double poset $\EE$ is special), the
$\EE$-partitions are precisely the
reverse $\left(P, \omega\right)$-partitions (for
$P = \left(E, <_1\right)$ and $\omega$ being a labelling of $P$
dictated by $<_2$) in the terminology
of \cite[\S 7.19]{Stanley-EC2}, and the power series
$\Gamma\left(\EE\right)$ is the $K_{P, \omega}$ of
\cite[\S 7.19]{Stanley-EC2}.
This can also be rephrased using
the notations of \cite[\S 5.2]{Reiner}: When the relation $<_2$ of a
double poset $\EE = \left(E, <_1, <_2\right)$ is a total order, we can
relabel the elements of $E$ by the integers $1, 2, \ldots, n$ in such
a way that $1 <_2 2 <_2 \cdots <_2 n$; then, the $\EE$-partitions are
the $P$-partitions in the terminology of \cite[Def. 5.12]{Reiner},
where $P$ is the labelled poset $\left(E, <_1\right)$; and furthermore,
our $\Gamma\left(\EE\right)$ is the $F_P\left(\xx\right)$ of
\cite[Def. 5.12]{Reiner}. Conversely, if $P$ is a labelled poset, then
the $F_P\left(\xx\right)$ of \cite[Def. 5.12]{Reiner} is our
$\Gamma\left(\left(P, <_P, <_{\ZZ}\right)\right)$.

\end{enumerate}

\end{example}

\section{The antipode theorem}
\label{sect.antipode}

We now come to the main results of this note. We first state a
theorem and a corollary which are not new, but will be reproven in
a more self-contained way which allows them to take their
(well-deserved) place as fundamental results rather than
afterthoughts in the theory of $\QSym$.

\begin{definition}
We let $S$ denote the antipode of $\QSym$.
\end{definition}

\begin{theorem}
\label{thm.antipode.Gammaw}
Let $\left(E, <_1, <_2\right)$ be a tertispecial double poset.
Let $w : E \to \left\{1, 2, 3, \ldots\right\}$. Then,
$S\left(\Gamma\left(\left(E, <_1, <_2\right), w\right)\right)
= \left(-1\right)^{\left|E\right|}
\Gamma\left(\left(E, >_1, <_2\right), w\right)$,
where $>_1$ denotes the opposite relation of $<_1$.
\end{theorem}

\begin{corollary}
\label{cor.antipode.Gamma}
Let $\left(E, <_1, <_2\right)$ be a tertispecial double poset.
Then, $S\left(\Gamma\left(\left(E, <_1, <_2\right)\right)\right)
= \left(-1\right)^{\left|E\right|}
\Gamma\left(\left(E, >_1, <_2\right)\right)$,
where $>_1$ denotes the opposite relation of $<_1$.
\end{corollary}

We shall give examples for consequences of these facts shortly
(Example~\ref{exam.antipode.Gammaw}), but
let us first explain where they have already appeared.
Corollary~\ref{cor.antipode.Gamma} is equivalent to
\cite[Corollary 5.27]{Reiner}\footnote{It is easiest to derive
\cite[Corollary 5.27]{Reiner} from our Corollary~\ref{cor.antipode.Gamma},
as this only requires setting $\EE = \left(P, <_P, <_{\ZZ}\right)$
(this is a special double poset, thus in particular a tertispecial
one) and noticing that
$\Gamma\left(\left(P, <_P, <_{\ZZ}\right)\right)
= F_P\left(\xx\right)$ and
$\Gamma\left(\left(P, >_P, <_{\ZZ}\right)\right)
= F_{P^{\operatorname{opp}}}\left(\xx\right)$, where all unexplained
notations are defined in \cite[Chp. 5]{Reiner}. But one can also
proceed in the opposite direction.}
(a result found by Malvenuto and Reutenauer, as well as by Ehrenborg
in an equivalent form).
Theorem~\ref{thm.antipode.Gammaw} is equivalent to Malvenuto's
and Reutenauer's \cite[Theorem 3.1]{Mal-Reu}\footnote{This equivalence
requires a bit of work to set up. To derive \cite[Theorem 3.1]{Mal-Reu}
from our Theorem~\ref{thm.antipode.Gammaw}, it is enough to contract
all undirected edges in $G$, denoting the vertex set of the new
graph by $E$, and then define two order relations $<_1$ and $<_2$ on
$E$ by
\[
\left(a <_1 b\right) \Longleftrightarrow \left(a \neq b,
\text{ and there exists a path from } a \text{ to } b \text{ in }
G \right)
\]
and
\[
\left(a <_2 b\right) \Longleftrightarrow \left(a \neq b,
\text{ and there exists a path from } a \text{ to } b \text{ in }
G' \right) .
\]
The map $w$ sends every $e \in E$ to the number of vertices
of $G$ that became $e$ when the edges were contracted. To show that
the resulting double poset $\left(E, <_1, <_2\right)$ is
tertispecial, we must notice that if $a$ is $<_1$-covered by $b$,
then $G$ had an edge from one of the vertices that became $a$ to
one of the vertices that became $b$. The ``$x_i$'s in $X$
satisfying a set of conditions'' (in the language of
\cite[Section 3]{Mal-Reu}) are then in 1-to-1 correspondence with
$\left(E, <_1, <_2\right)$-partitions (at least when
$X = \left\{1, 2, 3, \ldots\right\}$); this is not immediately
obvious but not hard to check either (the acyclicity of $G$ and
$G^\prime$ is used in the proof). As a result,
\cite[Theorem 3.1]{Mal-Reu} follows from
Theorem~\ref{thm.antipode.Gammaw} above.
With some harder work, one can conversely derive
our Theorem~\ref{thm.antipode.Gammaw} from
\cite[Theorem 3.1]{Mal-Reu}.}. We
nevertheless believe that our versions of these facts are more
natural and simpler than the ones appearing in existing
literature\footnote{That said, we would not be surprised if
Malvenuto and Reutenauer are aware of them and just have not
published them; after all, they have discovered both the original
version of Theorem~\ref{thm.antipode.Gammaw} in
\cite{Mal-Reu} and the notion of double posets in \cite{Mal-Reu-DP}.},
and if not,
then at least their proofs below are more in the nature of things.
% [todo: check that the equivalence to Malvenuto-Reutenauer is
% correct!]

To these known results, we add another, which seems to be unknown so
far (probably because it is far harder to state in the terminologies
of $\left(P, \omega\right)$-partitions or
equality-and-inequality conditions appearing in literature). First,
we need to introduce some notation:

\begin{definition}
\label{def.G-sets.terminology}
Let $G$ be a group, and let $E$ be a $G$-set.

\begin{itemize}

\item[(a)] Let $<$ be a
strict partial order relation on $E$. We say that $G$
\textit{preserves the relation $<$} if the following holds:
For every $g \in G$, $a \in E$ and $b \in E$ satisfying $a < b$,
we have $ga < gb$.

\item[(b)] Let $w : E \to \left\{1, 2, 3, \ldots\right\}$. We
say that $G$ \textit{preserves $w$} if every $g \in G$ and
$e \in E$ satisfy $w\left(ge\right) = w\left(e\right)$.

\item[(c)] Let $g \in G$. Assume that the set $E$ is finite.
We say that $g$ is \textit{$E$-even}
if the action of $g$ on $E$ (that is, the permutation of $E$
that sends every $e \in E$ to $ge$) is an even permutation
of $E$.

\item[(d)] If $X$ is any set, then the set $X^E$ of all maps
$E \to X$ becomes a $G$-set in the following way: For any
$\pi \in X^E$ and $g \in G$, we define the element $g\pi \in X^E$
to be the map sending each $e \in E$ to $\pi\left(g^{-1}e\right)$.

\item[(e)] Let $F$ be a further $G$-set. Assume that the set
$E$ is finite. An element $\pi \in F$
is said to be \textit{$E$-coeven} if every $g \in G$
satisfying $g\pi = \pi$ is $E$-even. A $G$-orbit $O$ on $F$ is said
to be \textit{$E$-coeven} if all elements of $O$ are $E$-coeven.

\end{itemize}
\end{definition}

Before we come to the promised result, let us state a simple fact:

\begin{lemma}
\label{lem.coeven.all-one}
Let $G$ be a group. Let $F$ and $E$ be $G$-sets such that $E$ is
finite. Let $O$ be a
$G$-orbit on $F$. Then, $O$ is $E$-coeven if and only if at least
one element of $O$ is $E$-coeven.
\end{lemma}

\begin{theorem}
\label{thm.antipode.GammawG}
Let $\EE = \left(E, <_1, <_2\right)$ be a tertispecial double poset.
Let $\Par \EE$ denote the set of all $\EE$-partitions.
Let $w : E \to \left\{1, 2, 3, \ldots\right\}$. Let $G$ be a finite
group which acts on $E$. Assume that $G$ preserves both
relations $<_1$ and $<_2$, and also preserves $w$.
Then, $G$ acts also on the set $\Par \EE$ of all
$\EE$-partitions; namely, $\Par \EE$ is a $G$-subset of the $G$-set
$\left\{1, 2, 3, \ldots\right\}^E$ (see
Definition~\ref{def.G-sets.terminology} (d) for the definition of
the latter).
%We say that an $\EE$-partition $\pi$ is
%\textit{even} if every $g \in G$ satisfying $g \pi = \pi$
%is $E$-even. We say that a $G$-orbit $O$ on
%$\Par \EE$ is \textit{even} if its elements are
%even (or, equivalently, one of its elements is even).
For any $G$-orbit $O$ on
$\operatorname{Par}\EE$, we define a monomial $\xx_{O, w}$
by
\[ \xx_{O, w} = \xx_{\pi, w} \qquad \text{ for some element }
\pi \text{ of } O
\]
(this does not depend on the choice of $\pi$).
Let
\[
\Gamma\left(\EE, w, G\right) = \sum_{O\text{ is a }
G\text{-orbit on } \operatorname{Par}\EE} \xx_{O, w}
\]
and
\[
\Gamma^+\left(\EE, w, G\right) = \sum_{O\text{ is an }E\text{-coeven }
G\text{-orbit on } \operatorname{Par}\EE} \xx_{O, w} .
\]
Then, $\Gamma\left(\EE, w, G\right)$ and
$\Gamma^+\left(\EE, w, G\right)$ belong to $\QSym$ and satisfy
\[
S\left(\Gamma\left(\EE, w, G\right)\right)
= \left(-1\right)^{\left|E\right|}
\Gamma^+\left(\left(E, >_1, <_2\right), w, G\right) .
\]
\end{theorem}

This theorem, which combines Theorem~\ref{thm.antipode.Gammaw} with the
ideas of P\'olya enumeration, is inspired by Jochemko's reciprocity
result for order polynomials \cite[Theorem 2.8]{Joch}, which can be
obtained from it by specializations (see Section~\ref{sect.jochemko}
for the details of how Jochemko's result follows from ours).

We shall now review a number of particular cases of
Theorem~\ref{thm.antipode.Gammaw}. Details on most of them will
be provided in forthcoming work.

\begin{example}
\label{exam.antipode.Gammaw}

\begin{enumerate}

\item[(a)] Corollary~\ref{cor.antipode.Gamma}
follows from Theorem~\ref{thm.antipode.Gammaw} by letting $w$
be the function which is constantly $1$.

\item[(b)] Let
$\alpha = \left(\alpha_1, \alpha_2, \ldots, \alpha_\ell\right)$
be a composition of a nonnegative integer $n$, and let
$\EE = \left(E, <_1, <_2\right)$ be the double poset defined
in Example~\ref{exam.Gamma} (b). Let
$w : \left\{1, 2, \ldots, \ell\right\} \to \left\{ 1, 2, 3,
\ldots \right\}$ be the map sending every $i$ to $\alpha_i$.
As Example~\ref{exam.Gamma} (b) shows, we have
$\Gamma\left(\EE, w\right) = M_\alpha$.
%But it is also easy
%to see that $\Gamma\left(\left(
%\left\{1, 2, \ldots, \ell\right\}, <_1, <_2\right)\right)
%= \sum_{\gamma \text{ is a composition of } n
% [todo: details!]
Thus, applying Theorem~\ref{thm.antipode.Gammaw} to these $\EE$
and $w$ yields
\begin{align*}
S\left(M_\alpha\right)
&= \left(-1\right)^\ell
   \Gamma\left(\left(E, >_1, <_2\right), w\right)
= \left(-1\right)^\ell
  \sum_{i_1 \geq i_2 \geq \cdots \geq i_\ell}
  x_{i_1}^{\alpha_1} x_{i_2}^{\alpha_2} \cdots
      x_{i_\ell}^{\alpha_\ell} \\
&= \left(-1\right)^\ell
  \sum_{i_1 \leq i_2 \leq \cdots \leq i_\ell}
  x_{i_1}^{\alpha_\ell} x_{i_2}^{\alpha_{\ell-1}} \cdots
      x_{i_\ell}^{\alpha_1}
= \left(-1\right)^\ell
  \sum_{\substack{\gamma \text{ is a composition of } n ; \\
        D\left(\gamma\right) \subseteq
        D\left(\left(\alpha_\ell, \alpha_{\ell-1},
                     \ldots, \alpha_1\right)\right)}}
  M_\gamma .
\end{align*}
This is the formula for $S\left(M_\alpha\right)$
given in \cite[(4.26)]{Malve-Thesis}, in
\cite[Theorem 5.11]{Reiner}, and in
\cite[Theorem 4.1]{BenSag} (originally due to Ehrenborg
and to Malvenuto and Reutenauer). It also shows that the
$\Gamma\left(\EE, w\right)$ for varying $\EE$ and $w$ span
the $\kk$-module $\QSym$.

\item[(c)] Applying Corollary~\ref{cor.antipode.Gamma} to the
double poset of Example~\ref{exam.Gamma} (c) (where the relation
$<_2$ is chosen to be a total order) yields the formula for the
antipode of a fundamental quasisymmetric function
(\cite[(4.27)]{Malve-Thesis}, \cite[(5.9)]{Reiner}, \cite[Theorem 5.1]{BenSag}).

\item[(d)] Let us use the notations of Example~\ref{exam.dp}.
For any partition $\lambda$, let $\lambda^t$ denote the
conjugate partition of $\lambda$.
Let $\mu$ and $\lambda$ be two partitions satisfying
$\mu \subseteq \lambda$. Then, there is a bijection
$\tau : Y\left(\lambda / \mu\right) \to
Y\left(\lambda^t / \mu^t\right)$ sending each
$\left(i, j\right) \in Y\left(\lambda / \mu\right)$
to $\left(j, i\right)$. This bijection is an isomorphism
of double posets from
$\left(Y\left(\lambda / \mu\right), >_1, <_2\right)$
to
$\left(Y\left(\lambda^t / \mu^t\right), >_1, >_2\right)$.
Thus, applying Corollary~\ref{cor.antipode.Gamma} to the
tertispecial double poset $\mathbf{Y}\left(\lambda / \mu\right)$,
we obtain
\begin{align}
S\left(\Gamma\left(\mathbf{Y}\left(\lambda / \mu\right)\right)\right)
&= \left(-1\right)^{\left|\lambda / \mu\right|}
\Gamma\left(\left(Y\left(\lambda / \mu\right), >_1, <_2\right)\right)
\nonumber \\
&= \left(-1\right)^{\left|\lambda / \mu\right|}
\Gamma\left(\left(Y\left(\lambda^t / \mu^t\right), >_1, >_2\right)\right)
.
\label{eq.exam.antipode.Gammaw.schur.1}
\end{align}
But from Example~\ref{exam.Gamma} (d), we know that
$\Gamma\left(\mathbf{Y}\left(\lambda / \mu\right)\right)
= s_{\lambda / \mu}$. Moreover, a similar argument using
\cite[Remark 2.12]{Reiner} shows that
$\Gamma\left(\left(Y\left(\lambda^t / \mu^t\right), >_1, >_2\right)\right)
= s_{\lambda^t / \mu^t}$. Hence,
\eqref{eq.exam.antipode.Gammaw.schur.1} rewrites as
\begin{equation}
S\left(s_{\lambda / \mu}\right)
= \left(-1\right)^{\left|\lambda / \mu\right|}
s_{\lambda^t / \mu^t} .
\label{eq.exam.antipode.Gammaw.schur.2}
\end{equation}
This is a well-known formula, and is usually stated
for $S$ being the antipode of the Hopf algebra of symmetric
(rather than quasisymmetric) functions, but the latter antipode
is a restriction of the antipode of $\QSym$.

It is also possible (but more difficult) to derive
\eqref{eq.exam.antipode.Gammaw.schur.2} by using the double
poset $\mathbf{Y}_h\left(\lambda / \mu\right)$ instead of
$\mathbf{Y}\left(\lambda / \mu\right)$. (This boils down to what
was done in \cite[proof of Corollary 5.29]{Reiner}.)

\item[(e)] Two results of Benedetti and Sagan
\cite[Theorems 8.1--8.2]{BenSag} on the antipodes of immaculate
functions can be obtained from Corollary~\ref{cor.antipode.Gamma}
using dualization.

\end{enumerate}

\end{example}

\section{Lemmas: packed $\EE$-partitions and comultiplications}
\label{sect.lemmas}

We shall now prepare for the proofs of our results. To this end,
we introduce the notion of a \textit{packed map}.

\begin{definition}
\begin{itemize}

\item[(a)]
An \textit{initial interval} will mean a set of the form
$\left\{1, 2, \ldots, \ell\right\}$ for some $\ell \in \NN$.

\item[(b)]
If $T$ is a set and $\pi : T \to \left\{1, 2, 3, \ldots\right\}$ is
a map, then $\pi$ is said to be \textit{packed} if $\pi\left(T\right)$
is an initial interval. Clearly, this initial interval must be
$\left\{1, 2, \ldots, \left|\pi\left(T\right)\right|\right\}$.

\end{itemize}
\end{definition}

\begin{proposition}
\label{prop.Gammaw.packed}
Let $\EE = \left(E, <_1, <_2\right)$ be a double poset. Let
$w : E \to \left\{1, 2, 3, \ldots\right\}$ be a map.
For every packed map $\pi : E \to \left\{1, 2, 3, \ldots\right\}$,
we define $\ev_w \pi$ to be the composition
$\left(\alpha_1, \alpha_2, \ldots, \alpha_\ell\right)$, where
$\ell = \left|\pi\left(E\right)\right|$ (so that
$\pi\left(E\right) = \left\{1, 2, \ldots, \ell\right\}$, since
$\pi$ is packed),
and where each
$\alpha_i$ is defined as $\sum_{e \in \pi^{-1}\left(i\right)}
w\left(e\right)$.
Then,
\begin{equation}
\label{eq.prop.Gammaw.packed}
\Gamma\left(\EE , w\right)
= \sum_{\varphi \text{ is a packed } \EE\text{-partition}}
M_{\ev_w \varphi} .
\end{equation}
\end{proposition}

\begin{proof}[Proof of Proposition~\ref{prop.Gammaw.packed}.]
For every finite subset $T$ of
$\left\{1, 2, 3, \ldots\right\}$, there exists a unique strictly
increasing bijection $\left\{1, 2, \ldots,
\left|T\right|\right\} \to T$. We shall denote this bijection by
$r_T$.
For every map $\pi : E \to \left\{1, 2, 3, \ldots\right\}$, we
define the \textit{packing of $\pi$} as the map
$r_{\pi\left(E\right)}^{-1} \circ \pi : E \to
\left\{1, 2, 3, \ldots\right\}$; this is a packed map (indeed,
its image is
$\left\{1, 2, \ldots, \left|\pi\left(E\right)\right|\right\}$),
and will be
denoted by $\pack \pi$. This map $\pack \pi$ is an $\EE$-partition
if and only if $\pi$ is an $\EE$-partition\footnote{Indeed,
$\pack \pi = r_{\pi\left(E\right)}^{-1} \circ \pi$. Since
$r_{\pi\left(E\right)}$ is strictly increasing, we thus see that,
for any given $e \in E$ and $f \in E$, the equivalences
\[
\left(\left(\pack \pi\right)\left(e\right)
      \leq \left(\pack \pi\right)\left(f\right)\right)
\Longleftrightarrow
\left( \pi\left(e\right) \leq \pi\left(f\right) \right)
\]
and
\[
\left(\left(\pack \pi\right)\left(e\right)
      < \left(\pack \pi\right)\left(f\right)\right)
\Longleftrightarrow
\left( \pi\left(e\right) < \pi\left(f\right) \right)
\]
hold. Hence, $\pack \pi$ is an $\EE$-partition
if and only if $\pi$ is an $\EE$-partition.}.

We shall show that for every packed $\EE$-partition $\varphi$, we
have
\begin{equation}
\sum_{\pi\text{ is an }\EE\text{-partition; } \pack \pi = \varphi}
\xx_{\pi, w} = M_{\ev_w \varphi} .
\label{pf.prop.Gammaw.packed.1}
\end{equation}
Once this is proven, it will follow that
\begin{align*}
\Gamma\left(\EE , w\right)
&= \sum_{\pi\text{ is an }\EE\text{-partition}}
  \xx_{\pi, w}
= \sum_{\varphi \text{ is a packed } \EE\text{-partition}}
  \underbrace{\sum_{\pi\text{ is an }\EE\text{-partition; } \pack \pi = \varphi}
              \xx_{\pi, w}}_{\substack{ = M_{\ev_w \varphi} \\
                             \text{(by \eqref{pf.prop.Gammaw.packed.1})}
                             }} \\
& \qquad \left(\text{since } \pack \pi \text{ is a packed }
             \EE\text{-partition for every }
             \EE\text{-partition } \pi\right) \\
&= \sum_{\varphi \text{ is a packed } \EE\text{-partition}}
M_{\ev_w \varphi} ,
\end{align*}
and Proposition~\ref{prop.Gammaw.packed} will be proven.

So it remains to prove \eqref{pf.prop.Gammaw.packed.1}. Let $\varphi$
be a packed $\EE$-partition. Let
$\ell = \left|\varphi\left(E\right)\right|$; thus
$\varphi\left(E\right) = \left\{1, 2, \ldots, \ell\right\}$
(since $\varphi$ is packed).
Let $\alpha_i = \sum_{e \in \varphi^{-1}\left(i\right)}
w\left(e\right)$ for every $i \in \left\{ 1, 2, \ldots, \ell \right\}$;
thus,
$\ev_w \varphi = \left(\alpha_1, \alpha_2, \ldots, \alpha_\ell\right)$
(by the definition of $\ev_w \varphi$).
Then, the right hand side of
\eqref{pf.prop.Gammaw.packed.1} rewrites as follows:
\begin{align}
M_{\ev_w \varphi}
&= \sum_{i_1 < i_2 < \cdots < i_\ell}
\underbrace{x_{i_1}^{\alpha_1} x_{i_2}^{\alpha_2} \cdots x_{i_\ell}^{\alpha_\ell}}
_{= \prod_{k = 1}^\ell x_{i_k}^{\alpha_k}}
= \sum_{i_1 < i_2 < \cdots < i_\ell}
\prod_{k = 1}^\ell \underbrace{x_{i_k}^{\alpha_k}}
                              _{\substack{= x_{i_k}^{\sum_{e \in \varphi^{-1}\left(k\right)}
                                w\left(e\right)} \\
                                \text{(since } \alpha_k = \sum_{e \in \varphi^{-1}\left(k\right)}
                                w\left(e\right) \text{)}}}
\nonumber \\
&= \sum_{i_1 < i_2 < \cdots < i_\ell}
\prod_{k = 1}^\ell \underbrace{x_{i_k}^{\sum_{e \in \varphi^{-1}\left(k\right)}
                                w\left(e\right)}}
                              _{\substack{= \prod_{e \in \varphi^{-1}\left(k\right)}
                                x_{i_k}^{w\left(e\right)} \\
                                = \prod_{e \in E;\ \varphi\left(e\right) = k}
                                x_{i_k}^{w\left(e\right)} }}
= \sum_{i_1 < i_2 < \cdots < i_\ell}
\prod_{k = 1}^\ell \prod_{e \in E;\ \varphi\left(e\right) = k}
\underbrace{x_{i_k}^{w\left(e\right)}}_{\substack{
         =x_{i_{\varphi\left(e\right)}}^{w\left(e\right)} \\
         \text{(since } k = \varphi\left(e\right) \text{)}
         }}
\nonumber \\
&= \sum_{i_1 < i_2 < \cdots < i_\ell}
\underbrace{\prod_{k = 1}^\ell \prod_{e \in E;\ \varphi\left(e\right) = k}
x_{i_{\varphi\left(e\right)}}^{w\left(e\right)}}
_{ = \prod_{e \in E} x_{i_{\varphi\left(e\right)}}^{w\left(e\right)}}
= \sum_{i_1 < i_2 < \cdots < i_\ell}
\prod_{e \in E} x_{i_{\varphi\left(e\right)}}^{w\left(e\right)}
\nonumber \\
&= \sum_{T \subseteq \left\{1, 2, 3, \ldots\right\} ; \ \left|T\right| = \ell}
\underbrace{\prod_{e \in E} x_{r_T\left(\varphi\left(e\right)\right)}^{w\left(e\right)}}
           _{= \prod_{e \in E} x_{\left(r_T\circ \varphi\right)\left(e\right)}^{w\left(e\right)}
             = \xx_{r_T\circ\varphi,w}}
= \sum_{T \subseteq \left\{1, 2, 3, \ldots\right\} ; \ \left|T\right| = \ell}
\xx_{r_T\circ\varphi,w}
\label{pf.prop.Gammaw.packed.1.pf.1}
\end{align}
\footnote{In the second-to-last equality, we have used the fact that
the strictly increasing sequences
$\left(i_1 < i_2 < \cdots < i_\ell\right)$ of positive integers are
in bijection with the subsets
$T \subseteq \left\{1, 2, 3, \ldots\right\}$
such that $\left|T\right| = \ell$. The bijection sends a sequence
$\left(i_1 < i_2 < \cdots < i_\ell\right)$ to the set of its entries;
its inverse map sends every $T$ to the sequence
$\left(r_T\left(1\right), r_T\left(2\right), \ldots,
r_T\left(\left|T\right|\right)\right)$.}.

On the other hand, recall that $\varphi$ is an $\EE$-partition.
Hence, every map $\pi$ satisfying $\pack \pi = \varphi$
is an $\EE$-partition (because, as we know, $\pack \pi$ is an
$\EE$-partition if and only if $\pi$ is an $\EE$-partition).
Thus, the $\EE$-partitions $\pi$ satisfying
$\pack \pi = \varphi$ are precisely the maps
$\pi : E \to \left\{1, 2, 3, \ldots\right\}$ satisfying
$\pack \pi = \varphi$. Hence,
\begin{align*}
\sum_{\pi\text{ is an }\EE\text{-partition; } \pack \pi = \varphi}
\xx_{\pi, w}
&= \sum_{\pi : E \to \left\{1, 2, 3, \ldots\right\} \text{; } \pack \pi = \varphi}
\xx_{\pi, w} \\
&= \sum_{T \subseteq \left\{1, 2, 3, \ldots\right\} ; \ \left|T\right| = \ell}
\sum_{\pi : E \to \left\{1, 2, 3, \ldots\right\} \text{; } \pack \pi = \varphi
\text{; } \pi\left(E\right) = T}
\xx_{\pi, w}
\end{align*}
(because if $\pi : E \to \left\{1, 2, 3, \ldots\right\}$ is a map
satisfying $\pack \pi = \varphi$, then
$\left|\pi\left(E\right)\right| = \ell$\ \ \ \ \footnote{\textit{Proof.}
Let $\pi : E \to \left\{1, 2, 3, \ldots\right\}$ be a map
satisfying $\pack \pi = \varphi$. The definition of $\pack \pi$
yields $\pack \pi = r_{\pi\left(E\right)}^{-1} \circ \pi$. Hence,
$\left|\left(\pack \pi\right)\left(E\right)\right|
= \left|\left(r_{\pi\left(E\right)}^{-1} \circ \pi\right)\left(E\right)\right|
= \left|r_{\pi\left(E\right)}^{-1} \left(\pi\left(E\right)\right)\right|
= \left|\pi\left(E\right)\right|$
(since $r_{\pi\left(E\right)}^{-1}$ is a bijection). Since
$\pack \pi = \varphi$, this rewrites as
$\left|\varphi\left(E\right)\right| = \left|\pi\left(E\right)\right|$.
Hence, $ \left|\pi\left(E\right)\right|
= \left|\varphi\left(E\right)\right| = \ell$, qed.}). But for every
$\ell$-element subset $T$ of
$\left\{1, 2, 3, \ldots\right\}$, there exists exactly
one $\pi : E \to \left\{1, 2, 3, \ldots\right\}$ satisfying
$\pack \pi = \varphi$ and $\pi\left(E\right) = T$: namely,
$\pi = r_T \circ \varphi$\ \ \ \ \footnote{\textit{Proof.}
Let $T$ be an $\ell$-element subset of $\left\{
1,2,3,\ldots\right\}  $. We need to show that there exists exactly one
$\pi:E\rightarrow\left\{  1,2,3,\ldots\right\}  $ satisfying
$\operatorname{pack}\pi=\varphi$ and $\pi\left(  E\right)  =T$: namely,
$\pi=r_{T}\circ\varphi$. In other words, we need to prove the following two claims:

\textit{Claim 1:} The map $r_{T}\circ\varphi$ is a map $\pi:E\rightarrow
\left\{  1,2,3,\ldots\right\}  $ satisfying $\operatorname{pack}\pi=\varphi$
and $\pi\left(  E\right)  =T$.

\textit{Claim 2:} If $\pi:E\rightarrow\left\{  1,2,3,\ldots\right\}  $ is a
map satisfying $\operatorname{pack}\pi=\varphi$ and $\pi\left(  E\right)  =T$,
then $\pi=r_{T}\circ\varphi$.

\textit{Proof of Claim 1.} We have $\left(  r_{T}\circ\varphi\right)  \left(
E\right)  =r_{T}\left(  \underbrace{\varphi\left(  E\right)  }_{=\left\{
1,2,\ldots,\ell\right\}  }\right)  =r_{T}\left(  \left\{  1,2,\ldots
,\underbrace{\ell}_{\substack{=\left\vert T\right\vert \\\text{(since }T\text{
is }\ell\text{-element)}}}\right\}  \right)  =r_{T}\left(  \left\{
1,2,\ldots,\left\vert T\right\vert \right\}  \right)  =T$ (by the definition
of $r_{T}$). Now, the definition of $\operatorname{pack}\left(  r_{T}%
\circ\varphi\right)  $ shows that
\begin{align*}
\operatorname{pack}\left(  r_{T}\circ\varphi\right)   & =r_{\left(  r_{T}%
\circ\varphi\right)  \left(  E\right)  }^{-1}\circ\left(  r_{T}\circ
\varphi\right)  =r_{T}^{-1}\circ\left(  r_{T}\circ\varphi\right)
\ \ \ \ \ \ \ \ \ \ \left(  \text{since }\left(  r_{T}\circ\varphi\right)
\left(  E\right)  =T\right) \\
& =\varphi.
\end{align*}
Thus, the map $r_{T}\circ\varphi:E\rightarrow\left\{  1,2,3,\ldots\right\}  $
satisfies $\operatorname{pack}\left(  r_{T}\circ\varphi\right)  =\varphi$ and
$\left(  r_{T}\circ\varphi\right)  \left(  E\right)  =T$. In other words,
$r_{T}\circ\varphi$ is a map $\pi:E\rightarrow\left\{  1,2,3,\ldots\right\}  $
satisfying $\operatorname{pack}\pi=\varphi$ and $\pi\left(  E\right)  =T$.
This proves Claim 1.

\textit{Proof of Claim 2.} Let $\pi:E\rightarrow\left\{  1,2,3,\ldots\right\}
$ be a map satisfying $\operatorname{pack}\pi=\varphi$ and $\pi\left(
E\right)  =T$. The definition of $\operatorname{pack}\pi$ shows that
$\operatorname{pack}\pi=r_{\pi\left(  E\right)  }^{-1}\circ\pi=r_{T}^{-1}%
\circ\pi$ (since $\pi\left(  E\right)  =T$). Hence, $r_{T}^{-1}\circ
\pi=\operatorname*{pack}\pi=\varphi$, so that $\pi=r_{T}\circ\varphi$. This
proves Claim 2.

Now, both Claims 1 and 2 are proven; hence, our proof is complete.
}. Therefore, for every $\ell$-element subset $T$ of
$\left\{1, 2, 3, \ldots\right\}$, we have
\[
\sum_{\pi : E \to \left\{1, 2, 3, \ldots\right\} \text{; } \pack \pi = \varphi
\text{; } \pi\left(E\right) = T}
\xx_{\pi, w}
= \xx_{r_T\circ\varphi,w} .
\]
Hence,
\begin{align*}
\sum_{\pi\text{ is an }\EE\text{-partition; } \pack \pi = \varphi}
\xx_{\pi, w}
&= \sum_{T \subseteq \left\{1, 2, 3, \ldots\right\} ; \ \left|T\right| = \ell}
\underbrace{\sum_{\pi : E \to \left\{1, 2, 3, \ldots\right\} \text{; } \pack \pi = \varphi
\text{; } \pi\left(E\right) = T}
\xx_{\pi, w}}_{= \xx_{r_T\circ\varphi,w}} \\
&= \sum_{T \subseteq \left\{1, 2, 3, \ldots\right\} ; \ \left|T\right| = \ell}
\xx_{r_T\circ\varphi,w}
= M_{\ev_w \varphi}
\end{align*}
(by \eqref{pf.prop.Gammaw.packed.1.pf.1}).
Thus, \eqref{pf.prop.Gammaw.packed.1}
is proven, and with it Proposition~\ref{prop.Gammaw.packed}.
\end{proof}

\begin{proof}[Proof of Proposition~\ref{prop.Gammaw.qsym}.]
Proposition~\ref{prop.Gammaw.qsym} follows immediately from
Proposition~\ref{prop.Gammaw.packed}.
\end{proof}

We shall now describe the coproduct of $\Gamma\left(\EE, w\right)$,
essentially giving the proof that is left to the reader in
\cite[Theorem 2.2]{Mal-Reu-DP}.

\begin{definition}
Let $\EE = \left(E, <_1, <_2\right)$ be a double poset.

\begin{itemize}

\item[(a)]
Then, $\Adm \EE$ will mean the set of all pairs
$\left(P, Q\right)$, where $P$ and $Q$ are subsets of $E$ satisfying
$P \cap Q = \varnothing$ and $P \cup Q = E$ and having the property
that no $p \in P$ and $q \in Q$ satisfy $q <_1 p$. These pairs
$\left(P, Q\right)$ are
called the \textit{admissible partitions} of $\EE$. (In the
terminology of \cite{Mal-Reu-DP}, they are the
\textit{decompositions} of $\left(E, <_1\right)$.)

\item[(b)] For
any subset $T$ of $E$, we let $\EE\mid_T$ denote the double poset
$\left(T, <_1, <_2\right)$, where $<_1$ and $<_2$ (by abuse of
notation) denote the restrictions of the relations $<_1$ and $<_2$
to $T$.

\end{itemize}
\end{definition}

\begin{proposition}
\label{prop.Gammaw.coprod}
Let $\EE = \left(E, <_1, <_2\right)$ be a double poset. Let
$w : E \to \left\{1, 2, 3, \ldots\right\}$ be a map.
Then,
\begin{equation}
\label{eq.prop.Gammaw.coprod}
\Delta\left(\Gamma\left(\EE, w\right)\right)
= \sum_{\left(P, Q\right) \in \Adm \EE}
\Gamma\left(\EE\mid_P, w\mid_P\right)
\otimes \Gamma\left(\EE\mid_Q, w\mid_Q\right) .
\end{equation}
\end{proposition}

A particular case of Proposition~\ref{prop.Gammaw.coprod} (namely,
the case when $w\left(e\right) = 1$ for each $e \in E$) appears in
\cite[Th\'eor\`eme 4.16]{Malve-Thesis}.

\begin{vershort}
We shall now outline a proof of this fact. The proof relies on
a simple bijection that an experienced combinatorialist will
have no trouble finding (and proving even less); let us just
give a brief outline of the argument\footnote{See the detailed
version of this note for an (almost) completely written-out
proof; I am afraid that the additional level of detail is of
no help to the understanding.}:

\begin{proof}[Proof of Proposition~\ref{prop.Gammaw.coprod}.]
Whenever
$\alpha = \left(\alpha_1, \alpha_2, \ldots, \alpha_\ell\right)$
is a composition and $k \in \left\{0, 1, \ldots, \ell\right\}$,
we introduce the notation
$\alpha\left[:k\right]$ for the composition
$\left(\alpha_1, \alpha_2, \ldots, \alpha_k\right)$, and the
notation $\alpha\left[k:\right]$ for the composition
$\left(\alpha_{k+1}, \alpha_{k+2}, \ldots, \alpha_\ell\right)$.
Now, the formula \eqref{eq.coproduct.M} can be rewritten as
follows:
\begin{align}
\label{pf.prop.Gammaw.coprod.DeltaM}
\Delta \left( M_\alpha \right)
&= \sum_{k=0}^{\ell} M_{\alpha\left[:k\right]}
\otimes M_{\alpha\left[k:\right]} \\
& \qquad \text{ for every } \ell \in \NN \text{ and every composition } \alpha
\text{ with } \ell \text{ entries.} \nonumber
\end{align}

Now, applying $\Delta$ to the equality
\eqref{eq.prop.Gammaw.packed} yields
\begin{align}
\Delta\left(\Gamma\left(\EE , w\right)\right)
&= \sum_{\varphi \text{ is a packed } \EE\text{-partition}}
\underbrace{\Delta\left(M_{\ev_w \varphi}\right)}_{
  \substack{ = \sum_{k=0}^{\left|\varphi\left(E\right)\right|}
  M_{\left(\ev_w \varphi\right)\left[:k\right]} \otimes
  M_{\left(\ev_w \varphi\right)\left[k:\right]} \\
  \text{(by \eqref{pf.prop.Gammaw.coprod.DeltaM})} }}
 \nonumber \\
&= \sum_{\varphi \text{ is a packed } \EE\text{-partition}}
\sum_{k=0}^{\left|\varphi\left(E\right)\right|}
M_{\left(\ev_w \varphi\right)\left[:k\right]} \otimes
M_{\left(\ev_w \varphi\right)\left[k:\right]} .
\label{pf.Gammaw.coprod.lhs}
\end{align}

On the other hand, rewriting each of the tensorands on the right
hand side of \eqref{eq.prop.Gammaw.coprod} using
\eqref{eq.prop.Gammaw.packed}, we obtain
\begin{align*}
%\label{pf.Gammaw.coprod.rhs}
& \sum_{\left(P, Q\right) \in \Adm \EE}
\Gamma\left(\EE\mid_P, w\mid_P\right)
\otimes \Gamma\left(\EE\mid_Q, w\mid_Q\right) \\
&= \sum_{\left(P, Q\right) \in \Adm \EE}
\left(\sum_{\varphi \text{ is a packed } \EE\mid_P\text{-partition}}
M_{\ev_{w\mid_P} \varphi}\right)
\otimes
\left(\sum_{\varphi \text{ is a packed } \EE\mid_Q\text{-partition}}
M_{\ev_{w\mid_Q} \varphi}\right) \\
& = \sum_{\left(P, Q\right) \in \Adm \EE}
\left(\sum_{\sigma \text{ is a packed } \EE\mid_P\text{-partition}}
M_{\ev_{w\mid_P} \sigma}\right)
\otimes
\left(\sum_{\tau \text{ is a packed } \EE\mid_Q\text{-partition}}
M_{\ev_{w\mid_Q} \tau}\right) \\
& = \sum_{\left(P, Q\right) \in \Adm \EE}
\sum_{\sigma \text{ is a packed } \EE\mid_P\text{-partition}}
\sum_{\tau \text{ is a packed } \EE\mid_Q\text{-partition}}
M_{\ev_{w\mid_P} \sigma}
\otimes
M_{\ev_{w\mid_Q} \tau} .
\end{align*}
We need to prove that the right hand sides of this equality and of
\eqref{pf.Gammaw.coprod.lhs} are equal (because then, it will follow
that so are the left hand sides, and thus
Proposition~\ref{prop.Gammaw.coprod} will be proven). For this, it
is clearly enough to exhibit a bijection between
\begin{itemize}
\item the pairs
$\left(\varphi, k\right)$ consisting of a packed $\EE$-partition
$\varphi$ and a
$k \in \left\{0, 1, \ldots, \left|\varphi\left(E\right)\right|
\right\}$
\end{itemize}
and
\begin{itemize}
\item the triples $\left(\left(P, Q\right), \sigma, \tau
\right)$ consisting of a $\left(P, Q\right) \in \Adm \EE$, a packed
$\EE\mid_P$-partition $\sigma$ and a packed $\EE\mid_Q$-partition
$\tau$
\end{itemize}
which bijection has the property that
whenever it maps $\left(\varphi, k\right)$ to
$\left(\left(P, Q\right), \sigma, \tau\right)$,
we have the equalities
$\left(\ev_w \varphi\right)\left[:k\right]
= \ev_{w\mid_P}\sigma$
and
$\left(\ev_w \varphi\right)\left[k:\right]
= \ev_{w\mid_Q}\tau$.
Such a bijection is easy to construct: Given
$\left(\varphi, k\right)$, it sets
$P = \varphi^{-1}\left(\left\{1, 2, \ldots, k\right\}\right)$,
$Q = \varphi^{-1}\left(\left\{k+1, k+2, \ldots, \left|\varphi\left(E\right)\right|\right\}\right)$,
$\sigma = \varphi\mid_P$ and
$\tau = \pack \left(\varphi\mid_Q\right)$\ \ \ \ \footnote{We
notice that these $P$, $Q$, $\sigma$ and $\tau$ satisfy
$\sigma\left(e\right) = \varphi\left(e\right)$ for every
$e \in P$, and $\tau\left(e\right) = \varphi\left(e\right) - k$
for every $e \in Q$.}. Conversely, given
$\left(\left(P, Q\right), \sigma, \tau\right)$,
the inverse bijection
sets $k = \left|\sigma\left(P\right)\right|$ and constructs
$\varphi$ as the map $E \to \left\{1, 2, 3, \ldots\right\}$
which sends every $e \in E$ to
$\begin{cases} \sigma\left(e\right), &\mbox{if } e \in P; \\
\tau\left(e\right) + k, &\mbox{if } e \in Q \end{cases}$.
Proving that this alleged bijection and its alleged inverse
bijection are well-defined and actually mutually inverse is
straightforward and left to the reader\footnote{The only
part of the argument that is a bit trickier is proving the
well-definedness of the inverse bijection: We need to show
that if $\left(\left(P, Q\right), \sigma, \tau
\right)$ is a triple consisting of a
$\left(P, Q\right) \in \Adm \EE$, a packed
$\EE\mid_P$-partition $\sigma$ and a packed $\EE\mid_Q$-partition
$\tau$, and if we set $k = \left|\sigma\left(P\right)\right|$,
then the map $\varphi : E \to \left\{1, 2, 3, \ldots\right\}$
which sends every $e \in E$ to
$\begin{cases} \sigma\left(e\right), &\mbox{if } e \in P; \\
\tau\left(e\right) + k, &\mbox{if } e \in Q \end{cases}$
is actually a packed $\EE$-partition.

Indeed, it is clear that this map $\varphi$ is packed. It remains
to show that it is an $\EE$-partition. To do so, we must prove
the following two claims:

\textit{Claim 1:} Every $e \in E$ and $f \in E$ satisfying
$e <_1 f$ satisfy
$\varphi\left(e\right) \leq \varphi\left(f\right)$.

\textit{Claim 2:} Every $e \in E$ and $f \in E$ satisfying
$e <_1 f$ and $f <_2 e$ satisfy
$\varphi\left(e\right) < \varphi\left(f\right)$.

We shall only prove Claim 1 (as the proof of Claim 2 is
similar). So let $e \in E$ and $f \in E$ be such that
$e <_1 f$. We need to show that
$\varphi\left(e\right) \leq \varphi\left(f\right)$.
We are in one of the following four cases:

\textit{Case 1:} We have $e \in P$ and $f \in P$.

\textit{Case 2:} We have $e \in P$ and $f \in Q$.

\textit{Case 3:} We have $e \in Q$ and $f \in P$.

\textit{Case 4:} We have $e \in Q$ and $f \in Q$.

In Case 1, our claim
$\varphi\left(e\right) \leq \varphi\left(f\right)$ follows
from the assumption that $\sigma$ is an
$\EE\mid_P$-partition (because in Case 1,
we have $\varphi\left(e\right) = \sigma\left(e\right)$
and $\varphi\left(f\right) = \sigma\left(f\right)$).
In Case 4, it follows from the
assumption that $\tau$ is an $\EE\mid_Q$-partition
(since in Case 4, we have
$\varphi\left(e\right) = \tau\left(e\right) + k$ and
$\varphi\left(f\right) = \tau\left(f\right) + k$). In
Case 2, it clearly holds (indeed,
if $e \in P$, then the definition of $\varphi$ yields
$\varphi\left(e\right) = \sigma\left(e\right) \leq k$,
and if $f \in Q$, then
the definition of $\varphi$ yields
$\varphi\left(f\right) = \tau\left(f\right) + k > k$;
therefore, in Case 2, we have
$\varphi\left(e\right) \leq k < \varphi\left(f\right)$).
Finally, Case 3 is impossible (because having $e \in Q$
and $f \in P$ and $e <_1 f$ would contradict
$\left(P, Q\right) \in \Adm \EE$). Thus, we have proven the
claim in each of the four cases, and consequently Claim 1 is
proven. As we have said above, Claim 2 is proven similarly.}.
\end{proof}
\end{vershort}

\begin{verlong}
The proof of Proposition \ref{prop.Gammaw.coprod} is based upon a simple
bijection. We shall introduce it after some preparations.

\begin{lemma}
\label{lem.Gammaw.coprod.bij.a}Let ${\mathbf{E}}=\left(  E,<_{1},<_{2}\right)
$ be a double poset.

Let $\mathcal{S}$ be the set of all pairs $\left(  \varphi,k\right)  $
consisting of a packed ${\mathbf{E}}$-partition $\varphi$ and a $k\in\left\{
0,1,\ldots,\left\vert \varphi\left(  E\right)  \right\vert \right\}  $.

Let $\mathcal{T}$ be the set of all triples $\left(  \left(  P,Q\right)
,\sigma,\tau\right)  $ consisting of a $\left(  P,Q\right)  \in
\operatorname{Adm}{\mathbf{E}}$, a packed ${\mathbf{E}}\mid_{P}$-partition
$\sigma$ and a packed ${\mathbf{E}}\mid_{Q}$-partition $\tau$.

For every $\ell\in\mathbb{Z}$, we let $\operatorname*{add}\nolimits_{\ell}$
denote the bijective map $\mathbb{Z}\rightarrow\mathbb{Z},\ z\mapsto z+\ell$.

Fix $\left(  \varphi,k\right)  \in\mathcal{S}$. Set%
\begin{align}
P  &  =\varphi^{-1}\left(  \left\{  1,2,\ldots,k\right\}  \right)
,\ \ \ \ \ \ \ \ \ \ Q=\varphi^{-1}\left(  \left\{  k+1,k+2,\ldots,\left\vert
\varphi\left(  E\right)  \right\vert \right\}  \right)
,\label{eq.lem.Gammaw.coprod.bij.def1}\\
\sigma &  =\varphi\mid_{P}\ \ \ \ \ \ \ \ \ \ \text{and}%
\ \ \ \ \ \ \ \ \ \ \tau=\operatorname*{add}\nolimits_{-k}\circ\left(
\varphi\mid_{Q}\right)  . \label{eq.lem.Gammaw.coprod.bij.def2}%
\end{align}
Then, $\left(  \left(  P,Q\right)  ,\sigma,\tau\right)  \in\mathcal{T}$.
\end{lemma}

\begin{lemma}
\label{lem.Gammaw.coprod.bij.b}Let $\mathbf{E}=\left(  E,<_{1},<_{2}\right)  $
be a double poset. Let $\mathcal{S}$ and $\mathcal{T}$ be defined as in Lemma
\ref{lem.Gammaw.coprod.bij.a}.

Fix $\left(  \left(  P,Q\right)  ,\sigma,\tau\right)  \in\mathcal{T}$. Set
$k=\left\vert \sigma\left(  P\right)  \right\vert $, and let $\varphi$ be the
map $E\rightarrow\left\{  1,2,3,\ldots\right\}  $ which sends every $e\in E$
to $%
\begin{cases}
\sigma\left(  e\right)  , & \text{if }e\in P;\\
\tau\left(  e\right)  +k, & \text{if }e\in Q
\end{cases}
$. Then, $\left(  \varphi,k\right)  \in\mathcal{S}$.
\end{lemma}

\begin{lemma}
\label{lem.Gammaw.coprod.bij}Let $\mathbf{E}=\left(  E,<_{1},<_{2}\right)  $
be a double poset. Let $\mathcal{S}$, $\mathcal{T}$ and $\operatorname*{add}%
\nolimits_{\ell}$ be defined as in Lemma \ref{lem.Gammaw.coprod.bij.a}.

Define a map $\Phi:\mathcal{S}\rightarrow\mathcal{T}$ as follows: Let $\left(
\varphi,k\right)  \in\mathcal{S}$. Then, define $P$, $Q$, $\sigma$ and $\tau$
by (\ref{eq.lem.Gammaw.coprod.bij.def1}) and
(\ref{eq.lem.Gammaw.coprod.bij.def2}). From Lemma
\ref{lem.Gammaw.coprod.bij.a}, we know that $\left(  \left(  P,Q\right)
,\sigma,\tau\right)  \in\mathcal{T}$. Define $\Phi\left(  \varphi,k\right)  $
to be $\left(  \left(  P,Q\right)  ,\sigma,\tau\right)  $. Thus, a map
$\Phi:\mathcal{S}\rightarrow\mathcal{T}$ is defined.

Define a map $\Psi:\mathcal{T}\rightarrow\mathcal{S}$ as follows: Let $\left(
\left(  P,Q\right)  ,\sigma,\tau\right)  \in\mathcal{T}$. Set $k=\left\vert
\sigma\left(  P\right)  \right\vert $, and let $\varphi$ be the map
$E\rightarrow\left\{  1,2,3,\ldots\right\}  $ which sends every $e\in E$ to $%
\begin{cases}
\sigma\left(  e\right)  , & \text{if }e\in P;\\
\tau\left(  e\right)  +k, & \text{if }e\in Q
\end{cases}
$. From Lemma \ref{lem.Gammaw.coprod.bij.b}, we know that $\left(
\varphi,k\right)  \in\mathcal{S}$. Set $\Psi\left(  \left(  P,Q\right)
,\sigma,\tau\right)  =\left(  \varphi,k\right)  $. Thus, a map $\Psi
:\mathcal{T}\rightarrow\mathcal{S}$ is defined.

The maps $\Phi:\mathcal{S}\rightarrow\mathcal{T}$ and $\Psi:\mathcal{T}%
\rightarrow\mathcal{S}$ are mutually inverse.
\end{lemma}

The preceding three lemmas should be obvious if the reader has
\textquotedblleft the right picture in their mind\textquotedblright. The
following proof is merely a formalization of the argument that such a picture
would straightforwardly produce; we are not sure whether it is actually worth
reading (as opposed to trying to conjure \textquotedblleft the right
picture\textquotedblright).

\begin{proof}
[Proof of Lemma \ref{lem.Gammaw.coprod.bij.a}.]We have $\left(  \varphi
,k\right)  \in\mathcal{S}$. Thus, $\varphi$ is a packed $\mathbf{E}%
$-partition, and $k$ is an element of $\left\{  0,1,\ldots,\left\vert
\varphi\left(  E\right)  \right\vert \right\}  $ (by the definition of
$\mathcal{S}$).

We have $\varphi\left(  E\right)  =\left\{  1,2,\ldots,\left\vert
\varphi\left(  E\right)  \right\vert \right\}  $ (since $\varphi$ is packed).

Now, $\left(  P,Q\right)  \in\operatorname*{Adm}\mathbf{E}$%
\ \ \ \ \footnote{\textit{Proof.} It is clear that $P$ and $Q$ are subsets of
$E$. Also, from (\ref{eq.lem.Gammaw.coprod.bij.def1}), we obtain%
\begin{align*}
P\cap Q  &  =\varphi^{-1}\left(  \left\{  1,2,\ldots,k\right\}  \right)
\cap\varphi^{-1}\left(  \left\{  k+1,k+2,\ldots,\left\vert \varphi\left(
E\right)  \right\vert \right\}  \right) \\
&  =\varphi^{-1}\left(  \underbrace{\left\{  1,2,\ldots,k\right\}
\cap\left\{  k+1,k+2,\ldots,\left\vert \varphi\left(  E\right)  \right\vert
\right\}  }_{=\varnothing}\right)  =\varphi^{-1}\left(  \varnothing\right)
=\varnothing
\end{align*}
and%
\begin{align*}
P\cup Q  &  =\varphi^{-1}\left(  \left\{  1,2,\ldots,k\right\}  \right)
\cup\varphi^{-1}\left(  \left\{  k+1,k+2,\ldots,\left\vert \varphi\left(
E\right)  \right\vert \right\}  \right) \\
&  =\varphi^{-1}\left(  \underbrace{\left\{  1,2,\ldots,k\right\}
\cup\left\{  k+1,k+2,\ldots,\left\vert \varphi\left(  E\right)  \right\vert
\right\}  }_{\substack{=\left\{  1,2,\ldots,\left\vert \varphi\left(
E\right)  \right\vert \right\}  =\varphi\left(  E\right)  \\\text{(since
}\varphi\text{ is packed)}}}\right)  =\varphi^{-1}\left(  \varphi\left(
E\right)  \right)  =E.
\end{align*}
Hence, in order to prove that $\left(  P,Q\right)  \in\operatorname*{Adm}%
\mathbf{E}$, it remains to show that no $p\in P$ and $q\in Q$ satisfy
$q<_{1}p$.
\par
Let us assume the contrary (for the sake of contradiction). Thus, let $p\in P$
and $q\in Q$ be such that $q<_{1}p$. Since $\varphi$ is an $\mathbf{E}%
$-partition, we have $\varphi\left(  q\right)  \leq\varphi\left(  p\right)  $
(because $q<_{1}p$). But $p\in P=\varphi^{-1}\left(  \left\{  1,2,\ldots
,k\right\}  \right)  $, so that $\varphi\left(  p\right)  \leq k$. On the
other hand, $q\in Q=\varphi^{-1}\left(  \left\{  k+1,k+2,\ldots,\left\vert
\varphi\left(  E\right)  \right\vert \right\}  \right)  $, so that
$\varphi\left(  q\right)  >k$. This contradicts $\varphi\left(  q\right)
\leq\varphi\left(  p\right)  \leq k$. This contradiction shows that our
assumption was false. Hence, the proof of $\left(  P,Q\right)  \in
\operatorname*{Adm}\mathbf{E}$ is complete.}. Furthermore, it is
straightforward to see that for every subset $T$ of $E$,%
\begin{equation}
\text{the map }\varphi\mid_{T}\text{ is an }\mathbf{E}\mid_{T}%
\text{-partition} \label{pf.lem.Gammaw.coprod.bij.a.1}%
\end{equation}
Applying this to $T=P$, we conclude that $\varphi\mid_{P}$ is an
$\mathbf{E}\mid_{P}$-partition.

Since $P=\varphi^{-1}\left(  \left\{  1,2,\ldots,k\right\}  \right)  $, we
have $\varphi\left(  P\right)  \subseteq\left\{  1,2,\ldots,k\right\}  $.
Moreover, this inclusion is actually an equality (since $\varphi\left(
E\right)  =\left\{  1,2,\ldots,\left\vert \varphi\left(  E\right)  \right\vert
\right\}  $)\ \ \ \ \footnote{The proof in more detail: Let $g\in\left\{
1,2,\ldots,k\right\}  $. Then, $g\in\left\{  1,2,\ldots,k\right\}
\subseteq\left\{  1,2,\ldots,\left\vert \varphi\left(  E\right)  \right\vert
\right\}  =\varphi\left(  E\right)  $. Thus, there exists some $e\in E$ such
that $g=\varphi\left(  e\right)  $. Consider this $e$. From $\varphi\left(
e\right)  =g\in\left\{  1,2,\ldots,k\right\}  $, we obtain $e\in\varphi
^{-1}\left(  \left\{  1,2,\ldots,k\right\}  \right)  =P$. Thus, $\varphi
\left(  e\right)  \in\varphi\left(  P\right)  $, so that $g=\varphi\left(
e\right)  \in\varphi\left(  P\right)  $. Now, let us forget that we fixed $g$.
We thus have proven that $g\in\varphi\left(  P\right)  $ for every
$g\in\left\{  1,2,\ldots,k\right\}  $. In other words, $\left\{
1,2,\ldots,k\right\}  \subseteq\varphi\left(  P\right)  $. Combining this with
$\varphi\left(  P\right)  \subseteq\left\{  1,2,\ldots,k\right\}  $, we obtain
$\varphi\left(  P\right)  =\left\{  1,2,\ldots,k\right\}  $, qed.}. In other
words, we have%
\begin{equation}
\varphi\left(  P\right)  =\left\{  1,2,\ldots,k\right\}  .
\label{pf.lem.Gammaw.coprod.bij.a.3}%
\end{equation}
Similarly,
\begin{equation}
\varphi\left(  Q\right)  =\left\{  k+1,k+2,\ldots,\left\vert \varphi\left(
E\right)  \right\vert \right\}  . \label{pf.lem.Gammaw.coprod.bij.a.4}%
\end{equation}
Hence,%
\begin{align*}
\left(  \operatorname*{add}\nolimits_{-k}\circ\left(  \varphi\mid_{Q}\right)
\right)  \left(  Q\right)   &  =\operatorname*{add}\nolimits_{-k}\left(
\underbrace{\left(  \varphi\mid_{Q}\right)  \left(  Q\right)  }_{=\varphi
\left(  Q\right)  =\left\{  k+1,k+2,\ldots,\left\vert \varphi\left(  E\right)
\right\vert \right\}  }\right)  =\operatorname*{add}\nolimits_{-k}\left(
\left\{  k+1,k+2,\ldots,\left\vert \varphi\left(  E\right)  \right\vert
\right\}  \right) \\
&  =\left\{  1,2,\ldots,\left\vert \varphi\left(  E\right)  \right\vert
-k\right\}
\end{align*}
(by the definition of $\operatorname*{add}\nolimits_{-k}$). Since $\left(
\varphi\mid_{P}\right)  \left(  P\right)  =\varphi\left(  P\right)  =\left\{
1,2,\ldots,k\right\}  $ is an initial interval, we deduce that the
$\mathbf{E}\mid_{P}$-partition $\varphi\mid_{P}$ is packed. Thus,
$\sigma=\varphi\mid_{P}$ is a packed $\mathbf{E}\mid_{P}$-partition.

On the other hand, (\ref{pf.lem.Gammaw.coprod.bij.a.1}) (applied to $T=Q$)
shows that $\varphi\mid_{Q}$ is an $\mathbf{E}\mid_{Q}$-partition. Hence, the
map $\operatorname*{add}\nolimits_{-k}\circ\left(  \varphi\mid_{Q}\right)  $
is an $\mathbf{E}\mid_{Q}$-partition (since the map $\operatorname*{add}%
\nolimits_{-k}$ is strictly increasing, and since $\left(  \operatorname*{add}%
\nolimits_{-k}\circ\left(  \varphi\mid_{Q}\right)  \right)  \left(  Q\right)
=\left\{  1,2,\ldots,\left\vert \varphi\left(  E\right)  \right\vert
-k\right\}  \subseteq\left\{  1,2,3,\ldots\right\}  $). This $\mathbf{E}%
\mid_{Q}$-partition $\operatorname*{add}\nolimits_{-k}\circ\left(  \varphi
\mid_{Q}\right)  $ is packed (since $\left(  \operatorname*{add}%
\nolimits_{-k}\circ\left(  \varphi\mid_{Q}\right)  \right)  \left(  Q\right)
=\left\{  1,2,\ldots,\left\vert \varphi\left(  E\right)  \right\vert
-k\right\}  $ is an initial interval). Thus, $\tau=\operatorname*{add}%
\nolimits_{-k}\circ\left(  \varphi\mid_{Q}\right)  $ is a packed
$\mathbf{E}\mid_{Q}$-partition.

We now know that $\left(  P,Q\right)  \in\operatorname*{Adm}\mathbf{E}$, that
$\sigma$ is a packed $\mathbf{E}\mid_{P}$-partition, and that $\tau$ is a
packed $\mathbf{E}\mid_{Q}$-partition. In other words, we know that $\left(
\left(  P,Q\right)  ,\sigma,\tau\right)  \in\mathcal{T}$. This proves Lemma
\ref{lem.Gammaw.coprod.bij.a}.
\end{proof}

\begin{proof}
[Proof of Lemma \ref{lem.Gammaw.coprod.bij.b}.]We have $\left(  \left(
P,Q\right)  ,\sigma,\tau\right)  \in\mathcal{T}$. According to the definition
of $\mathcal{T}$, this means that $\left(  P,Q\right)  \in\operatorname{Adm}%
{\mathbf{E}}$, that $\sigma$ is a packed ${\mathbf{E}}\mid_{P}$-partition
$\sigma$, and that $\tau$ is a packed ${\mathbf{E}}\mid_{Q}$-partition.

From $\left(  P,Q\right)  \in\operatorname*{Adm}\mathbf{E}$, we conclude that
$P$ and $Q$ are subsets of $E$ satisfying $P\cap Q=\varnothing$ and $P\cup
Q=E$ and having the property that
\begin{equation}
\text{no }p\in P\text{ and }q\in Q\text{ satisfy }q<_{1}p.
\label{pf.lem.Gammaw.coprod.bij.b.Adm}%
\end{equation}
Since the map $\sigma$ is packed, we have $\sigma\left(  P\right)  =\left\{
1,2,\ldots,\left\vert \sigma\left(  P\right)  \right\vert \right\}  =\left\{
1,2,\ldots,k\right\}  $ (since $\left\vert \sigma\left(  P\right)  \right\vert
=k$).

Since the map $\tau$ is packed, we have $\tau\left(  P\right)  =\left\{
1,2,\ldots,\left\vert \tau\left(  P\right)  \right\vert \right\}  $.

The definition of $\varphi$ shows that%
\begin{equation}
\varphi\left(  e\right)  =\sigma\left(  e\right)
\ \ \ \ \ \ \ \ \ \ \text{for every }e\in P.
\label{pf.lem.Gammaw.coprod.bij.b.1o}%
\end{equation}
for every $e\in P$. Hence, $\varphi\left(  P\right)  =\sigma\left(  P\right)
=\left\{  1,2,\ldots,k\right\}  $.

Also, the definition of $\varphi$ shows that%
\begin{equation}
\varphi\left(  e\right)  =\tau\left(  e\right)
+k\ \ \ \ \ \ \ \ \ \ \text{for every }e\in Q.
\label{pf.lem.Gammaw.coprod.bij.b.2o}%
\end{equation}
Thus,
\begin{align*}
\varphi\left(  Q\right)   &  =\left\{  u+k\ \mid\ u\in\underbrace{\tau\left(
Q\right)  }_{=\left\{  1,2,\ldots,\left\vert \tau\left(  P\right)  \right\vert
\right\}  }\right\}  =\left\{  u+k\ \mid\ u\in\left\{  1,2,\ldots,\left\vert
\tau\left(  P\right)  \right\vert \right\}  \right\} \\
&  =\left\{  k+1,k+2,\ldots,k+\left\vert \tau\left(  P\right)  \right\vert
\right\}  .
\end{align*}
Now, $E=P\cup Q$, so that%
\begin{align}
\varphi\left(  E\right)   &  =\varphi\left(  P\cup Q\right)
=\underbrace{\varphi\left(  P\right)  }_{=\left\{  1,2,\ldots,k\right\}  }%
\cup\underbrace{\varphi\left(  Q\right)  }_{=\left\{  k+1,k+2,\ldots
,k+\left\vert \tau\left(  P\right)  \right\vert \right\}  }\nonumber\\
&  =\left\{  1,2,\ldots,k\right\}  \cup\left\{  k+1,k+2,\ldots,k+\left\vert
\tau\left(  P\right)  \right\vert \right\}  =\left\{  1,2,\ldots,k+\left\vert
\tau\left(  P\right)  \right\vert \right\}  .
\label{pf.lem.Gammaw.coprod.bij.b.3o}%
\end{align}
Thus, $\varphi\left(  E\right)  $ is an initial interval; in other words, the
map $\varphi$ is packed. Furthermore, (\ref{pf.lem.Gammaw.coprod.bij.b.3o})
shows that $\left\vert \varphi\left(  E\right)  \right\vert
=k+\underbrace{\left\vert \tau\left(  P\right)  \right\vert }_{\geq0}\geq k$,
so that $k\in\left\{  0,1,\ldots,\left\vert \varphi\left(  E\right)
\right\vert \right\}  $.

We shall now show that $\varphi$ is an $\mathbf{E}$-partition. To do so, we
must prove the following two claims:

\textit{Claim 1:} Every $e\in E$ and $f\in E$ satisfying $e<_{1}f$ satisfy
$\varphi\left(  e\right)  \leq\varphi\left(  f\right)  $.

\textit{Claim 2:} Every $e\in E$ and $f\in E$ satisfying $e<_{1}f$ and
$f<_{2}e$ satisfy $\varphi\left(  e\right)  <\varphi\left(  f\right)  $.

We shall only prove Claim 1 (as the proof of Claim 2 is similar). So let $e\in
E$ and $f\in E$ be such that $e<_{1}f$. We need to show that $\varphi\left(
e\right)  \leq\varphi\left(  f\right)  $. We are in one of the following four cases:

\textit{Case 1:} We have $e\in P$ and $f\in P$.

\textit{Case 2:} We have $e\in P$ and $f\in Q$.

\textit{Case 3:} We have $e\in Q$ and $f\in P$.

\textit{Case 4:} We have $e\in Q$ and $f\in Q$.

In Case 1, our claim $\varphi\left(  e\right)  \leq\varphi\left(  f\right)  $
follows from the assumption that $\sigma$ is an ${\mathbf{E}}\mid_{P}%
$-partition (because in Case 1, we have $\varphi\left(  e\right)
=\sigma\left(  e\right)  $ and $\varphi\left(  f\right)  =\sigma\left(
f\right)  $). In Case 4, it follows from the assumption that $\tau$ is an
${\mathbf{E}}\mid_{Q}$-partition (since in Case 4, we have $\varphi\left(
e\right)  =\tau\left(  e\right)  +k$ and $\varphi\left(  f\right)
=\tau\left(  f\right)  +k$). In Case 2, it clearly holds (indeed, if $e\in P$,
then the definition of $\varphi$ yields $\varphi\left(  e\right)
=\sigma\left(  e\right)  \leq k$, and if $f\in Q$, then the definition of
$\varphi$ yields $\varphi\left(  f\right)  =\tau\left(  f\right)  +k>k$;
therefore, in Case 2, we have $\varphi\left(  e\right)  \leq k<\varphi\left(
f\right)  $). Finally, Case 3 is impossible (because having $e\in Q$ and $f\in
P$ and $e<_{1}f$ would contradict (\ref{pf.lem.Gammaw.coprod.bij.b.Adm})).
Thus, we have proven the claim in each of the four cases, and consequently
Claim 1 is proven. As we have said above, Claim 2 is proven similarly. Thus,
we have proven that $\varphi$ is an $\mathbf{E}$-partition.

Altogether, we now know that $\varphi$ is a packed ${\mathbf{E}}$-partition,
and that $k\in\left\{  0,1,\ldots,\left\vert \varphi\left(  E\right)
\right\vert \right\}  $. In other words, $\left(  \varphi,k\right)
\in\mathcal{S}$. This proves Lemma \ref{lem.Gammaw.coprod.bij.b}.
\end{proof}

\begin{proof}
[Proof of Lemma \ref{lem.Gammaw.coprod.bij}.]We need to prove the following
two claims:

\textit{Claim 1:} We have $\Phi\circ\Psi=\operatorname*{id}$.

\textit{Claim 2:} We have $\Psi\circ\Phi=\operatorname*{id}$.

\textit{Proof of Claim 1:} Fix $\left(  \left(  P,Q\right)  ,\sigma
,\tau\right)  \in\mathcal{T}$. Set $k=\left\vert \sigma\left(  P\right)
\right\vert $, and let $\varphi$ be the map $E\rightarrow\left\{
1,2,3,\ldots\right\}  $ which sends every $e\in E$ to $%
\begin{cases}
\sigma\left(  e\right)  , & \text{if }e\in P;\\
\tau\left(  e\right)  +k, & \text{if }e\in Q
\end{cases}
$. The definition of $\Psi$ thus yields $\Psi\left(  \left(  P,Q\right)
,\sigma,\tau\right)  =\left(  \varphi,k\right)  $. We shall now show that
$\Phi\left(  \varphi,k\right)  =\left(  \left(  P,Q\right)  ,\sigma
,\tau\right)  $.

Lemma \ref{lem.Gammaw.coprod.bij.b} shows that $\left(  \varphi,k\right)
\in\mathcal{S}$. In other words, $\varphi$ is a packed ${\mathbf{E}}%
$-partition, and we have $k\in\left\{  0,1,\ldots,\left\vert \varphi\left(
E\right)  \right\vert \right\}  $. Since $\varphi$ is packed, we have
$\varphi\left(  E\right)  =\left\{  1,2,\ldots,\left\vert \varphi\left(
E\right)  \right\vert \right\}  $.

The map $\operatorname*{add}\nolimits_{k}:\mathbb{Z}\rightarrow\mathbb{Z}$ is
a bijection, and its inverse is $\left(  \operatorname*{add}\nolimits_{k}%
\right)  ^{-1}=\operatorname*{add}\nolimits_{-k}$.

Since the map $\sigma$ is packed, we have $\sigma\left(  P\right)  =\left\{
1,2,\ldots,\left\vert \sigma\left(  P\right)  \right\vert \right\}  =\left\{
1,2,\ldots,k\right\}  $ (since $\left\vert \sigma\left(  P\right)  \right\vert
=k$).

From $\left(  P,Q\right)  \in\operatorname*{Adm}\mathbf{E}$, we conclude that
$P$ and $Q$ are subsets of $E$ satisfying $P\cap Q=\varnothing$ and $P\cup
Q=E$. Hence, $Q=E\setminus P$ and $P=E\setminus Q$.

The definition of $\varphi$ shows that%
\begin{equation}
\varphi\left(  e\right)  =\sigma\left(  e\right)
\ \ \ \ \ \ \ \ \ \ \text{for every }e\in P.
\label{pf.lem.Gammaw.coprod.bij.c.1}%
\end{equation}
Hence, $\varphi\mid_{P}=\sigma$. Also, the definition of $\varphi$ shows that%
\begin{equation}
\varphi\left(  e\right)  =\tau\left(  e\right)
+k\ \ \ \ \ \ \ \ \ \ \text{for every }e\in Q.
\label{pf.lem.Gammaw.coprod.bij.c.2}%
\end{equation}
Thus, every $e\in Q$ satisfies%
\begin{align}
\varphi\left(  e\right)   &  =\tau\left(  e\right)  +k=\operatorname*{add}%
\nolimits_{k}\left(  \tau\left(  e\right)  \right)
\ \ \ \ \ \ \ \ \ \ \left(  \text{since }\operatorname*{add}\nolimits_{k}%
\left(  \tau\left(  e\right)  \right)  \text{ is defined to be }\tau\left(
e\right)  +k\right) \nonumber\\
&  =\left(  \operatorname*{add}\nolimits_{k}\circ\tau\right)  \left(
e\right)  . \label{pf.lem.Gammaw.coprod.bij.c.2b}%
\end{align}
Hence, $\varphi\mid_{E}=\operatorname*{add}\nolimits_{k}\circ\tau$, so that
$\tau=\underbrace{\left(  \operatorname*{add}\nolimits_{k}\right)  ^{-1}%
}_{=\operatorname*{add}\nolimits_{-k}}\circ\left(  \varphi\mid_{E}\right)
=\operatorname*{add}\nolimits_{-k}\circ\left(  \varphi\mid_{E}\right)  $.

Furthermore, $P=\varphi^{-1}\left(  \left\{  1,2,\ldots,k\right\}  \right)
$\ \ \ \ \footnote{\textit{Proof.} Let $e\in\varphi^{-1}\left(  \left\{
1,2,\ldots,k\right\}  \right)  $. Thus, $e\in E$ and $\varphi\left(  e\right)
\in\left\{  1,2,\ldots,k\right\}  $. If we had $e\in Q$, then we would have
\begin{align*}
\varphi\left(  e\right)   &  =\underbrace{\tau\left(  e\right)  }%
_{>0}+k\ \ \ \ \ \ \ \ \ \ \left(  \text{by
(\ref{pf.lem.Gammaw.coprod.bij.c.2})}\right) \\
&  >k,
\end{align*}
which would contradict $\varphi\left(  e\right)  \in\left\{  1,2,\ldots
,k\right\}  $. Hence, we cannot have $e\in Q$. Thus, $e\in E\setminus Q=P$.
\par
Now, let us forget that we fixed $e$. Thus we have proven that $e\in P$ for
every $e\in\varphi^{-1}\left(  \left\{  1,2,\ldots,k\right\}  \right)  $. In
other words, $\varphi^{-1}\left(  \left\{  1,2,\ldots,k\right\}  \right)
\subseteq P$.
\par
On the other hand, fix $p\in P$. Then, $\varphi\left(  p\right)
=\sigma\left(  p\right)  $ (by (\ref{pf.lem.Gammaw.coprod.bij.c.1})). Hence,
$\varphi\left(  p\right)  =\sigma\left(  p\right)  \in\sigma\left(  P\right)
=\left\{  1,2,\ldots,k\right\}  $, so that $p\in\varphi^{-1}\left(  \left\{
1,2,\ldots,k\right\}  \right)  $.
\par
Now, let us forget that we fixed $p$. Thus we have proven that $p\in
\varphi^{-1}\left(  \left\{  1,2,\ldots,k\right\}  \right)  $ for every $p\in
P$. In other words, $P\subseteq\varphi^{-1}\left(  \left\{  1,2,\ldots
,k\right\}  \right)  $. Combining this with $\varphi^{-1}\left(  \left\{
1,2,\ldots,k\right\}  \right)  \subseteq P$, we obtain $P=\varphi^{-1}\left(
\left\{  1,2,\ldots,k\right\}  \right)  $, qed.} and $Q=\varphi^{-1}\left(
\left\{  k+1,k+2,\ldots,\left\vert \varphi\left(  E\right)  \right\vert
\right\}  \right)  $\ \ \ \ \footnote{\textit{Proof.} Let $e\in\varphi
^{-1}\left(  \left\{  k+1,k+2,\ldots,\left\vert \varphi\left(  E\right)
\right\vert \right\}  \right)  $. Thus, $e\in E$ and $\varphi\left(  e\right)
\in\left\{  k+1,k+2,\ldots,\left\vert \varphi\left(  E\right)  \right\vert
\right\}  $. If we had $e\in P$, then we would have
\begin{align*}
\varphi\left(  e\right)   &  =\sigma\left(  e\right)
\ \ \ \ \ \ \ \ \ \ \left(  \text{by (\ref{pf.lem.Gammaw.coprod.bij.c.1}%
)}\right) \\
&  \in\sigma\left(  P\right)  =\left\{  1,2,\ldots,k\right\}
\end{align*}
and therefore $\varphi\left(  e\right)  \leq k$, which would contradict
$\varphi\left(  e\right)  \in\left\{  k+1,k+2,\ldots,\left\vert \varphi\left(
E\right)  \right\vert \right\}  $. Hence, we cannot have $e\in P$. Thus, $e\in
E\setminus P=Q$.
\par
Now, let us forget that we fixed $e$. Thus we have proven that $e\in Q$ for
every $e\in\varphi^{-1}\left(  \left\{  k+1,k+2,\ldots,\left\vert
\varphi\left(  E\right)  \right\vert \right\}  \right)  $. In other words,
$\varphi^{-1}\left(  \left\{  k+1,k+2,\ldots,\left\vert \varphi\left(
E\right)  \right\vert \right\}  \right)  \subseteq Q$.
\par
On the other hand, fix $q\in Q$. Then, $\varphi\left(  q\right)  =\tau\left(
q\right)  +k$ (by (\ref{pf.lem.Gammaw.coprod.bij.c.2})). Hence, $\varphi
\left(  q\right)  =\underbrace{\tau\left(  q\right)  }_{>0}+k>k$. Combining
this with $\varphi\left(  q\right)  \in\varphi\left(  E\right)  =\left\{
1,2,\ldots,\left\vert \varphi\left(  E\right)  \right\vert \right\}  $, we
obtain $\varphi\left(  q\right)  \in\left\{  k+1,k+2,\ldots,\left\vert
\varphi\left(  E\right)  \right\vert \right\}  $. Hence, $q\in\varphi
^{-1}\left(  \left\{  k+1,k+2,\ldots,\left\vert \varphi\left(  E\right)
\right\vert \right\}  \right)  $.
\par
Now, let us forget that we fixed $q$. Thus we have proven that $q\in
\varphi^{-1}\left(  \left\{  k+1,k+2,\ldots,\left\vert \varphi\left(
E\right)  \right\vert \right\}  \right)  $ for every $q\in Q$. In other words,
$Q\subseteq\varphi^{-1}\left(  \left\{  k+1,k+2,\ldots,\left\vert
\varphi\left(  E\right)  \right\vert \right\}  \right)  $. Combining this with
$\varphi^{-1}\left(  \left\{  k+1,k+2,\ldots,\left\vert \varphi\left(
E\right)  \right\vert \right\}  \right)  \subseteq Q$, we obtain
$Q=\varphi^{-1}\left(  \left\{  k+1,k+2,\ldots,\left\vert \varphi\left(
E\right)  \right\vert \right\}  \right)  $, qed.}. Altogether, we thus know
that
\begin{align*}
P  &  =\varphi^{-1}\left(  \left\{  1,2,\ldots,k\right\}  \right)
,\ \ \ \ \ \ \ \ \ \ Q=\varphi^{-1}\left(  \left\{  k+1,k+2,\ldots,\left\vert
\varphi\left(  E\right)  \right\vert \right\}  \right)  ,\\
\sigma &  =\varphi\mid_{P}\ \ \ \ \ \ \ \ \ \ \text{and}%
\ \ \ \ \ \ \ \ \ \ \tau=\operatorname*{add}\nolimits_{-k}\circ\left(
\varphi\mid_{Q}\right)  .
\end{align*}
These equations are identical with the equations
(\ref{eq.lem.Gammaw.coprod.bij.def1}) and (\ref{eq.lem.Gammaw.coprod.bij.def2}%
) that were used in the definition of $\Phi\left(  \varphi,k\right)  $. Hence,
the definition of $\Phi$ shows that $\Phi\left(  \varphi,k\right)  =\left(
\left(  P,Q\right)  ,\sigma,\tau\right)  $. Thus, $\left(  \left(  P,Q\right)
,\sigma,\tau\right)  =\Phi\underbrace{\left(  \varphi,k\right)  }%
_{=\Psi\left(  \left(  P,Q\right)  ,\sigma,\tau\right)  }=\Phi\left(
\Psi\left(  \left(  P,Q\right)  ,\sigma,\tau\right)  \right)  $.

Now, let us forget that we fixed $\left(  \left(  P,Q\right)  ,\sigma
,\tau\right)  $. We thus have shown that $\Phi\left(  \Psi\left(  \left(
P,Q\right)  ,\sigma,\tau\right)  \right)  =\left(  \left(  P,Q\right)
,\sigma,\tau\right)  $ for every $\left(  \left(  P,Q\right)  ,\sigma
,\tau\right)  \in\mathcal{T}$. In other words, $\Phi\circ\Psi
=\operatorname*{id}$. This proves Claim 1.

\textit{Proof of Claim 2:} Fix $\left(  \varphi,k\right)  \in\mathcal{S}$.
Define $P$, $Q$, $\sigma$ and $\tau$ by (\ref{eq.lem.Gammaw.coprod.bij.def1})
and (\ref{eq.lem.Gammaw.coprod.bij.def2}). The definition of $\Phi$ shows that
$\Phi\left(  \varphi,k\right)  =\left(  P,Q,\sigma,\tau\right)  $. From Lemma
\ref{lem.Gammaw.coprod.bij.a}, we know that $\left(  \left(  P,Q\right)
,\sigma,\tau\right)  \in\mathcal{T}$. In other words, we know that $\left(
P,Q\right)  \in\operatorname{Adm}{\mathbf{E}}$, that $\sigma$ is a packed
${\mathbf{E}}\mid_{P}$-partition, and that $\tau$ is a packed ${\mathbf{E}%
}\mid_{Q}$-partition.

From $\left(  P,Q\right)  \in\operatorname*{Adm}\mathbf{E}$, we conclude that
$P$ and $Q$ are subsets of $E$ satisfying $P\cap Q=\varnothing$ and $P\cup
Q=E$.

We have $\varphi\left(  P\right)  =\left\{  1,2,\ldots,k\right\}  $. (This was
proven in our proof of Lemma \ref{lem.Gammaw.coprod.bij.a} above; see the
equality (\ref{pf.lem.Gammaw.coprod.bij.a.3}).)

We have $\sigma=\varphi\mid_{P}$. Thus, for every $e\in P$, we have
$\sigma\left(  e\right)  =\left(  \varphi\mid_{P}\right)  \left(  e\right)
=\varphi\left(  e\right)  $. In other words, for every $e\in P$, we have
\begin{equation}
\varphi\left(  e\right)  =\sigma\left(  e\right)  .
\label{pf.lem.Gammaw.coprod.bij.c.11}%
\end{equation}
Also, $\tau=\operatorname*{add}\nolimits_{-k}\circ\left(  \varphi\mid
_{Q}\right)  $. Hence, for every $e\in Q$, we have%
\begin{align*}
\tau\left(  e\right)   &  =\left(  \operatorname*{add}\nolimits_{-k}%
\circ\left(  \varphi\mid_{Q}\right)  \right)  \left(  e\right)
=\operatorname*{add}\nolimits_{-k}\left(  \underbrace{\left(  \varphi\mid
_{Q}\right)  \left(  e\right)  }_{=\varphi\left(  e\right)  }\right) \\
&  =\operatorname*{add}\nolimits_{-k}\left(  \varphi\left(  e\right)  \right)
=\varphi\left(  e\right)  +\left(  -k\right)  \ \ \ \ \ \ \ \ \ \ \left(
\text{by the definition of }\operatorname*{add}\nolimits_{-k}\right) \\
&  =\varphi\left(  e\right)  -k.
\end{align*}
Thus, for every $e\in Q$, we have%
\[
\varphi\left(  e\right)  =\tau\left(  e\right)  +k.
\]
Combining this with (\ref{pf.lem.Gammaw.coprod.bij.c.11}), we conclude that%
\begin{equation}
\varphi\left(  e\right)  =%
\begin{cases}
\sigma\left(  e\right)  , & \text{if }e\in P;\\
\tau\left(  e\right)  +k, & \text{if }e\in Q
\end{cases}
\ \ \ \ \ \ \ \ \ \ \text{for every }e\in E
\label{pf.lem.Gammaw.coprod.bij.c.12}%
\end{equation}
\footnote{The right hand side of this equality makes sense because $P\cap
Q=\varnothing$ and $P\cup Q=E$.}. Moreover, $\sigma$ is a packed map; thus,
$\sigma\left(  P\right)  =\left\{  1,2,\ldots,\left\vert \sigma\left(
P\right)  \right\vert \right\}  $. Hence,%
\[
\left\{  1,2,\ldots,\left\vert \sigma\left(  P\right)  \right\vert \right\}
=\underbrace{\sigma}_{=\varphi\mid_{P}}\left(  P\right)  =\left(  \varphi
\mid_{P}\right)  \left(  P\right)  =\varphi\left(  P\right)  =\left\{
1,2,\ldots,k\right\}  .
\]
Thus, $\left\vert \sigma\left(  P\right)  \right\vert =k$.

So we know that $k=\left\vert \sigma\left(  P\right)  \right\vert $, and that
$\varphi$ is the map $E\rightarrow\left\{  1,2,3,\ldots\right\}  $ which sends
every $e\in E$ to $%
\begin{cases}
\sigma\left(  e\right)  , & \text{if }e\in P;\\
\tau\left(  e\right)  +k, & \text{if }e\in Q
\end{cases}
$ (because of (\ref{pf.lem.Gammaw.coprod.bij.c.12})). Thus, our $k$ and our
$\varphi$ are precisely the $k$ and the $\varphi$ in the definition of
$\Psi\left(  \left(  P,Q\right)  ,\sigma,\tau\right)  $. Hence, $\Psi\left(
\left(  P,Q\right)  ,\sigma,\tau\right)  =\left(  \varphi,k\right)  $. Thus,
$\left(  \varphi,k\right)  =\Psi\underbrace{\left(  \left(  P,Q\right)
,\sigma,\tau\right)  }_{=\Phi\left(  \varphi,k\right)  }=\Psi\left(
\Phi\left(  \varphi,k\right)  \right)  $.

Now, let us forget that we fixed $\left(  \varphi,k\right)  $. We thus have
shown that $\Psi\left(  \Phi\left(  \varphi,k\right)  \right)  =\left(
\varphi,k\right)  $ for every $\left(  \varphi,k\right)  \in\mathcal{S}$. In
other words, $\Psi\circ\Phi=\operatorname*{id}$. This proves Claim 2.

Now, both Claims 1 and 2 are proven. Thus, the maps $\Phi$ and $\Psi$ are
mutually inverse. This proves Lemma \ref{lem.Gammaw.coprod.bij}.
\end{proof}

\begin{proof}
[Proof of Proposition~\ref{prop.Gammaw.coprod}.]Define $\mathcal{S}$,
$\mathcal{T}$, $\Phi$ and $\Psi$ as in Lemma \ref{lem.Gammaw.coprod.bij}. From
Lemma \ref{lem.Gammaw.coprod.bij}, we know that the maps $\Phi$ and $\Psi$ are
mutually inverse. Hence, $\Phi$ is a bijection from $\mathcal{S}$ to
$\mathcal{T}$.

Whenever $\alpha=\left(  \alpha_{1},\alpha_{2},\ldots,\alpha_{\ell}\right)  $
is a composition and $k\in\left\{  0,1,\ldots,\ell\right\}  $, we introduce
the notation $\alpha\left[  :k\right]  $ for the composition $\left(
\alpha_{1},\alpha_{2},\ldots,\alpha_{k}\right)  $, and the notation
$\alpha\left[  k:\right]  $ for the composition $\left(  \alpha_{k+1}%
,\alpha_{k+2},\ldots,\alpha_{\ell}\right)  $. Now, the formula
\eqref{eq.coproduct.M} can be rewritten as follows:
\begin{align}
\Delta\left(  M_{\alpha}\right)   &  =\sum_{k=0}^{\ell}M_{\alpha\left[
:k\right]  }\otimes M_{\alpha\left[  k:\right]  }%
\label{pf.prop.Gammaw.coprod.long.DeltaM}\\
&  \qquad\text{ for every }\ell\in{\mathbb{N}}\text{ and every composition
}\alpha\text{ with }\ell\text{ entries.}\nonumber
\end{align}


Let us observe a simple fact: For any $\left(  \varphi,k\right)
\in\mathcal{S}$, we have%
\begin{equation}
\left(  \operatorname{ev}_{w}\varphi\right)  \left[  :k\right]
=\operatorname{ev}_{w\mid_{P}}\sigma\ \ \ \ \ \ \ \ \ \ \text{and}%
\ \ \ \ \ \ \ \ \ \ \left(  \operatorname{ev}_{w}\varphi\right)  \left[
k:\right]  =\operatorname{ev}_{w\mid_{Q}}\tau,
\label{pf.prop.Gammaw.coprod.evs}%
\end{equation}
where $\left(  \left(  P,Q\right)  ,\sigma,\tau\right)  =\Phi\left(
\varphi,k\right)  $\ \ \ \ \footnote{\textit{Proof of
(\ref{pf.prop.Gammaw.coprod.evs}):} Let $\left(  \varphi,k\right)
\in\mathcal{S}$. Let $\left(  \left(  P,Q\right)  ,\sigma,\tau\right)
=\Phi\left(  \varphi,k\right)  $. We must prove the equalities
(\ref{pf.prop.Gammaw.coprod.evs}).
\par
For every $\ell\in\mathbb{Z}$, define the map $\operatorname*{add}%
\nolimits_{\ell}:\mathbb{Z}\rightarrow\mathbb{Z}$ as in Lemma
\ref{lem.Gammaw.coprod.bij.a}.
\par
Let $\ell=\left\vert \varphi\left(  E\right)  \right\vert $; thus,
$\varphi\left(  E\right)  =\left\{  1,2,\ldots,\ell\right\}  $ (since
$\varphi$ is packed). For each $i\in\left\{  1,2,\ldots,\ell\right\}  $,
define $\alpha_{i}\in\mathbb{N}$ by
\begin{equation}
\alpha_{i}=\sum_{e\in\varphi^{-1}\left(  i\right)  }w\left(  e\right)  .
\label{pf.prop.Gammaw.coprod.evs.pf.alphai}%
\end{equation}
Then, $\operatorname*{ev}\nolimits_{w}\varphi=\left(  \alpha_{1},\alpha
_{2},\ldots,\alpha_{\ell}\right)  $ (by the definition of $\operatorname*{ev}%
\nolimits_{w}\varphi$).
\par
We have $\left(  \varphi,k\right)  \in\mathcal{S}$. Thus, $\varphi$ is a
packed $\mathbf{E}$-partition, and $k$ is an element of $\left\{
0,1,\ldots,\left\vert \varphi\left(  E\right)  \right\vert \right\}  $ (by the
definition of $\mathcal{S}$). Thus, $k\in\left\{  0,1,\ldots,\left\vert
\varphi\left(  E\right)  \right\vert \right\}  =\left\{  0,1,\ldots
,\ell\right\}  $ (since $\left\vert \varphi\left(  E\right)  \right\vert
=\ell$).
\par
Now, from $\operatorname*{ev}\nolimits_{w}\varphi=\left(  \alpha_{1}%
,\alpha_{2},\ldots,\alpha_{\ell}\right)  $, we obtain%
\[
\left(  \operatorname{ev}_{w}\varphi\right)  \left[  :k\right]  =\left(
\alpha_{1},\alpha_{2},\ldots,\alpha_{k}\right)  \ \ \ \ \ \ \ \ \ \ \text{and}%
\ \ \ \ \ \ \ \ \ \ \left(  \operatorname{ev}_{w}\varphi\right)  \left[
k:\right]  =\left(  \alpha_{k+1},\alpha_{k+2},\ldots,\alpha_{\ell}\right)  .
\]
\par
Recall that $\left(  \left(  P,Q\right)  ,\sigma,\tau\right)  =\Phi\left(
\varphi,k\right)  $. Hence, $P$, $Q$, $\sigma$ and $\tau$ are defined by
(\ref{eq.lem.Gammaw.coprod.bij.def1}) and (\ref{eq.lem.Gammaw.coprod.bij.def2}%
) (according to the definition of $\Phi$). We know (from Lemma
\ref{lem.Gammaw.coprod.bij.a}) that $\left(  \left(  P,Q\right)  ,\sigma
,\tau\right)  \in\mathcal{T}$. In other words, we know that $\left(
P,Q\right)  \in\operatorname{Adm}{\mathbf{E}}$, that $\sigma$ is a packed
${\mathbf{E}}\mid_{P}$-partition, and that $\tau$ is a packed ${\mathbf{E}%
}\mid_{Q}$-partition.
\par
For every $e\in P$, we have%
\begin{equation}
\underbrace{\sigma}_{\substack{=\varphi\mid_{P}\\\text{(by
(\ref{eq.lem.Gammaw.coprod.bij.def2}))}}}\left(  e\right)  =\left(
\varphi\mid_{P}\right)  \left(  e\right)  =\varphi\left(  e\right)  .
\label{pf.prop.Gammaw.coprod.evs.pf.1}%
\end{equation}
\par
For every $e\in Q$, we have%
\begin{align}
\underbrace{\tau}_{\substack{=\operatorname*{add}\nolimits_{-k}\circ\left(
\varphi\mid_{Q}\right)  \\\text{(by (\ref{eq.lem.Gammaw.coprod.bij.def2}))}%
}}\left(  e\right)   &  =\left(  \operatorname*{add}\nolimits_{-k}\circ\left(
\varphi\mid_{Q}\right)  \right)  \left(  e\right)  =\operatorname*{add}%
\nolimits_{-k}\left(  \left(  \varphi\mid_{Q}\right)  \left(  e\right)
\right) \nonumber\\
&  =\underbrace{\left(  \varphi\mid_{Q}\right)  \left(  e\right)  }%
_{=\varphi\left(  e\right)  }+\left(  -k\right)  \ \ \ \ \ \ \ \ \ \ \left(
\text{by the definition of }\operatorname*{add}\nolimits_{-k}\right)
\nonumber\\
&  =\varphi\left(  e\right)  -k. \label{pf.prop.Gammaw.coprod.evs.pf.2}%
\end{align}
\par
For every $i\in\left\{  1,2,\ldots,k\right\}  $, we have%
\begin{align}
\sigma^{-1}\left(  i\right)   &  =\left\{  e\in P\ \mid\ \underbrace{\sigma
\left(  e\right)  }_{\substack{=\varphi\left(  e\right)  \\\text{(by
(\ref{pf.prop.Gammaw.coprod.evs.pf.1}))}}}=i\right\}  =\left\{  e\in
P\ \mid\ \varphi\left(  e\right)  =i\right\} \nonumber\\
&  =\underbrace{\left\{  e\in E\ \mid\ \varphi\left(  e\right)  =i\right\}
}_{=\varphi^{-1}\left(  i\right)  }\cap\underbrace{P}_{=\varphi^{-1}\left(
\left\{  1,2,\ldots,k\right\}  \right)  }\nonumber\\
&  =\varphi^{-1}\left(  i\right)  \cap\varphi^{-1}\left(  \left\{
1,2,\ldots,k\right\}  \right)  =\varphi^{-1}\left(  i\right)
\label{pf.prop.Gammaw.coprod.evs.pf.3}%
\end{align}
(since $\varphi^{-1}\left(  i\right)  \subseteq\varphi^{-1}\left(  \left\{
1,2,\ldots,k\right\}  \right)  $ (because $i\in\left\{  1,2,\ldots,k\right\}
$).
\par
For every $i\in\left\{  1,2,\ldots,\ell-k\right\}  $, we have%
\begin{align}
\tau^{-1}\left(  i\right)   &  =\left\{  e\in Q\ \mid\ \underbrace{\tau\left(
e\right)  }_{\substack{=\varphi\left(  e\right)  -k\\\text{(by
(\ref{pf.prop.Gammaw.coprod.evs.pf.2}))}}}=i\right\}  =\left\{  e\in
Q\ \mid\ \underbrace{\varphi\left(  e\right)  -k=i}_{\Longleftrightarrow
\ \left(  \varphi\left(  e\right)  =k+i\right)  }\right\} \nonumber\\
&  =\left\{  e\in Q\ \mid\ \varphi\left(  e\right)  =k+i\right\}
=\underbrace{\left\{  e\in E\ \mid\ \varphi\left(  e\right)  =k+i\right\}
}_{=\varphi^{-1}\left(  k+i\right)  }\cap\underbrace{Q}_{\substack{=\varphi
^{-1}\left(  \left\{  k+1,k+2,\ldots,\left\vert \varphi\left(  E\right)
\right\vert \right\}  \right)  \\=\varphi^{-1}\left(  \left\{  k+1,k+2,\ldots
,\ell\right\}  \right)  \\\text{(since }\left\vert \varphi\left(  E\right)
\right\vert =\ell\text{)}}}\nonumber\\
&  =\varphi^{-1}\left(  k+i\right)  \cap\varphi^{-1}\left(  \left\{
k+1,k+2,\ldots,\ell\right\}  \right)  =\varphi^{-1}\left(  k+i\right)
\label{pf.prop.Gammaw.coprod.evs.pf.4}%
\end{align}
(since $\varphi^{-1}\left(  k+i\right)  \subseteq\varphi^{-1}\left(  \left\{
k+1,k+2,\ldots,\ell\right\}  \right)  $ (since $k+i\in\left\{  k+1,k+2,\ldots
,\ell\right\}  $ (since $i\in\left\{  1,2,\ldots,\ell-k\right\}  $))).
\par
We have $\varphi\left(  Q\right)  =\left\{  1,2,\ldots,k\right\}  $. (This was
proven in our proof of Lemma \ref{lem.Gammaw.coprod.bij.a} above; see the
equality (\ref{pf.lem.Gammaw.coprod.bij.a.3}).) But $\sigma=\varphi\mid_{P}$,
so that $\sigma\left(  P\right)  =\left(  \varphi\mid_{P}\right)  \left(
P\right)  =\varphi\left(  P\right)  =\left\{  1,2,\ldots,k\right\}  $. Hence,
$\left\vert \sigma\left(  P\right)  \right\vert =\left\vert \left\{
1,2,\ldots,k\right\}  \right\vert =k$. Therefore, the definition of
$\operatorname{ev}_{w\mid_{P}}\sigma$ shows that $\operatorname{ev}_{w\mid
_{P}}\sigma=\left(  \beta_{1},\beta_{2},\ldots,\beta_{k}\right)  $, where each
$\beta_{i}$ is defined as $\sum_{e\in\sigma^{-1}\left(  i\right)  }\left(
w\mid_{P}\right)  \left(  e\right)  $. Thus, every $i\in\left\{
1,2,\ldots,k\right\}  $ satisfies%
\[
\beta_{i}=\underbrace{\sum_{e\in\sigma^{-1}\left(  i\right)  }}%
_{\substack{=\sum_{e\in\varphi^{-1}\left(  i\right)  }\\\text{(by
(\ref{pf.prop.Gammaw.coprod.evs.pf.3}))}}}\underbrace{\left(  w\mid
_{P}\right)  \left(  e\right)  }_{=w\left(  e\right)  }=\sum_{e\in\varphi
^{-1}\left(  i\right)  }w\left(  e\right)  =\alpha_{i}%
\ \ \ \ \ \ \ \ \ \ \left(  \text{by
(\ref{pf.prop.Gammaw.coprod.evs.pf.alphai})}\right)  .
\]
Hence, $\left(  \beta_{1},\beta_{2},\ldots,\beta_{k}\right)  =\left(
\alpha_{1},\alpha_{2},\ldots,\alpha_{k}\right)  =\left(  \operatorname{ev}%
_{w}\varphi\right)  \left[  :k\right]  $, so that $\left(  \operatorname{ev}%
_{w}\varphi\right)  \left[  :k\right]  =\left(  \beta_{1},\beta_{2}%
,\ldots,\beta_{k}\right)  =\operatorname{ev}_{w\mid_{P}}\sigma$.
\par
We have $\varphi\left(  Q\right)  =\left\{  k+1,k+2,\ldots,\left\vert
\varphi\left(  E\right)  \right\vert \right\}  $. (This was proven in our
proof of Lemma \ref{lem.Gammaw.coprod.bij.a} above; see the equality
(\ref{pf.lem.Gammaw.coprod.bij.a.4}).) But
\begin{align*}
\tau\left(  Q\right)   &  =\left\{  \underbrace{\tau\left(  e\right)
}_{\substack{=\varphi\left(  e\right)  -k\\\text{(by
(\ref{pf.prop.Gammaw.coprod.evs.pf.2}))}}}\ \mid\ e\in Q\right\}  =\left\{
\varphi\left(  e\right)  -k\ \mid\ e\in Q\right\} \\
&  =\left\{  u-k\ \mid\ u\in\varphi\left(  Q\right)  \right\}  .
\end{align*}
Thus,
\begin{align*}
\left\vert \tau\left(  Q\right)  \right\vert  &  =\left\vert \varphi\left(
Q\right)  \right\vert =\left\vert \left\{  k+1,k+2,\ldots,\left\vert
\varphi\left(  E\right)  \right\vert \right\}  \right\vert
\ \ \ \ \ \ \ \ \ \ \left(  \text{since }\varphi\left(  Q\right)  =\left\{
k+1,k+2,\ldots,\left\vert \varphi\left(  E\right)  \right\vert \right\}
\right) \\
&  =\underbrace{\left\vert \varphi\left(  E\right)  \right\vert }_{=\ell
}-k=\ell-k.
\end{align*}
Therefore, the definition of $\operatorname{ev}_{w\mid_{Q}}\tau$ shows that
$\operatorname{ev}_{w\mid_{Q}}\tau=\left(  \gamma_{1},\gamma_{2},\ldots
,\gamma_{\ell-k}\right)  $, where each $\gamma_{i}$ is defined as $\sum
_{e\in\tau^{-1}\left(  i\right)  }\left(  w\mid_{Q}\right)  \left(  e\right)
$. Thus, every $i\in\left\{  1,2,\ldots,\ell-k\right\}  $ satisfies%
\begin{align*}
\gamma_{i}  &  =\underbrace{\sum_{e\in\tau^{-1}\left(  i\right)  }%
}_{\substack{=\sum_{e\in\varphi^{-1}\left(  k+i\right)  }\\\text{(by
(\ref{pf.prop.Gammaw.coprod.evs.pf.4}))}}}\underbrace{\left(  w\mid
_{Q}\right)  \left(  e\right)  }_{=w\left(  e\right)  }=\sum_{e\in\varphi
^{-1}\left(  k+i\right)  }w\left(  e\right)  =\alpha_{k+i}\\
&  \ \ \ \ \ \ \ \ \ \ \left(
\begin{array}
[c]{c}%
\text{since (\ref{pf.prop.Gammaw.coprod.evs.pf.alphai}) (applied to }k+i\text{
instead of }i\text{)}\\
\text{shows that }\alpha_{k+i}=\sum_{e\in\varphi^{-1}\left(  k+i\right)
}w\left(  e\right)
\end{array}
\right)  .
\end{align*}
Hence, $\left(  \gamma_{1},\gamma_{2},\ldots,\gamma_{\ell-k}\right)  =\left(
\alpha_{k+1},\alpha_{k+2},\ldots,\alpha_{\ell}\right)  =\left(
\operatorname{ev}_{w}\varphi\right)  \left[  k:\right]  $, so that $\left(
\operatorname{ev}_{w}\varphi\right)  \left[  k:\right]  =\left(  \gamma
_{1},\gamma_{2},\ldots,\gamma_{\ell-k}\right)  =\operatorname{ev}_{w\mid_{Q}%
}\tau$.
\par
Thus, both parts of (\ref{pf.prop.Gammaw.coprod.evs}) are proven.}.

Now, applying $\Delta$ to the equality \eqref{eq.prop.Gammaw.packed} yields
\begin{align}
\Delta\left(  \Gamma\left(  {\mathbf{E}},w\right)  \right)   &  =\sum
_{\varphi\text{ is a packed }{\mathbf{E}}\text{-partition}}\underbrace{\Delta
\left(  M_{\operatorname{ev}_{w}\varphi}\right)  }_{\substack{=\sum
_{k=0}^{\left\vert \varphi\left(  E\right)  \right\vert }M_{\left(
\operatorname{ev}_{w}\varphi\right)  \left[  :k\right]  }\otimes M_{\left(
\operatorname{ev}_{w}\varphi\right)  \left[  k:\right]  }\\\text{(by
\eqref{pf.prop.Gammaw.coprod.long.DeltaM}, applied to }\alpha
=\operatorname*{ev}\nolimits_{w}\varphi\text{ and }\ell=\left\vert
\varphi\left(  E\right)  \right\vert \text{)}}}\nonumber\\
&  =\underbrace{\sum_{\varphi\text{ is a packed }{\mathbf{E}}\text{-partition}%
}\sum_{k=0}^{\left\vert \varphi\left(  E\right)  \right\vert }}%
_{\substack{=\sum_{\left(  \varphi,k\right)  \in\mathcal{S}}\\\text{(by the
definition of }\mathcal{S}\text{)}}}M_{\left(  \operatorname{ev}_{w}%
\varphi\right)  \left[  :k\right]  }\otimes M_{\left(  \operatorname{ev}%
_{w}\varphi\right)  \left[  k:\right]  }\nonumber\\
&  =\sum_{\left(  \varphi,k\right)  \in\mathcal{S}}M_{\left(
\operatorname{ev}_{w}\varphi\right)  \left[  :k\right]  }\otimes M_{\left(
\operatorname{ev}_{w}\varphi\right)  \left[  k:\right]  }%
.\label{pf.Gammaw.coprod.long.lhs}%
\end{align}


On the other hand, rewriting each of the tensorands on the right hand side of
\eqref{eq.prop.Gammaw.coprod} using \eqref{eq.prop.Gammaw.packed}, we obtain
\begin{align*}
&  \sum_{\left(  P,Q\right)  \in\operatorname{Adm}{\mathbf{E}}}%
\underbrace{\Gamma\left(  {\mathbf{E}}\mid_{P},w\mid_{P}\right)
}_{\substack{=\sum_{\varphi\text{ is a packed }{\mathbf{E}}\mid_{P}%
\text{-partition}}M_{\operatorname{ev}_{w\mid_{P}}\varphi}\\\text{(by
\eqref{eq.prop.Gammaw.packed})}}}\otimes\underbrace{\Gamma\left(  {\mathbf{E}%
}\mid_{Q},w\mid_{Q}\right)  }_{\substack{=\sum_{\varphi\text{ is a packed
}{\mathbf{E}}\mid_{Q}\text{-partition}}M_{\operatorname{ev}_{w\mid_{Q}}%
\varphi}\\\text{(by \eqref{eq.prop.Gammaw.packed})}}}\\
&  =\sum_{\left(  P,Q\right)  \in\operatorname{Adm}{\mathbf{E}}}\left(
\sum_{\varphi\text{ is a packed }{\mathbf{E}}\mid_{P}\text{-partition}%
}M_{\operatorname{ev}_{w\mid_{P}}\varphi}\right)  \otimes\left(  \sum
_{\varphi\text{ is a packed }{\mathbf{E}}\mid_{Q}\text{-partition}%
}M_{\operatorname{ev}_{w\mid_{Q}}\varphi}\right)  \\
&  =\sum_{\left(  P,Q\right)  \in\operatorname{Adm}{\mathbf{E}}}\left(
\sum_{\sigma\text{ is a packed }{\mathbf{E}}\mid_{P}\text{-partition}%
}M_{\operatorname{ev}_{w\mid_{P}}\sigma}\right)  \otimes\left(  \sum
_{\tau\text{ is a packed }{\mathbf{E}}\mid_{Q}\text{-partition}}%
M_{\operatorname{ev}_{w\mid_{Q}}\tau}\right)  \\
&  =\underbrace{\sum_{\left(  P,Q\right)  \in\operatorname{Adm}{\mathbf{E}}%
}\sum_{\sigma\text{ is a packed }{\mathbf{E}}\mid_{P}\text{-partition}}%
\sum_{\tau\text{ is a packed }{\mathbf{E}}\mid_{Q}\text{-partition}}%
}_{\substack{=\sum_{\left(  \left(  P,Q\right)  ,\sigma,\tau\right)
\in\mathcal{T}}\\\text{(by the definition of }\mathcal{T}\text{)}%
}}M_{\operatorname{ev}_{w\mid_{P}}\sigma}\otimes M_{\operatorname{ev}%
_{w\mid_{Q}}\tau}\\
&  =\sum_{\left(  \left(  P,Q\right)  ,\sigma,\tau\right)  \in\mathcal{T}%
}M_{\operatorname{ev}_{w\mid_{P}}\sigma}\otimes M_{\operatorname{ev}%
_{w\mid_{Q}}\tau}\\
&  =\sum_{\left(  \varphi,k\right)  \in\mathcal{S}}M_{\left(
\operatorname{ev}_{w}\varphi\right)  \left[  :k\right]  }\otimes M_{\left(
\operatorname{ev}_{w}\varphi\right)  \left[  k:\right]  }%
\end{align*}
(here, we have substituted $\Phi\left(  \varphi,k\right)  $ for $\left(
\left(  P,Q\right)  ,\sigma,\tau\right)  $ in the sum, using the fact that
$\Phi$ is a bijection from $\mathcal{S}$ to $\mathcal{T}$, and using the
equalities (\ref{pf.prop.Gammaw.coprod.evs}) to rewrite the addend
$M_{\operatorname{ev}_{w\mid_{P}}\sigma}\otimes M_{\operatorname{ev}%
_{w\mid_{Q}}\tau}$ as $M_{\left(  \operatorname{ev}_{w}\varphi\right)  \left[
:k\right]  }\otimes M_{\left(  \operatorname{ev}_{w}\varphi\right)  \left[
k:\right]  }$). Comparing this with (\ref{pf.Gammaw.coprod.long.lhs}), we
obtain%
\[
\Delta\left(  \Gamma\left(  {\mathbf{E}},w\right)  \right)  =\sum_{\left(
P,Q\right)  \in\operatorname{Adm}{\mathbf{E}}}\Gamma\left(  {\mathbf{E}}%
\mid_{P},w\mid_{P}\right)  \otimes\Gamma\left(  {\mathbf{E}}\mid_{Q},w\mid
_{Q}\right)  .
\]
This proves Proposition \ref{prop.Gammaw.coprod}.
\end{proof}
\end{verlong}

We note in passing that there is also a rule for multiplying
quasisymmetric functions of the form $\Gamma\left(\EE, w\right)$.
Namely, if $\EE$ and $\FF$ are two double posets and $u$ and $v$
are corresponding maps, then $\Gamma\left(\EE, u\right)
\Gamma\left(\FF, v\right) = \Gamma\left(\EE \FF, w\right)$ for a
map $w$ which is defined to be $u$ on the subset $\EE$ of
$\EE \FF$, and $v$ on the subset $\FF$ of $\EE \FF$. Here, $\EE \FF$
is a double poset defined as in \cite[\S 2.1]{Mal-Reu-DP}.
Combined with Proposition~\ref{prop.Gammaw.qsym}, this fact gives
a combinatorial proof for the fact that $\QSym$ is a $\kk$-algebra,
as well as for some standard formulas for multiplications of
quasisymmetric functions; similarly,
Proposition~\ref{prop.Gammaw.coprod} can be used to derive the
well-known formulas for $\Delta M_\alpha$, $\Delta L_\alpha$,
$\Delta s_{\lambda / \mu}$ etc. (although, of course, we have
already used the formula for $\Delta M_\alpha$ in our proof of
Proposition~\ref{prop.Gammaw.coprod}).

\section{Proof of Theorem~\ref{thm.antipode.Gammaw}}
\label{sect.proof}

Before we come to the proof of Theorem~\ref{thm.antipode.Gammaw},
let us state three simple lemmas:

\begin{lemma}
\label{lem.admissible.cover}
Let $\EE = \left(E, <_1, <_2\right)$ be a double poset.
Let $P$ and $Q$ be subsets of $E$ such that
$P \cap Q = \varnothing$ and $P \cup Q = E$.
Assume that there exist no $p \in P$ and $q \in Q$ such that
$q$ is $<_1$-covered by $p$. Then, $\left(P, Q\right) \in
\Adm \EE$.
\end{lemma}

\begin{proof}[Proof of Lemma~\ref{lem.admissible.cover}.]
For any $a \in E$ and $b \in E$, we let $\left[a, b\right]$
denote the subset \newline
$\left\{e \in E \mid a <_1 e <_1 b\right\}$ of $E$. It is
clear that if $a$, $b$ and $c$ are three elements of $E$
satisfying $a <_1 c <_1 b$, then
both $\left[a, c\right]$ and $\left[c, b\right]$ are proper
subsets of $\left[a, b\right]$, and therefore
\begin{equation}
\text{both numbers }
\left|\left[a, c\right]\right| \text{ and }
\left|\left[c, b\right]\right| \text{ are smaller than }
\left|\left[a, b\right]\right|.
\label{pf.lem.admissible.cover.1}
\end{equation}

A pair $\left(p, q\right) \in P \times Q$ is said to be a
\textit{malposition} if it satisfies $q <_1 p$. Now, let us
assume (for the sake of contradiction) that there exists a
malposition. Fix a malposition $\left(u, v\right)$ for which the
value $\left|\left[u, v\right]\right|$ is minimum. Thus,
$u \in P$, $v \in Q$ and $v <_1 u$, but $v$ is not
$<_1$-covered by $u$ (since there exist no $p \in P$ and $q \in Q$
such that $q$ is $<_1$-covered by $p$). Hence, there exists a
$w \in E$ such that $v <_1 w <_1 u$ (since $v <_1 u$). Consider
this $w$. Applying \eqref{pf.lem.admissible.cover.1} to $a = v$,
$c = w$ and $b = u$, we see that both numbers
$\left|\left[u, w\right]\right|$ and
$\left|\left[w, v\right]\right|$ are smaller than
$\left|\left[u, v\right]\right|$, and therefore neither
$\left(u, w\right)$ nor $\left(w, v\right)$ is a malposition
(since we picked $\left(u, v\right)$ to be a malposition with
minimum $\left|\left[u, v\right]\right|$). But
$w \in E = P \cup Q$, so that either $w \in P$ or $w \in Q$.
If $w \in P$, then $\left(w, v\right)$ is a malposition;
if $w \in Q$, then $\left(u, w\right)$ is a malposition. In
either case, we obtain a contradiction to the fact that
neither $\left(u, w\right)$ nor $\left(w, v\right)$ is a malposition.
This contradiction shows that our assumption was wrong.
Hence, there exists no malposition. Consequently,
$\left(P, Q\right) \in \Adm \EE$.
\end{proof}

\begin{lemma}
\label{lem.tertispecial.subset}
Let $\EE = \left(E, <_1, <_2\right)$ be a tertispecial
double poset.
Let $\left(P,Q\right) \in \Adm \EE$. Then, $\EE\mid_P$ is
a tertispecial double poset.
\end{lemma}

\begin{proof}[Proof of Lemma~\ref{lem.tertispecial.subset}.]
We need to show that the double poset
$\EE\mid_P = \left(P, <_1, <_2\right)$ is tertispecial. In other
words, we need to show that if $a$ and $b$ are two elements
of $P$ such that $a$ is $<_1$-covered by $b$ as element of
the set $P$, then $a$ and $b$ are $<_2$-comparable.\footnote{Here, we
are using the following notation: If $T$ is a subset of $E$, and if
$u$ and $v$ are two elements of $T$, then we say that ``$u$ is
$<_1$-covered by $v$ as element of the set $T$'' if and only if
$u$ is $<_{1, T}$-covered by $v$, where
$<_{1,T}$ denotes the relation $<_1$ on the set $T$ (not the relation
$<_1$ on the set $E$). In general, saying that $u$ is $<_1$-covered
by $v$ as element of the set $T$ is \textbf{not} equivalent to
saying that $u$ is $<_1$-covered by $v$ as element of
the set $E$ (because there might be an element $w$ of $E$ satisfying
$u <_1 w <_1 v$, but no such element that belongs to $T$). Rather,
$u$ is $<_1$-covered by $v$ as element of the set $T$ if and only if
$u <_1 v$ and there exists no $w \in T$ satisfying $u <_1 w <_1 v$.}

Let $a$ and $b$ be two elements of $P$ such that $a$ is
$<_1$-covered by $b$ as element of the set $P$. Thus,
$a <_1 b$, and
\begin{equation}
\text{there exists no } c \in P \text{ satisfying }
a <_1 c <_1 b .
\label{pf.lem.tertispecial.subset.1}
\end{equation}
Now, if $c \in E$ is such that $a <_1 c <_1 b$, then $c$ must
belong to $P$\ \ \ \ \footnote{\textit{Proof.} Assume the
contrary. Thus, $c \notin P$. But
$\left(P, Q\right) \in \Adm \EE$. Thus, $P \cap Q = \varnothing$,
$P \cup Q = E$, and
\begin{equation}
\text{no } p \in P \text{ and } q \in Q \text{ satisfy }
q <_1 p .
\label{pf.lem.tertispecial.subset.2}
\end{equation}
From $c \in E$ and $c \notin P$, we obtain
$c \in E\setminus P \subseteq Q$ (since $P \cup Q = E$).
Applying \eqref{pf.lem.tertispecial.subset.2} to $p = b$ and
$q = c$, we thus conclude that we cannot have $c <_1 b$.
This contradicts $c <_1 b$. This contradiction shows that our
assumption was false, qed.}, which entails a
contradiction to \eqref{pf.lem.tertispecial.subset.1}. Thus, there
is no $c \in E$ satisfying $a <_1 c <_1 b$. Therefore (and
because we have $a <_1 b$), we see that $a$ is $<_1$-covered
by $b$ as element of the set $E$. Since $\EE$ is tertispecial,
this yields that $a$ and $b$ are $<_2$-comparable.

Thus, we have shown that if $a$ and $b$ are two elements
of $P$ such that $a$ is $<_1$-covered by $b$ as element of
the set $P$, then $a$ and $b$ are $<_2$-comparable. This proves
Lemma~\ref{lem.tertispecial.subset}.

(We could similarly show that $\EE\mid_Q$ is a tertispecial
double poset; but we will not use this.)
\end{proof}

\begin{lemma}
\label{lem.Gammaw.empty}Let ${\mathbf{E}}=\left(  E,<_{1},<_{2}\right)  $ be a
double poset. Let $w:E\rightarrow\left\{  1,2,3,\ldots\right\}  $ be a map.

\begin{enumerate}
\item[(a)] If $E=\varnothing$, then $\Gamma\left(  \mathbf{E},w\right)  =1$.

\item[(b)] If $E\neq\varnothing$, then $\varepsilon\left(  \Gamma\left(
\mathbf{E},w\right)  \right)  =0$.
\end{enumerate}
\end{lemma}

\begin{proof}
[Proof of Lemma \ref{lem.Gammaw.empty}.] (a) Part (a) is obvious (since there is
only one $\mathbf{E}$-partition when $E=\varnothing$).

(b) Observe that $\Gamma\left(  \mathbf{E},w\right)  $ is a homogeneous power
series of degree $\sum_{e\in E}w\left(  e\right)  $. When $E \neq \varnothing$,
this degree is $>0$, and thus the power series $\Gamma\left(  \mathbf{E},w\right)$
is annihilated by $\varepsilon$ (since $\varepsilon$ annihilates any
homogeneous power series in $\QSym$ whose degree is $> 0$).
\end{proof}

\begin{proof}[Proof of Theorem~\ref{thm.antipode.Gammaw}.]
We shall
prove Theorem~\ref{thm.antipode.Gammaw} by strong induction over
$\left|E\right|$. The induction base ($\left|E\right|=0$) is left to
the reader; we start with the induction step. Consider a
tertispecial double
poset $\EE = \left(E, <_1, <_2\right)$ with $\left|E\right| > 0$ and
a map $w : E \to \left\{1, 2, 3, \ldots\right\}$, and
assume that Theorem~\ref{thm.antipode.Gammaw} is proven for all
tertispecial double posets of smaller size.

We have $\left| E \right| > 0$ and thus $E \neq \varnothing$. Hence,
Lemma~\ref{lem.Gammaw.empty} (b) shows that
$\varepsilon \left( \Gamma\left(\EE, w\right) \right) = 0$. Thus,
$\left( u \circ \varepsilon \right) \left( \Gamma \left( \EE, w \right) \right)
= u \left( \underbrace{\varepsilon \left( \Gamma \left( \EE, w \right) \right)}_{= 0} \right)
= u \left( 0 \right) = 0$.

The upper commutative pentagon of \eqref{eq.antipode} shows that
$u \circ \varepsilon = m \circ \left(S \otimes \id\right) \circ
\Delta$. Applying both sides of this equality to
$\Gamma\left(\EE, w\right)$, we obtain
$\left(u \circ \varepsilon\right)
\left(\Gamma\left(\EE, w\right)\right)
= \left(m \circ \left(S \otimes \id\right) \circ
\Delta\right) \left(\Gamma\left(\EE, w\right)\right)$.
Since
$\left(u \circ \varepsilon\right)
\left(\Gamma\left(\EE, w\right)\right) = 0$, this
becomes
\begin{align}
0
&= \left(m \circ \left(S \otimes \id\right) \circ
\Delta\right) \left(\Gamma\left(\EE, w\right)\right)
= m \left(\left(S \otimes \id\right) \left(
\Delta \left(\Gamma\left(\EE, w\right)\right)\right)\right)
\nonumber\\
&= m \left(\left(S \otimes \id\right) \left(
\sum_{\left(P, Q\right) \in \Adm \EE}
\Gamma\left(\EE\mid_P, w\mid_P\right)
\otimes \Gamma\left(\EE\mid_Q, w\mid_Q\right) \right) \right)
\qquad \left(\text{by \eqref{eq.prop.Gammaw.coprod}}\right)
\nonumber\\
&= m \left(\sum_{\left(P, Q\right) \in \Adm \EE}
S\left(\Gamma\left(\EE\mid_P, w\mid_P\right)\right)
\otimes
\Gamma\left(\EE\mid_Q, w\mid_Q\right)\right)
\nonumber\\
&= \sum_{\left(P, Q\right) \in \Adm \EE}
S\left(\Gamma\left(\EE\mid_P, w\mid_P\right)\right)
\Gamma\left(\EE\mid_Q, w\mid_Q\right)
\nonumber \\
&= S \left(\Gamma\left(\EE\mid_E, w\mid_E\right)\right)
\Gamma\left(\EE\mid_\varnothing, w\mid_\varnothing\right)
+ \sum_{\substack{\left(P, Q\right) \in \Adm \EE ; \\
                  \left|P\right| < \left|E\right|}}
S\left(\Gamma\left(\EE\mid_P, w\mid_P\right)\right)
\Gamma\left(\EE\mid_Q, w\mid_Q\right)
\label{pf.thm.antipode.Gammaw.Req.1}
\end{align}
(since the only pair $\left(P, Q\right) \in \Adm \EE$ satisfying
$\left|P\right| = \left|E\right|$ is $\left(E, \varnothing\right)$,
whereas all other pairs $\left(P, Q\right) \in \Adm \EE$
satisfy $\left|P\right| < \left|E\right|$).

But whenever $\left(P, Q\right) \in \Adm \EE$ is such that
$\left|P\right| < \left|E\right|$, the double poset
$\EE\mid_P = \left(P, <_1, <_2\right)$ is tertispecial
(by Lemma~\ref{lem.tertispecial.subset}), and
therefore we have
$S\left(\Gamma\left(\EE\mid_P, w\mid_P\right)\right)
= S\left(\Gamma\left(\left(P, <_1, <_2\right), w\mid_P\right)\right)
= \left(-1\right)^{\left|P\right|}
\Gamma\left(\left(P, >_1, <_2\right), w\mid_P\right)$ (by the
induction hypothesis). Hence,
\eqref{pf.thm.antipode.Gammaw.Req.1} rewrites as
\begin{align}
0
&= S \left(\Gamma\left(\underbrace{\EE\mid_E}_{=\EE},
           \underbrace{w\mid_E}_{=w}\right)\right)
\underbrace{\Gamma\left(\EE\mid_\varnothing, w\mid_\varnothing\right)
           }_{\substack{
           =\Gamma\left(\left(\varnothing, <_1, <_2\right), w\mid_\varnothing\right)
           =1 \\ \text{ (by Lemma \ref{lem.Gammaw.empty} (a))}}}
   \nonumber \\
& \qquad + \sum_{\substack{\left(P, Q\right) \in \Adm \EE ; \\
                  \left|P\right| < \left|E\right|}}
\left(-1\right)^{\left|P\right|}
\Gamma\left(\left(P, >_1, <_2\right), w\mid_P\right)
\Gamma\left(\EE\mid_Q, w\mid_Q\right)
\nonumber \\
& = S\left(\Gamma\left(\EE, w\right)\right)
+ \sum_{\substack{\left(P, Q\right) \in \Adm \EE ; \\
                  \left|P\right| < \left|E\right|}}
\left(-1\right)^{\left|P\right|}
\Gamma\left(\left(P, >_1, <_2\right), w\mid_P\right)
\Gamma\left(\EE\mid_Q, w\mid_Q\right) .
\nonumber
\end{align}
Thus,
\begin{equation}
S\left(\Gamma\left(\EE, w\right)\right)
= - \sum_{\substack{\left(P, Q\right) \in \Adm \EE ; \\
                  \left|P\right| < \left|E\right|}}
\left(-1\right)^{\left|P\right|}
\Gamma\left(\left(P, >_1, <_2\right), w\mid_P\right)
\Gamma\left(\EE\mid_Q, w\mid_Q\right) .
\label{pf.thm.antipode.Gammaw.Seq}
\end{equation}
We shall now prove that
\begin{equation}
0 = \sum_{\left(P, Q\right) \in \Adm \EE}
\left(-1\right)^{\left|P\right|}
\Gamma\left(\left(P, >_1, <_2\right), w\mid_P\right)
\Gamma\left(\EE\mid_Q, w\mid_Q\right) .
\label{pf.thm.antipode.Gammaw.Req}
\end{equation}
But first, let us explain how this will complete our proof. In fact,
the only pair $\left(P, Q\right) \in \Adm \EE$ satisfying
$\left|P\right| = \left|E\right|$ is $\left(E, \varnothing\right)$,
whereas all other pairs $\left(P, Q\right) \in \Adm \EE$
satisfy $\left|P\right| < \left|E\right|$. Hence,
if \eqref{pf.thm.antipode.Gammaw.Req} is proven, then we can
conclude that
\begin{align*}
0 &= \sum_{\left(P, Q\right) \in \Adm \EE}
\left(-1\right)^{\left|P\right|}
\Gamma\left(\left(P, >_1, <_2\right), w\mid_P\right)
\Gamma\left(\EE\mid_Q, w\mid_Q\right) \\
&= \left(-1\right)^{\left|E\right|}
\Gamma\left(\left(E, >_1, <_2\right), \underbrace{w\mid_E}_{=w}\right)
\underbrace{\Gamma\left(\EE\mid_\varnothing, w\mid_\varnothing\right)
           }_{
           =\Gamma\left(\left(\varnothing, <_1, <_2\right), w\mid_\varnothing\right)
           =1}
\\
&\qquad + \sum_{\substack{\left(P, Q\right) \in \Adm \EE ; \\
                  \left|P\right| < \left|E\right|}}
\left(-1\right)^{\left|P\right|}
\Gamma\left(\left(P, >_1, <_2\right), w\mid_P\right)
\Gamma\left(\EE\mid_Q, w\mid_Q\right)
\\
&= \left(-1\right)^{\left|E\right|} \Gamma\left(\left(E, >_1, <_2\right), w\right) \\
& \qquad
+ \sum_{\substack{\left(P, Q\right) \in \Adm \EE ; \\
                  \left|P\right| < \left|E\right|}}
\left(-1\right)^{\left|P\right|}
\Gamma\left(\left(P, >_1, <_2\right), w\mid_P\right)
\Gamma\left(\EE\mid_Q, w\mid_Q\right) ,
\end{align*}
so that
\begin{align*}
\left(-1\right)^{\left|E\right|} \Gamma\left(\left(E, >_1, <_2\right), w\right)
&= - \sum_{\substack{\left(P, Q\right) \in \Adm \EE ; \\
                  \left|P\right| < \left|E\right|}}
\left(-1\right)^{\left|P\right|}
\Gamma\left(\left(P, >_1, <_2\right), w\mid_P\right)
\Gamma\left(\EE\mid_Q, w\mid_Q\right) \\
&= S\left(\Gamma\left(\underbrace{\EE}_{=\left(E, <_1, <_2\right)}, w\right)\right)
\qquad \left(\text{by \eqref{pf.thm.antipode.Gammaw.Seq}}\right) \\
&= S\left(\Gamma\left(\left(E, <_1, <_2\right), w\right)\right) ,
\end{align*}
and thus
$S\left(\Gamma\left(\left(E, <_1, <_2\right)\right), w\right)
= \left(-1\right)^{\left|E\right|}
\Gamma\left(\left(E, >_1, <_2\right), w\right)$, which completes the
induction step and thus the proof of Theorem~\ref{thm.antipode.Gammaw}.
It thus remains to prove \eqref{pf.thm.antipode.Gammaw.Req}.

For every subset $P$ of $E$, we have

\begin{align}
\Gamma\left(\left(P, >_1, <_2\right), w\mid_P\right)
&= \sum_{\pi \text{ is a }\left(P, >_1, <_2\right)\text{-partition}}
\xx_{\pi, w\mid_P}
\nonumber \\
& \qquad \left(\text{by the definition of }
 \Gamma\left(\left(P, >_1, <_2\right), w\mid_P\right) \right)
\nonumber \\
& = \sum_{\sigma \text{ is a }\left(P, >_1, <_2\right)\text{-partition}}
\xx_{\sigma, w\mid_P} .
\label{pf.thm.antipode.Gammaw.Req.pf.Gamma1}
\end{align}

For every subset $Q$ of $E$, we have

\begin{align}
\Gamma\left(\underbrace{\EE\mid_Q}_{=\left(Q, <_1, <_2\right)}, w\mid_Q\right)
&= \Gamma\left(\left(Q, <_1, <_2\right), w\mid_Q\right)
\nonumber \\
&= \sum_{\pi \text{ is a }\left(Q, <_1, <_2\right)\text{-partition}}
\xx_{\pi, w\mid_Q}
\nonumber \\
& \qquad \left(\text{by the definition of }
 \Gamma\left(\left(Q, <_1, <_2\right), w\mid_Q\right) \right)
\nonumber \\
& = \sum_{\tau \text{ is a }\left(Q, <_1, <_2\right)\text{-partition}}
\xx_{\tau, w\mid_Q} .
\label{pf.thm.antipode.Gammaw.Req.pf.Gamma2}
\end{align}

Now,
\begin{align*}
& \sum_{\left(P, Q\right) \in \Adm \EE}
\left(-1\right)^{\left|P\right|}
\underbrace{\Gamma\left(\left(P, >_1, <_2\right), w\mid_P\right)}_{
 \substack{ = \sum_{\sigma \text{ is a }\left(P, >_1, <_2\right)\text{-partition}}
            \xx_{\sigma, w\mid_P} \\
            \text{(by \eqref{pf.thm.antipode.Gammaw.Req.pf.Gamma1})}}}
\underbrace{\Gamma\left(\EE\mid_Q, w\mid_Q\right)}_{
 \substack{ = \sum_{\tau \text{ is a }\left(Q, <_1, <_2\right)\text{-partition}}
            \xx_{\tau, w\mid_Q} \\
            \text{(by \eqref{pf.thm.antipode.Gammaw.Req.pf.Gamma2})}}}
\\
&= \sum_{\left(P, Q\right) \in \Adm \EE}
\left(-1\right)^{\left|P\right|}
\left(\sum_{\sigma \text{ is a }\left(P, >_1, <_2\right)\text{-partition}}
\xx_{\sigma, w\mid_P}\right)
\left(\sum_{\tau \text{ is a }\left(Q, <_1, <_2\right)\text{-partition}}
\xx_{\tau, w\mid_Q}\right) \\
&= \sum_{\left(P, Q\right) \in \Adm \EE}
\left(-1\right)^{\left|P\right|}
\sum_{\sigma \text{ is a }\left(P, >_1, <_2\right)\text{-partition}}
\sum_{\tau \text{ is a }\left(Q, <_1, <_2\right)\text{-partition}}
\xx_{\sigma, w\mid_P} \xx_{\tau, w\mid_Q} \\
&= \sum_{\left(P, Q\right) \in \Adm \EE}
\left(-1\right)^{\left|P\right|}
\sum_{\substack{\left(\sigma, \tau\right); \\
                \sigma : P \to \left\{1, 2, 3, \ldots\right\}; \\
                \tau : Q \to \left\{1, 2, 3, \ldots\right\}; \\
                \sigma \text{ is a }\left(P, >_1, <_2\right)\text{-partition;} \\
                \tau \text{ is a }\left(Q, <_1, <_2\right)\text{-partition}}}
\xx_{\sigma, w\mid_P} \xx_{\tau, w\mid_Q} \\
&= \sum_{\left(P, Q\right) \in \Adm \EE}
\left(-1\right)^{\left|P\right|}
\sum_{\substack{\pi : E \to \left\{1, 2, 3, \ldots\right\}; \\
                \pi\mid_P \text{ is a }\left(P, >_1, <_2\right)\text{-partition;} \\
                \pi\mid_Q \text{ is a }\left(Q, <_1, <_2\right)\text{-partition}}}
\underbrace{\xx_{\pi\mid_P, w\mid_P} \xx_{\pi\mid_Q, w\mid_Q}}_{=\xx_{\pi, w}} \\
& \qquad \left(
 \begin{array}{c}
 \text{here, we have substituted } \left(\pi\mid_P, \pi\mid_Q\right)
 \text{ for } \left(\sigma, \tau\right) \text{ in the inner sum,} \\
 \text{ since every pair } \left(\sigma, \tau\right)
 \text{ consisting of a map }
 \sigma : P \to \left\{1, 2, 3, \ldots\right\} \\
 \text{ and a map } \tau : Q \to \left\{1, 2, 3, \ldots\right\} \\
 \text{ can be written as } \left(\pi\mid_P, \pi\mid_Q\right)
 \text{ for a unique }
 \pi : E \to \left\{1, 2, 3, \ldots\right\} \\
 \text{(namely, for the }
 \pi : E \to \left\{1, 2, 3, \ldots\right\}
 \text{ that is defined to send every } \\
 e \in P \text{ to }
 \sigma\left(e\right) \text{ and to send every } e \in Q
 \text{ to } \tau\left(e\right) \text{)}
 \end{array}
 \right) \\
& = \sum_{\left(P, Q\right) \in \Adm \EE}
\left(-1\right)^{\left|P\right|}
\sum_{\substack{\pi : E \to \left\{1, 2, 3, \ldots\right\}; \\
                \pi\mid_P \text{ is a }\left(P, >_1, <_2\right)\text{-partition;} \\
                \pi\mid_Q \text{ is a }\left(Q, <_1, <_2\right)\text{-partition}}}
\xx_{\pi, w} \\
& = \sum_{\pi : E \to \left\{1, 2, 3, \ldots\right\}}
\left(
\sum_{\substack{\left(P, Q\right) \in \Adm \EE ; \\
                \pi\mid_P \text{ is a }\left(P, >_1, <_2\right)\text{-partition;} \\
                \pi\mid_Q \text{ is a }\left(Q, <_1, <_2\right)\text{-partition}}}
\left(-1\right)^{\left|P\right|}
\right)
\xx_{\pi, w} .
\end{align*}
%\footnote{Here, for the third-to-last equality sign, we have
%replaced the indices $\sigma$ and $\tau$ by a single index
%$\pi$, which is the map $E \to \left\{1, 2, 3, \ldots\right\}$ which
%is defined to send every $e \in P$ to $\sigma\left(e\right)$ and to
%send every $e \in Q$ to $\tau\left(e\right)$.}.
In order to prove
that this sum is $0$ (and thus to prove
\eqref{pf.thm.antipode.Gammaw.Req} and finish our proof of
Theorem~\ref{thm.antipode.Gammaw}), it therefore is enough to show
that for every map $\pi : E \to \left\{1, 2, 3, \ldots\right\}$, we
have
\begin{equation}
\sum_{\substack{\left(P, Q\right) \in \Adm \EE ; \\
                \pi\mid_P \text{ is a }\left(P, >_1, <_2\right)\text{-partition;} \\
                \pi\mid_Q \text{ is a }\left(Q, <_1, <_2\right)\text{-partition}}}
\left(-1\right)^{\left|P\right|}
= 0 .
\label{pf.thm.antipode.Gammaw.signrev}
\end{equation}

\begin{vershort}
Hence, let us fix a map $\pi : E \to \left\{1, 2, 3, \ldots\right\}$.
Our goal is now to prove \eqref{pf.thm.antipode.Gammaw.signrev}.
To do so, we denote by $Z$ the set of all
$\left(P, Q\right) \in \Adm \EE$ such that
$\pi\mid_P$ is a $\left(P, >_1, <_2\right)$-partition and
$\pi\mid_Q$ is a $\left(Q, <_1, <_2\right)$-partition. We are going
to define an involution $T : Z \to Z$ of the set $Z$ having the
property that, for any $\left(P, Q\right) \in Z$, if we write
$T\left(\left(P, Q\right)\right)$ in the form
$\left(P', Q'\right)$, then
$\left(-1\right)^{\left|P'\right|}
= - \left(-1\right)^{\left|P\right|}$. Once such an involution $T$
is found, it will be clear that it matches the addends on the left
hand side of \eqref{pf.thm.antipode.Gammaw.signrev} into pairs of
mutually cancelling addends\footnote{In fact, the
$\left(-1\right)^{\left|P'\right|}
= - \left(-1\right)^{\left|P\right|}$ condition makes it clear that
$T$ has no fixed points. Therefore, to each addend on the left
hand side of \eqref{pf.thm.antipode.Gammaw.signrev} corresponds an
addend with opposite sign, which cancels it: Namely, for each
$ \left(A, B\right) \in Z$, the addend for
$\left(P, Q\right) = \left(A, B\right)$ is cancelled by the addend
for $\left(P, Q\right) = T\left(\left(A, B\right)\right)$.}, and so
\eqref{pf.thm.antipode.Gammaw.signrev}
will follow and we will be done. It thus remains to find $T$.
\end{vershort}

\begin{verlong}
Hence, let us fix a map $\pi : E \to \left\{1, 2, 3, \ldots\right\}$.
Our goal is now to prove \eqref{pf.thm.antipode.Gammaw.signrev}.
To do so, we denote by $Z$ the set of all
$\left(P, Q\right) \in \Adm \EE$ such that
$\pi\mid_P$ is a $\left(P, >_1, <_2\right)$-partition and
$\pi\mid_Q$ is a $\left(Q, <_1, <_2\right)$-partition. We are going
to define an involution $T : Z \to Z$ of the set $Z$ having the
property that, for any $\left(P, Q\right) \in Z$, if we write
$T\left(\left(P, Q\right)\right)$ in the form
$\left(P', Q'\right)$, then
$\left(-1\right)^{\left|P'\right|}
= - \left(-1\right)^{\left|P\right|}$. Once such an involution $T$
is found, the equality \eqref{pf.thm.antipode.Gammaw.signrev} will
follow\footnote{Here is the argument in detail:

Assume that we have found an involution $T:Z\rightarrow Z$ of the set $Z$
having the property that, for any $\left(  P,Q\right)  \in Z$, if we write
$T\left(  \left(  P,Q\right)  \right)  $ in the form $\left(  P^{\prime
},Q^{\prime}\right)  $, then
\begin{equation}
\left(  -1\right)  ^{\left\vert P^{\prime}\right\vert }=-\left(  -1\right)
^{\left\vert P\right\vert }.\label{pf.thm.antipode.Gammaw.signrev.final.1}%
\end{equation}
We now need to prove the equality \eqref{pf.thm.antipode.Gammaw.signrev}.

The map $Z$ is an involution, and thus a bijection.

Now, let $Z_{0}$ be the subset $\left\{  \left(  P,Q\right)  \in
Z\ \mid\ \left\vert P\right\vert \text{ is even}\right\}  $ of $Z$. Thus, for
every $\left(  P,Q\right)  \in Z$, we have the following logical equivalence:%
\begin{equation}
\left(  \left(  P,Q\right)  \in Z_{0}\right)  \ \Longleftrightarrow\ \left(
\left\vert P\right\vert \text{ is even}\right)
.\label{pf.thm.antipode.Gammaw.signrev.final.defZ0}%
\end{equation}
Hence, for every $\left(  P,Q\right)  \in Z$, we have the following logical
equivalence:%
\begin{align}
\left(  \left(  P,Q\right)  \notin Z_{0}\right)  \   & \Longleftrightarrow
\ \left(  \left\vert P\right\vert \text{ is not even}\right)  \nonumber\\
& \ \ \ \ \ \ \ \ \ \ \left(
\begin{array}
[c]{c}%
\text{this equivalence is obtained from
(\ref{pf.thm.antipode.Gammaw.signrev.final.defZ0})}\\
\text{by replacing each part by its negation}%
\end{array}
\right)  \nonumber\\
& \Longleftrightarrow\ \left(  \left\vert P\right\vert \text{ is odd}\right)
.\label{pf.thm.antipode.Gammaw.signrev.final.defZ1}%
\end{align}


Now, for every $\left(  P,Q\right)  \in Z$, we have the following logical
equivalence:%
\begin{equation}
\left(  \left(  P,Q\right)  \in Z_{0}\right)  \ \Longleftrightarrow\ \left(
T\left(  \left(  P,Q\right)  \right)  \notin Z_{0}\right)
.\label{pf.thm.antipode.Gammaw.signrev.final.3}%
\end{equation}


\textit{Proof of \ref{pf.thm.antipode.Gammaw.signrev.final.3}):} Let $\left(
P,Q\right)  \in Z$. Write $T\left(  \left(  P,Q\right)  \right)  \in Z$ in the
form $\left(  P^{\prime},Q^{\prime}\right)  $. Then,
(\ref{pf.thm.antipode.Gammaw.signrev.final.defZ1}) (applied to $\left(
P^{\prime},Q^{\prime}\right)  $ instead of $\left(  P,Q\right)  $) shows that
we have the following logical equivalence:%
\[
\left(  \left(  P^{\prime},Q^{\prime}\right)  \notin Z_{0}\right)
\ \Longleftrightarrow\ \left(  \left\vert P^{\prime}\right\vert \text{ is
odd}\right)  .
\]
Thus, we have the following logical equivalence:%
\begin{align*}
\left(  \left(  P^{\prime},Q^{\prime}\right)  \notin Z_{0}\right)  \   &
\Longleftrightarrow\ \left(  \left\vert P^{\prime}\right\vert \text{ is
odd}\right)  \ \Longleftrightarrow\ \left(  \underbrace{\left(  -1\right)
^{\left\vert P^{\prime}\right\vert }}_{\substack{=-\left(  -1\right)
^{\left\vert P\right\vert }\\\text{(by
(\ref{pf.thm.antipode.Gammaw.signrev.final.1}))}}}=-1\right)
\ \Longleftrightarrow\ \left(  -\left(  -1\right)  ^{\left\vert P\right\vert
}=-1\right)  \\
& \Longleftrightarrow\ \left(  \left(  -1\right)  ^{\left\vert P\right\vert
}=1\right)  \ \Longleftrightarrow\ \left(  \left\vert P\right\vert \text{ is
even}\right)  \ \Longleftrightarrow\ \left(  \left(  P,Q\right)  \in
Z_{0}\right)  \\
& \ \ \ \ \ \ \ \ \ \ \left(  \text{by
(\ref{pf.thm.antipode.Gammaw.signrev.final.defZ0})}\right)  .
\end{align*}
Hence, we have the following logical equivalence:%
\[
\left(  \left(  P,Q\right)  \in Z_{0}\right)  \ \Longleftrightarrow\ \left(
\underbrace{\left(  P^{\prime},Q^{\prime}\right)  }_{=T\left(  \left(
P,Q\right)  \right)  }\notin Z_{0}\right)  \ \Longleftrightarrow\ \left(
T\left(  \left(  P,Q\right)  \right)  \notin Z_{0}\right)  .
\]
This proves (\ref{pf.thm.antipode.Gammaw.signrev.final.3}).

Now,%
\[
\sum_{\substack{\left(  P,Q\right)  \in Z;\\\left\vert P\right\vert \text{ is
even}}}\underbrace{\left(  -1\right)  ^{\left\vert P\right\vert }%
}_{\substack{=1\\\text{(since }\left\vert P\right\vert \text{ is even)}%
}}=\underbrace{\sum_{\substack{\left(  P,Q\right)  \in Z;\\\left\vert
P\right\vert \text{ is even}}}}_{\substack{=\sum_{\substack{\left(
P,Q\right)  \in Z;\\\left(  P,Q\right)  \in Z_{0}}}\\\text{(because for every
}\left(  P,Q\right)  \in Z\text{, the condition}\\\left(  \left\vert
P\right\vert \text{ is even}\right)  \text{ is equivalent to }\left(  \left(
P,Q\right)  \in Z_{0}\right)  \\\text{(by
(\ref{pf.thm.antipode.Gammaw.signrev.final.defZ0})))}}}1=\sum
_{\substack{\left(  P,Q\right)  \in Z;\\\left(  P,Q\right)  \in Z_{0}}}1
\]
and%
\begin{align*}
\sum_{\substack{\left(  P,Q\right)  \in Z;\\\left\vert P\right\vert \text{ is
odd}}}\underbrace{\left(  -1\right)  ^{\left\vert P\right\vert }%
}_{\substack{=-1\\\text{(since }\left\vert P\right\vert \text{ is odd)}}}  &
=\underbrace{\sum_{\substack{\left(  P,Q\right)  \in Z;\\\left\vert
P\right\vert \text{ is odd}}}}_{\substack{=\sum_{\substack{\left(  P,Q\right)
\in Z;\\\left(  P,Q\right)  \notin Z_{0}}}\\\text{(because for every }\left(
P,Q\right)  \in Z\text{, the condition}\\\left(  \left\vert P\right\vert
\text{ is odd}\right)  \text{ is equivalent to }\left(  \left(  P,Q\right)
\notin Z_{0}\right)  \\\text{(by
(\ref{pf.thm.antipode.Gammaw.signrev.final.defZ1})))}}}\left(  -1\right)
=\sum_{\substack{\left(  P,Q\right)  \in Z;\\\left(  P,Q\right)  \notin Z_{0}%
}}\left(  -1\right)  \\
& =\underbrace{\sum_{\substack{\left(  P,Q\right)  \in Z;\\T\left(  \left(
P,Q\right)  \right)  \notin Z_{0}}}}_{\substack{=\sum_{\substack{\left(
P,Q\right)  \in Z;\\\left(  P,Q\right)  \in Z_{0}}}\\\text{(because for every
}\left(  P,Q\right)  \in Z\text{, the condition}\\\left(  T\left(  \left(
P,Q\right)  \right)  \notin Z_{0}\right)  \text{ is equivalent to }\left(
\left(  P,Q\right)  \in Z_{0}\right)  \\\text{(by
(\ref{pf.thm.antipode.Gammaw.signrev.final.3})))}}}\left(  -1\right) \\
&\ \ \ \ \ \ \ \ \ \ \left(
\begin{array}
[c]{c}%
\text{here, we have substituted }T\left(  \left(  P,Q\right)  \right)  \text{
for }\left(  P,Q\right)  \\
\text{in the sum, since the map }T:Z\rightarrow Z\text{ is a bijection}%
\end{array}
\right)  \\
& =\sum_{\substack{\left(  P,Q\right)  \in Z;\\\left(  P,Q\right)  \in Z_{0}%
}}\left(  -1\right)  =-\sum_{\substack{\left(  P,Q\right)  \in Z;\\\left(
P,Q\right)  \in Z_{0}}}1.
\end{align*}
Finally,
\begin{align*}
& \underbrace{\sum_{\substack{\left(  P,Q\right)  \in\operatorname{Adm}%
{\mathbf{E}};\\\pi\mid_{P}\text{ is a }\left(  P,>_{1},<_{2}\right)
\text{-partition;}\\\pi\mid_{Q}\text{ is a }\left(  Q,<_{1},<_{2}\right)
\text{-partition}}}}_{\substack{=\sum_{\left(  P,Q\right)  \in Z}\\\text{(by
the definition of }Z\text{)}}}\left(  -1\right)  ^{\left\vert P\right\vert
}\\
& =\sum_{\left(  P,Q\right)  \in Z}\left(  -1\right)  ^{\left\vert
P\right\vert }=\underbrace{\sum_{\substack{\left(  P,Q\right)  \in
Z;\\\left\vert P\right\vert \text{ is even}}}\left(  -1\right)  ^{\left\vert
P\right\vert }}_{=\sum_{\substack{\left(  P,Q\right)  \in Z;\\\left(
P,Q\right)  \in Z_{0}}}1}+\underbrace{\sum_{\substack{\left(  P,Q\right)  \in
Z;\\\left\vert P\right\vert \text{ is odd}}}\left(  -1\right)  ^{\left\vert
P\right\vert }}_{=-\sum_{\substack{\left(  P,Q\right)  \in Z;\\\left(
P,Q\right)  \in Z_{0}}}1}\\
& =\sum_{\substack{\left(  P,Q\right)  \in Z;\\\left(  P,Q\right)  \in Z_{0}%
}}1+\left(  -\sum_{\substack{\left(  P,Q\right)  \in Z;\\\left(  P,Q\right)
\in Z_{0}}}1\right)  =0.
\end{align*}
Thus, \eqref{pf.thm.antipode.Gammaw.signrev} is proven.
} and we will be done. It thus remains to find $T$.
\end{verlong}

The definition of $T$ is simple (although it will take us a while to prove
that it is well-defined): Let $F$ be the subset of $E$ consisting of those
$e\in E$ which have minimum $\pi\left(  e\right)  $. Then, $F$ is a nonempty
subposet of the poset $\left(  E,<_{2}\right)  $, and hence has a minimal
element $f$ (that is, an element $f$ such that no $g\in F$ satisfies $g<_{2}
f$). Fix such an $f$. Now, the map $T$ sends a $\left(  P,Q\right)  \in Z$ to
$
\begin{cases}
\left(  P\cup\left\{  f\right\}  ,Q\setminus\left\{  f\right\}  \right)  , &
\text{if }f\notin P;\\
\left(  P\setminus\left\{  f\right\}  ,Q\cup\left\{  f\right\}  \right)  , &
\text{if }f\in P
\end{cases}
$.

In order to prove that the map $T$ is well-defined, we need to prove that its
output values all belong to $Z$. In other words, we need to prove that
\begin{equation}
\begin{cases}
\left(  P\cup\left\{  f\right\}  ,Q\setminus\left\{  f\right\}  \right)  , &
\text{if }f\notin P;\\
\left(  P\setminus\left\{  f\right\}  ,Q\cup\left\{  f\right\}  \right)  , &
\text{if }f\in P
\end{cases}
\in Z
\label{pf.thm.antipode.Gammaw.Zwd}
\end{equation}
for every $\left(  P,Q\right)  \in Z$.

\textit{Proof of \eqref{pf.thm.antipode.Gammaw.Zwd}:} Fix $\left(  P,Q\right)
\in Z$. Thus, $\left(  P,Q\right)  $ is an element of
$\operatorname{Adm} \EE$ with the property that
$\pi\mid_{P}$ is a $\left(  P,>_{1},<_{2}\right)  $-partition
and $\pi\mid_{Q}$ is a $\left(  Q,<_{1},<_{2}\right)  $-partition.

From $\left(  P,Q\right)  \in\operatorname{Adm}{ \EE }$, we see that
$P\cap Q=\varnothing$ and $P\cup Q=E$, and furthermore that
\begin{equation}
\text{no }p\in P\text{ and }q\in Q\text{ satisfy }q<_{1}
p.
\label{pf.thm.antipode.Gammaw.Zwd.pf.Adm}
\end{equation}


We know that $f$ belongs to the set $F$, which is the subset of $E$ consisting
of those $e\in E$ which have minimum $\pi\left(  e\right)  $. Thus,
\begin{equation}
\pi\left(  f\right)  \leq\pi\left(  h\right)  \qquad\text{for every }h\in
E.
\label{pf.thm.antipode.Gammaw.Zwd.pf.min}
\end{equation}


Moreover,
\begin{equation}
\pi\left(  f\right)  <\pi\left(  h\right)
\qquad\text{for every } h\in E \text{ satisfying } h<_{2}f
\label{pf.thm.antipode.Gammaw.Zwd.pf.min2}
\end{equation}
\footnote{\textit{Proof of \eqref{pf.thm.antipode.Gammaw.Zwd.pf.min2}:} Let
$h\in E$ be such that $h<_{2}f$. We must prove
\eqref{pf.thm.antipode.Gammaw.Zwd.pf.min2}. Indeed, assume the contrary. Thus,
$\pi\left(  f\right)  \geq\pi\left(  h\right)  $. Combined with
\eqref{pf.thm.antipode.Gammaw.Zwd.pf.min}, this shows that $\pi\left(
f\right)  =\pi\left(  h\right)  $. Our definition of $F$ shows that $F$ is the
subset of $E$ consisting of those $e\in E$ satisfying $\pi\left(  e\right)
=\pi\left(  f\right)  $ (since $f\in F$). Therefore, $h\in F$ (since
$\pi\left(  h\right)  =\pi\left(  f\right)  $). But $f$ is a minimal element
of $F$. In other words, no $g\in F$ satisfies $g<_{2}f$. This contradicts the
fact that $h\in F$ satisfies $h<_{2}f$. This contradiction proves that our
assumption was wrong, qed.}.

We need to prove \eqref{pf.thm.antipode.Gammaw.Zwd}. We are in one of the
following two cases:

\textit{Case 1:} We have $f\in P$.

\textit{Case 2:} We have $f\notin P$.

Let us first consider Case 1. In this case, we have $f\in P$.

Recall that $P\cap Q=\varnothing$ and $P\cup Q=E$. From this, we easily obtain
$\left(  P\setminus\left\{  f\right\}  \right)  \cap\left(  Q\cup\left\{
f\right\}  \right)  =\varnothing$ and $\left(  P\setminus\left\{  f\right\}
\right)  \cup\left(  Q\cup\left\{  f\right\}  \right)  =E$.

Furthermore, there exist no $p\in P\setminus\left\{  f\right\}  $ and $q\in
Q\cup\left\{  f\right\}  $ such that $q$ is $<_{1}$-covered by
$p$\ \ \ \ \footnote{\textit{Proof.}
Assume the contrary. Thus, there exist $p\in
P\setminus\left\{  f\right\}  $ and $q\in Q\cup\left\{  f\right\}  $ such that
$q$ is $<_{1}$-covered by $p$. Consider such $p$ and $q$.
\par
We know that $q$ is $<_{1}$-covered by $p$, and thus we have $q<_{1}p$. Also,
$p\in P\setminus\left\{  f\right\}  \subseteq P$. Hence, if we had $q\in Q$,
then we would obtain a contradiction to
\eqref{pf.thm.antipode.Gammaw.Zwd.pf.Adm}. Hence, we cannot have $q\in Q$.
Therefore, $q=f$ (since $q\in Q\cup\left\{  f\right\}  $ but not $q\in Q$).
Hence, $f=q<_{1}p$, so that $p>_{1}f$. Therefore, $\pi\left(  p\right)
\leq\pi\left(  f\right)  $ (since $\pi\mid_{P}$ is a
$\left(  P,>_{1},<_{2}\right)  $-partition, and since both $f$
and $p$ belong to $P$).
\par
Now, recall that $q$ is $<_{1}$-covered by $p$. Hence, $q$ and $p$ are
$<_{2}$-comparable (since $E$ is tertispecial).
In other words, $f$ and $p$ are
$<_{2}$-comparable (since $q=f$). In other words, either $f<_{2}p$ or $f=p$ or
$p<_{2}f$. But $p<_{2}f$ cannot hold (because if we had $p<_{2}f$, then
\eqref{pf.thm.antipode.Gammaw.Zwd.pf.min2} (applied to $h=p$) would lead to
$\pi\left(  f\right)  <\pi\left(  p\right)  $, which would contradict
$\pi\left(  p\right)  \leq\pi\left(  f\right)  $), and $f=p$ cannot hold
either (since $f<_{1}p$). Thus, we must have $f<_{2}p$.
\par
Now, $\pi\mid_{P}$ is a $\left(  P,>_{1},<_{2}\right)  $-partition. Hence,
$\pi\left(  p\right)  <\pi\left(  f\right)  $ (since $p>_{1}f$ and $f<_{2}p$,
and since $p$ and $f$ both lie in $P$).
But \eqref{pf.thm.antipode.Gammaw.Zwd.pf.min} (applied to $h=p$) shows that
$\pi\left(  f\right)  \leq\pi\left(  p\right)  $. Hence, $\pi\left(  p\right)
<\pi\left(  f\right)  \leq\pi\left(  p\right)  $, a contradiction. Thus, our
assumption was wrong, qed.}. Hence, Lemma \ref{lem.admissible.cover} (applied
to $P\setminus\left\{  f\right\}  $ and $Q\cup\left\{  f\right\}  $ instead of
$P$ and $Q$) shows that $\left(  P\setminus\left\{  f\right\}  ,Q\cup\left\{
f\right\}  \right)  \in\operatorname{Adm}{ \EE }$.

Furthermore, $\pi\mid_{P}$ is a $\left(  P,>_{1},<_{2}\right)  $-partition,
and therefore $\pi\mid_{P\setminus\left\{  f\right\}  }$ is a $\left(
P\setminus\left\{  f\right\}  ,>_{1},<_{2}\right)  $-partition (since
$P\setminus\left\{  f\right\}  \subseteq P$).

Furthermore, $\pi\mid_{Q\cup\left\{  f\right\}  }$ is a $\left(  Q\cup\left\{
f\right\}  ,<_{1},<_{2}\right)  $-partition\footnote{\textit{Proof.} In order
to prove this, we need to verify the following two claims:
\par
\textit{Claim 1:} Every $a\in Q\cup\left\{  f\right\}  $ and $b\in
Q\cup\left\{  f\right\}  $ satisfying $a<_{1}b$ satisfy $\pi\left(  a\right)
\leq\pi\left(  b\right)  $;
\par
\textit{Claim 2:} Every $a\in Q\cup\left\{  f\right\}  $ and $b\in
Q\cup\left\{  f\right\}  $ satisfying $a<_{1}b$ and $b<_{2}a$ satisfy
$\pi\left(  a\right)  <\pi\left(  b\right)  $.
\par
\textit{Proof of Claim 1:} Let $a\in Q\cup\left\{  f\right\}  $ and $b\in
Q\cup\left\{  f\right\}  $ be such that $a<_{1}b$. We need to prove that
$\pi\left(  a\right)  \leq\pi\left(  b\right)  $. If $a=f$, then this follows
immediately from \eqref{pf.thm.antipode.Gammaw.Zwd.pf.min} (applied to $h=b$).
Hence, we WLOG assume that $a\neq f$. Thus, $a\in Q$ (since $a\in
Q\cup\left\{  f\right\}  $). Now, if $b\in P$, then $a<_{1}b$ contradicts
\eqref{pf.thm.antipode.Gammaw.Zwd.pf.Adm} (applied to $p=b$ and $q=a$). Hence,
we cannot have $b\in P$. Therefore, $b\in E\setminus P=Q$ (since $P\cap
Q=\varnothing$ and $P\cup Q=E$). Thus, $\pi\left(  a\right)  \leq\pi\left(
b\right)  $ follows immediately from the fact that $\pi\mid_{Q}$ is a $\left(
Q,<_{1},<_{2}\right)  $-partition (since $a \in Q$ and $b \in Q$).
This proves Claim 1.
\par
\textit{Proof of Claim 2:} Let $a\in Q\cup\left\{  f\right\}  $ and $b\in
Q\cup\left\{  f\right\}  $ be such that $a<_{1}b$ and $b<_{2}a$. We need to
prove that $\pi\left(  a\right)  <\pi\left(  b\right)  $. If $a=f$, then this
follows immediately from \eqref{pf.thm.antipode.Gammaw.Zwd.pf.min2} (applied
to $h=b$). Hence, we WLOG assume that $a\neq f$. Thus, $a\in Q$ (since $a\in
Q\cup\left\{  f\right\}  $). Now, if $b\in P$, then $a<_{1}b$ contradicts
\eqref{pf.thm.antipode.Gammaw.Zwd.pf.Adm} (applied to $p=b$ and $q=a$). Hence,
we cannot have $b\in P$. Therefore, $b\in E\setminus P=Q$ (since $P\cap
Q=\varnothing$ and $P\cup Q=E$). Thus, $\pi\left(  a\right)  <\pi\left(
b\right)  $ follows immediately from the fact that $\pi\mid_{Q}$ is a $\left(
Q,<_{1},<_{2}\right)  $-partition (since $a \in Q$ and $b \in Q$).
This proves Claim 2.
\par
Now, both Claim 1 and Claim 2 are proven, and we are done.}.

Altogether, we now know that $\left(  P\setminus\left\{  f\right\}
,Q\cup\left\{  f\right\}  \right)  \in\operatorname{Adm}{ \EE }$, that
$\pi\mid_{P\setminus\left\{  f\right\}  }$ is a $\left(  P\setminus\left\{
f\right\}  ,>_{1},<_{2}\right)  $-partition, and that $\pi\mid_{Q\cup\left\{
f\right\}  }$ is a $\left(  Q\cup\left\{  f\right\}  ,<_{1},<_{2}\right)
$-partition. In other words, $\left(  P\setminus\left\{  f\right\}
,Q\cup\left\{  f\right\}  \right)  \in Z$ (by the definition of $Z$). Thus,
\begin{align*}
\begin{cases}
\left(  P\cup\left\{  f\right\}  ,Q\setminus\left\{  f\right\}  \right)  , &
\text{if }f\notin P;\\
\left(  P\setminus\left\{  f\right\}  ,Q\cup\left\{  f\right\}  \right)  , &
\text{if }f\in P
\end{cases}
& =\left(  P\setminus\left\{  f\right\}  ,Q\cup\left\{  f\right\}  \right)
\ \ \ \ \ \ \ \ \ \ \left(  \text{since }f\in P\right)  \\
& \in Z.
\end{align*}
Hence, \eqref{pf.thm.antipode.Gammaw.Zwd} is proven in Case 1.

Let us next consider Case 2. In this case, we have $f\notin P$.

Recall that $P\cap Q=\varnothing$ and $P\cup Q=E$. From this, we easily obtain
$\left(  P\cup\left\{  f\right\}  \right)  \cap\left(  Q\setminus\left\{
f\right\}  \right)  =\varnothing$ and $\left(  P\cup\left\{  f\right\}
\right)  \cup\left(  Q\setminus\left\{  f\right\}  \right)  =E$.

Furthermore, there exist no $p\in P\cup\left\{  f\right\}  $ and $q\in
Q\setminus\left\{  f\right\}  $ such that $q$ is $<_{1}$-covered by
$p$\ \ \ \ \footnote{\textit{Proof.} Assume the contrary. Thus, there exist
$p\in P\cup\left\{  f\right\}  $ and $q\in Q\setminus\left\{  f\right\}  $
such that $q$ is $<_{1}$-covered by $p$. Consider such $p$ and $q$.
\par
We have $f\notin P$ and thus $f\in E\setminus P=Q$ (since $P\cap
Q=\varnothing$ and $P\cup Q=E$).
\par
We know that $q$ is $<_{1}$-covered by $p$, and thus we have $q<_{1}p$. Also,
$q\in Q\setminus\left\{  f\right\}  \subseteq Q$. Hence, if we had $p\in P$,
then we would obtain a contradiction to
\eqref{pf.thm.antipode.Gammaw.Zwd.pf.Adm}. Hence, we cannot have $p\in P$.
Therefore, $p=f$ (since $p\in P\cup\left\{  f\right\}  $ but not $p\in P$).
Hence, $q<_{1}p=f$. Therefore, $\pi\left(  q\right)  \leq\pi\left(  f\right)
$ (since $q\in Q$ and $f\in Q$, and since $\pi\mid_{Q}$ is a $\left(
Q,<_{1},<_{2}\right)  $-partition). Thus, we cannot have $q<_{2}f$ (because if
we had $q<_{2}f$, then \eqref{pf.thm.antipode.Gammaw.Zwd.pf.min2} (applied to
$h=q$) would show that $\pi\left(  f\right)  <\pi\left(  q\right)  $, which
would contradict $\pi\left(  q\right)  \leq\pi\left(  f\right)  $).
\par
Now, recall that $q$ is $<_{1}$-covered by $p$. Hence, $q$ and $p$ are
$<_{2}$-comparable (since $E$ is tertispecial).
In other words, $q$ and $f$ are
$<_{2}$-comparable (since $p=f$). In other words, either $q<_{2}f$ or $q=f$ or
$f<_{2}q$. But we cannot have $q<_{2}f$ (as we have just shown), and we cannot
have $q=f$ either (since $q<_{1}f$). Thus, we must have $f<_{2}q$.
\par
From $q<_{1}f$ and $f<_{2}q$, we conclude that $\pi\left(  q\right)
<\pi\left(  f\right)  $ (since $\pi\mid_{Q}$ is a
$\left(  Q,<_{1},<_{2}\right)  $-partition, and since
$q\in Q$ and $f\in Q$). But
\eqref{pf.thm.antipode.Gammaw.Zwd.pf.min} (applied to $h=q$) shows that
$\pi\left(  f\right)  \leq\pi\left(  q\right)  $. Hence, $\pi\left(  q\right)
<\pi\left(  f\right)  \leq\pi\left(  q\right)  $, a contradiction. Thus, our
assumption was wrong, qed.}. Hence, Lemma \ref{lem.admissible.cover} (applied
to $P\cup\left\{  f\right\}  $ and $Q\setminus\left\{  f\right\}  $ instead of
$P$ and $Q$) shows that $\left(  P\cup\left\{  f\right\}  ,Q\setminus\left\{
f\right\}  \right)  \in\operatorname{Adm}{ \EE }$.

Furthermore, $\pi\mid_{Q}$ is a $\left(  Q,<_{1},<_{2}\right)  $-partition,
and therefore $\pi\mid_{Q\setminus\left\{  f\right\}  }$ is a $\left(
Q\setminus\left\{  f\right\}  ,<_{1},<_{2}\right)  $-partition (since
$Q\setminus\left\{  f\right\}  \subseteq Q$).

Furthermore, $\pi\mid_{P\cup\left\{  f\right\}  }$ is a $\left(  P\cup\left\{
f\right\}  ,>_{1},<_{2}\right)  $-partition\footnote{\textit{Proof.} In order
to prove this, we need to verify the following two claims:
\par
\textit{Claim 1:} Every $a\in P\cup\left\{  f\right\}  $ and $b\in
P\cup\left\{  f\right\}  $ satisfying $a>_{1}b$ satisfy $\pi\left(  a\right)
\leq\pi\left(  b\right)  $;
\par
\textit{Claim 2:} Every $a\in P\cup\left\{  f\right\}  $ and $b\in
P\cup\left\{  f\right\}  $ satisfying $a>_{1}b$ and $b<_{2}a$ satisfy
$\pi\left(  a\right)  <\pi\left(  b\right)  $.
\par
\textit{Proof of Claim 1:} Let $a\in P\cup\left\{  f\right\}  $ and $b\in
P\cup\left\{  f\right\}  $ be such that $a>_{1}b$. We need to prove that
$\pi\left(  a\right)  \leq\pi\left(  b\right)  $. If $a=f$, then this follows
immediately from \eqref{pf.thm.antipode.Gammaw.Zwd.pf.min} (applied to $h=b$).
Hence, we WLOG assume that $a\neq f$. Thus, $a\in P$ (since $a\in
P\cup\left\{  f\right\}  $). Now, if $b\in Q$, then $b<_{1}a$ contradicts
\eqref{pf.thm.antipode.Gammaw.Zwd.pf.Adm} (applied to $p=a$ and $q=b$). Hence,
we cannot have $b\in Q$. Therefore, $b\in E\setminus Q=P$ (since $P\cap
Q=\varnothing$ and $P\cup Q=E$). Thus, $\pi\left(  a\right)  \leq\pi\left(
b\right)  $ follows immediately from the fact that $\pi\mid_{P}$ is a $\left(
P,>_{1},<_{2}\right)  $-partition (since $a \in P$ and $b \in P$).
This proves Claim 1.
\par
\textit{Proof of Claim 2:} Let $a\in P\cup\left\{  f\right\}  $ and $b\in
P\cup\left\{  f\right\}  $ be such that $a>_{1}b$ and $b<_{2}a$. We need to
prove that $\pi\left(  a\right)  <\pi\left(  b\right)  $. If $a=f$, then this
follows immediately from \eqref{pf.thm.antipode.Gammaw.Zwd.pf.min2} (applied
to $h=b$). Hence, we WLOG assume that $a\neq f$. Thus, $a\in P$ (since $a\in
P\cup\left\{  f\right\}  $). Now, if $b\in Q$, then $b<_{1}a$ contradicts
\eqref{pf.thm.antipode.Gammaw.Zwd.pf.Adm} (applied to $p=a$ and $q=b$). Hence,
we cannot have $b\in Q$. Therefore, $b\in E\setminus Q=P$ (since $P\cap
Q=\varnothing$ and $P\cup Q=E$). Thus, $\pi\left(  a\right)  <\pi\left(
b\right)  $ follows immediately from the fact that $\pi\mid_{P}$ is a $\left(
P,>_{1},<_{2}\right)  $-partition (since $a \in P$ and $b \in P$).
This proves Claim 2.
\par
Now, both Claim 1 and Claim 2 are proven, and we are done.}.

Altogether, we now know that $\left(  P\cup\left\{  f\right\}  ,Q\setminus
\left\{  f\right\}  \right)  \in\operatorname{Adm}{ \EE }$, that $\pi
\mid_{P\cup\left\{  f\right\}  }$ is a $\left(  P\cup\left\{  f\right\}
,>_{1},<_{2}\right)  $-partition, and that $\pi\mid_{Q\setminus\left\{
f\right\}  }$ is a $\left(  Q\setminus\left\{  f\right\}  ,<_{1},<_{2}\right)
$-partition. In other words, $\left(  P\cup\left\{  f\right\}  ,Q\setminus
\left\{  f\right\}  \right)  \in Z$ (by the definition of $Z$). Thus,
\begin{align*}
\begin{cases}
\left(  P\cup\left\{  f\right\}  ,Q\setminus\left\{  f\right\}  \right)  , &
\text{if }f\notin P;\\
\left(  P\setminus\left\{  f\right\}  ,Q\cup\left\{  f\right\}  \right)  , &
\text{if }f\in P
\end{cases}
& =\left(  P\cup\left\{  f\right\}  ,Q\setminus\left\{  f\right\}  \right)
\ \ \ \ \ \ \ \ \ \ \left(  \text{since }f\notin P\right)  \\
& \in Z.
\end{align*}
Hence, \eqref{pf.thm.antipode.Gammaw.Zwd} is proven in Case 2.

We have now proven \eqref{pf.thm.antipode.Gammaw.Zwd} in both Cases 1 and 2.
Thus, \eqref{pf.thm.antipode.Gammaw.Zwd} always holds. In other words, the map
$T$ is well-defined.

What the map $T$ does to a pair $\left(  P,Q\right)  \in Z$ can be described
as moving the element $f$ from the set where it resides (either $P$ or $Q$) to
the other set. Clearly, doing this twice gives us the original pair back.
Hence, the map $T$ is an involution. Furthermore, for any $\left(  P,Q\right)
\in Z$, if we write $T\left(  \left(  P,Q\right)  \right)  $ in the form
$\left(  P^{\prime},Q^{\prime}\right)  $, then $\left(  -1\right)
^{\left\vert P^{\prime}\right\vert }=-\left(  -1\right)  ^{\left\vert
P\right\vert }$ (because $P^{\prime}=
\begin{cases}
P\cup\left\{  f\right\}  , & \text{if }f\notin P;\\
P\setminus\left\{  f\right\}  , & \text{if }f\in P
\end{cases}
$). As we have already explained, this proves
\eqref{pf.thm.antipode.Gammaw.signrev}. And this, in turn, completes the
induction step of the proof of Theorem~\ref{thm.antipode.Gammaw}.
\end{proof}

\section{Proof of Theorem~\ref{thm.antipode.GammawG}}
\label{sect.proofG}

Before we begin proving Theorem~\ref{thm.antipode.GammawG}, we state a
criterion for $\EE$-partitions that is less wasteful (in the sense that
it requires fewer verifications) than the definition:

\begin{lemma}
\label{lem.Epartition.cover}
Let $\EE = \left(E, <_1, <_2\right)$ be a tertispecial double poset.
Let $\phi : E \to \left\{1, 2, 3, \ldots\right\}$ be a map. Assume
that the following two conditions hold:

\begin{itemize}

\item \textit{Condition 1:} If $e \in E$ and $f \in E$ are such that
$e$ is $<_1$-covered by $f$, and if we have $e <_2 f$, then
$\phi\left(e\right) \leq \phi\left(f\right)$.

\item \textit{Condition 2:} If $e \in E$ and $f \in E$ are such that
$e$ is $<_1$-covered by $f$, and if we have $f <_2 e$, then
$\phi\left(e\right) < \phi\left(f\right)$.

\end{itemize}

Then, $\phi$ is an $\EE$-partition.
\end{lemma}

\begin{proof}[Proof of Lemma~\ref{lem.Epartition.cover}.]
For any $a \in E$ and $b \in E$, we define a subset
$\left[a, b\right]$ of $E$ as in the proof of
Lemma~\ref{lem.admissible.cover}.

We need to show that $\phi$ is an $\EE$-partition. In other words,
we need to prove the following two claims:

\textit{Claim 1:} Every $e \in E$ and $f \in E$ satisfying
$e <_1 f$ satisfy $\phi\left(e\right) \leq \phi\left(f\right)$.

\textit{Claim 2:} Every $e \in E$ and $f \in E$ satisfying
$e <_1 f$ and $f <_2 e$ satisfy
$\phi\left(e\right) < \phi\left(f\right)$.

\textit{Proof of Claim 1:} Assume the contrary. Thus, there
exists a pair $\left(e, f\right) \in E \times E$ satisfying
$e <_1 f$ but not $\phi\left(e\right) \leq \phi\left(f\right)$.
Such a pair will be called a \textit{malrelation}. Fix a
malrelation $\left(u, v\right)$ for which the value
$\left|\left[u, v\right]\right|$ is minimum (such a
$\left(u, v\right)$ exists, since there exists a malrelation).
Thus, $u \in E$ and $v \in E$ and $u <_1 v$ but not
$\phi\left(u\right) \leq \phi\left(v\right)$.

If $u$ was $<_1$-covered by $v$, then we would obtain
$\phi\left(u\right) \leq \phi\left(v\right)$
\ \ \ \ \footnote{\textit{Proof.} Assume that $u$ is
$<_1$-covered by $v$. Thus, $u$ and $v$ are $<_2$-comparable
(since the poset $\EE$ is tertispecial). In other words,
we have either $u <_2 v$ or $u = v$ or $v <_2 u$. In the
first of these three cases, we obtain
$\phi\left(u\right) \leq \phi\left(v\right)$ by applying
Condition 1 to $e = u$ and $f = v$. In the third of these
cases, we obtain
$\phi\left(u\right) < \phi\left(v\right)$ (and thus
$\phi\left(u\right) \leq \phi\left(v\right)$) 
by applying Condition 2 to $e = u$ and $f = v$. The second
of these cases cannot happen because $u <_1 v$. Thus, we
always have $\phi\left(u\right) \leq \phi\left(v\right)$,
qed.}, which would
contradict the assumption that we do not have
$\phi\left(u\right) \leq \phi\left(v\right)$. Hence, $u$ is not
$<_1$-covered by $v$. Consequently, there exists a $w \in E$
such that $u <_1 w <_1 v$ (since $u <_1 v$). Consider this
$w$. Applying \eqref{pf.lem.admissible.cover.1} to $a = u$,
$c = w$ and $b = v$, we see that both numbers
$\left|\left[u, w\right]\right|$ and
$\left|\left[w, v\right]\right|$ are smaller than
$\left|\left[u, v\right]\right|$, and therefore neither
$\left(u, w\right)$ nor $\left(w, v\right)$ is a malrelation
(since we picked $\left(u, v\right)$ to be a malrelation with
minimum $\left|\left[u, v\right]\right|$). Therefore, we have
$\phi\left(u\right) \leq \phi\left(w\right)$ and
$\phi\left(w\right) \leq \phi\left(v\right)$ (since $u <_1 w$
and $w <_1 v$). Combining these two inequalities, we obtain
$\phi\left(u\right) \leq \phi\left(v\right)$. This contradicts
the assumption that we do not have
$\phi\left(u\right) \leq \phi\left(v\right)$. This contradiction
concludes the proof of Claim 1.

Instead of Claim 2, we shall prove the following stronger claim:

\textit{Claim 3:} Every $e \in E$ and $f \in E$ satisfying
$e <_1 f$ and not $e <_2 f$ satisfy
$\phi\left(e\right) < \phi\left(f\right)$.

\textit{Proof of Claim 3:} Assume the contrary. Thus, there
exists a pair $\left(e, f\right) \in E \times E$ satisfying
$e <_1 f$ and not $e <_2 f$ but not
$\phi\left(e\right) < \phi\left(f\right)$.
Such a pair will be called a \textit{malrelation}. Fix a
malrelation $\left(u, v\right)$ for which the value
$\left|\left[u, v\right]\right|$ is minimum (such a
$\left(u, v\right)$ exists, since there exists a malrelation).
Thus, $u \in E$ and $v \in E$ and $u <_1 v$ and not $u <_2 v$
but not $\phi\left(u\right) < \phi\left(v\right)$.

If $u$ was $<_1$-covered by $v$, then we would obtain
$\phi\left(u\right) < \phi\left(v\right)$
easily\footnote{\textit{Proof.} Assume that $u$ is
$<_1$-covered by $v$. Thus, $u$ and $v$ are $<_2$-comparable
(since the poset $\EE$ is tertispecial). In other words,
we have either $u <_2 v$ or $u = v$ or $v <_2 u$. Since
neither $u <_2 v$ nor $u = v$ can hold (indeed, $u <_2 v$
is ruled out by assumption, whereas $u = v$ is ruled out by
$u <_1 v$), we thus have $v <_2 u$. Therefore,
$\phi\left(u\right) < \phi\left(v\right)$
by Condition 2 (applied to $e = u$ and $f = v$), qed.}, which
would contradict the assumption that we do not have
$\phi\left(u\right) < \phi\left(v\right)$. Hence, $u$ is not
$<_1$-covered by $v$. Consequently, there exists a $w \in E$
such that $u <_1 w <_1 v$ (since $u <_1 v$). Consider this
$w$. Applying \eqref{pf.lem.admissible.cover.1} to $a = u$,
$c = w$ and $b = v$, we see that both numbers
$\left|\left[u, w\right]\right|$ and
$\left|\left[w, v\right]\right|$ are smaller than
$\left|\left[u, v\right]\right|$, and therefore neither
$\left(u, w\right)$ nor $\left(w, v\right)$ is a malrelation
(since we picked $\left(u, v\right)$ to be a malrelation with
minimum $\left|\left[u, v\right]\right|$).

But $\phi\left(v\right) \leq \phi\left(u\right)$ (since we do
not have $\phi\left(u\right) < \phi\left(v\right)$). On the
other hand, $u <_1 w$ and therefore $\phi\left(u\right) \leq
\phi\left(w\right)$ (by Claim 1). Furthermore, $w <_1 v$ and
thus $\phi\left(w\right) \leq \phi\left(v\right)$ (by Claim 1).
The chain of inequalities
$\phi\left(v\right) \leq \phi\left(u\right)
\leq \phi\left(w\right) \leq \phi\left(v\right)$ ends with
the same term that it begins with; therefore, it must be a chain
of equalities. In other words, we have
$\phi\left(v\right) = \phi\left(u\right)
= \phi\left(w\right) = \phi\left(v\right)$.

Now, using $\phi\left(w\right) = \phi\left(v\right)$, we can
see that $w <_2 v$\ \ \ \ \footnote{\textit{Proof.} Assume
the contrary. Thus, we do not have $w <_2 v$. But
$\phi\left(w\right) = \phi\left(v\right)$ shows that we do not
have $\phi\left(w\right) < \phi\left(v\right)$. Hence,
$\left(w, v\right)$ is a malrelation (since $w <_1 v$ and not
$w <_2 v$ but not $\phi\left(w\right) < \phi\left(v\right)$).
This contradicts the fact that $\left(w, v\right)$ is not
a malrelation. This contradiction completes the proof.}.
The same argument (applied to $u$ and $w$ instead of $w$ and
$v$) shows that $u <_2 w$. Thus, $u <_2 w <_2 v$, which
contradicts the fact that we do not have $u <_2 v$. This
contradiction proves Claim 3.

\textit{Proof of Claim 2:} The condition ``$f <_2 e$'' is stronger
than ``not $e <_2 f$''. Thus, Claim 2 follows from Claim 3.

Claims 1 and 2 are now both proven, and so
Lemma~\ref{lem.Epartition.cover} follows.
\end{proof}

\begin{proof}
[Proof of Lemma \ref{lem.coeven.all-one}.] Consider the following three
logical statements:

\textit{Statement 1:} The orbit $O$ is $E$-coeven.

\textit{Statement 2:} All elements of $O$ are $E$-coeven.

\textit{Statement 3:} At least one element of $O$ is $E$-coeven.

Statements 1 and 2 are equivalent (according to the definition of when an
orbit is $E$-coeven). Our goal is to prove that Statements 1 and 3 are
equivalent (because this is precisely what Lemma \ref{lem.coeven.all-one}
says). Thus, it clearly suffices to show that Statements 2 and 3 are
equivalent. Since Statement 2 obviously implies Statement 3, we therefore only
need to show that Statement 3 implies Statement 2. Thus, assume that Statement
3 holds. We need to prove that Statement 2 holds.

There exists at least one $E$-coeven $\phi\in O$ (because we assumed that
Statement 3 holds). Consider this $\phi$. Now, let $\pi\in O$ be arbitrary. We
shall show that $\pi$ is $E$-coeven.

We know that $\phi$ is $E$-coeven. In other words,
\begin{equation}
\text{every }g\in G\text{ satisfying }g\phi=\phi\text{ is }E\text{-even.}%
\label{pf.lem.coeven-all-one.1}%
\end{equation}


Now, let $g\in G$ be such that $g\pi=\pi$. Since $\phi$ belongs to the
$G$-orbit $O$, we have $O=G\phi$. Now, $\pi\in O=G\phi$. In other words, there
exists some $h\in G$ such that $\pi=h\phi$. Consider this $h$. We have
$g\pi=\pi$. Since $\pi=h\phi$, this rewrites as $gh\phi=h\phi$. In other
words, $h^{-1}gh\phi=\phi$. Thus, (\ref{pf.lem.coeven-all-one.1}) (applied to
$h^{-1}gh$ instead of $g$) shows that $h^{-1}gh$ is $E$-even. In other words,%
\begin{equation}
\text{the action of }h^{-1}gh\text{ on }E\text{ is an even permutation of
}E\text{.}\label{pf.lem.coeven-all-one.2}%
\end{equation}


Now, let $\varepsilon$ be the group homomorphism from $G$ to
$\operatorname*{Aut}E$ which describes the $G$-action on $E$. Then,
$\varepsilon\left(  h^{-1}gh\right)  $ is the action of $h^{-1}gh$ on $E$, and
thus is an even permutation of $E$ (by (\ref{pf.lem.coeven-all-one.2})).

But since $\varepsilon$ is a group homomorphism, we have $\varepsilon\left(
h^{-1}gh\right)  =\varepsilon\left(  h\right)  ^{-1}\varepsilon\left(
g\right)  \varepsilon\left(  h\right)  $. Thus, the permutations
$\varepsilon\left(  h^{-1}gh\right)  $ and $\varepsilon\left(  g\right)  $ of
$E$ are conjugate. Since the permutation $\varepsilon\left(  h^{-1}gh\right)
$ is even, this shows that the permutation $\varepsilon\left(  g\right)  $ is
even. In other words, the action of $g$ on $E$ is an even permutation of $E$.
In other words, $g$ is $E$-even.

Now, let us forget that we fixed $g$. We thus have shown that every $g\in G$
satisfying $g\pi=\pi$ is $E$-even. In other words, $\pi$ is $E$-coeven.

Let us now forget that we fixed $\pi$. Thus, we have proven that every $\pi\in
O$ is $E$-coeven. In other words, Statement 2 holds. We have thus shown that
Statement 3 implies Statement 2. Consequently, Statements 2 and 3 are
equivalent, and so the proof of Lemma \ref{lem.coeven.all-one} is complete.
\end{proof}

Next, we will show three simple properties of posets on which groups act.

\begin{proposition}
\label{prop.G-poset.quot.poset}
Let $E$ be a set. Let $<_1$ be a strict partial order relation on $E$.
Let $G$ be a finite group which acts on $E$. Assume that $G$ preserves
the relation $<_1$.

Let $g \in G$. Let $E^{g}$ be the set of all orbits under the action of
$g$ on $E$. Define a binary relation $<_{1}^{g}$ on $E^{g}$ by
\[
\left(  u<_{1}^{g}v\right)  \Longleftrightarrow\left(  \text{there exist }a\in
u\text{ and }b\in v\text{ with }a<_{1}b\right)  .
\]
Then, $<_1^g$ is a strict partial order relation.
\end{proposition}

Proposition~\ref{prop.G-poset.quot.poset} is precisely \cite[Lemma 2.4]{Joch},
but let us outline the proof for the sake of completeness:

\begin{proof}[Proof of Proposition~\ref{prop.G-poset.quot.poset}.]
Let us first show that the relation $<_{1}^{g}$ is irreflexive. Indeed, assume
the contrary. Thus, there exists a $u\in E^{g}$ such that $u<_{1}^{g}u$.
Consider this $u$. Since $u<_{1}^{g}u$, there exist $a\in u$ and $b\in u$ with
$a<_{1}b$. Consider these $a$ and $b$. There exists a $k\in\mathbb{N}$ such
that $b=g^{k}a$ (since $a$ and $b$ both lie in one and the same $g$-orbit
$u$). Consider this $k$.
\par
The $g$-orbit $u$ of $a$ is finite (since $g$ is finite).
Thus, there exists a positive integer $n$
such that $g^{n}a=a$. Consider this $n$. Notice that $g^{np}a=\left(
g^{n}\right)  ^{p}a=a$ for every $p\in\mathbb{N}$ (since $g^{n}a=a$).
\par
Now, $a<_{1}b=g^{k}a$. Since $G$ preserves the relation $<_{1}$, this shows
that $ha<_{1}hg^{k}a$ for every $h\in G$. Thus, $g^{\ell k}a<_{1}g^{\ell
k}g^{k}a$ for every $\ell\in\mathbb{N}$. Hence, $g^{\ell k}a<_{1}g^{\ell
k}g^{k}a=g^{\left(  \ell+1\right)  k}a$ for every $\ell\in\mathbb{N}$.
Consequently, $g^{0k}a<_{1}g^{1k}a<_{1}g^{2k}a<_{1}\cdots<_{1}g^{nk}a$. Thus,
$g^{0k}a<_{1}g^{nk}a=a$ (since $g^{np}a=a$ for every $p\in\mathbb{N}$), which
contradicts $g^{0k}a=1_Ga=a$. This contradiction proves that our assumption was
wrong. Hence, the relation $<_{1}^{g}$ is irreflexive.
\par
Let us next show that the relation $<_{1}^{g}$ is transitive. Indeed, let $u$,
$v$ and $w$ be three elements of $E^{g}$ such that $u<_{1}^{g}v$ and
$v<_{1}^{g}w$. We must prove that $u<_{1}^{g}w$.
\par
There exist $a\in u$ and $b\in v$ with $a<_{1}b$ (since $u<_{1}^{g}v$).
Consider these $a$ and $b$.
\par
There exist $a^{\prime}\in v$ and $b^{\prime}\in w$ with $a^{\prime}
<_{1}b^{\prime}$ (since $v<_{1}^{g}w$). Consider these $a^{\prime}$ and
$b^{\prime}$.
\par
The elements $b$ and $a^{\prime}$ lie in one and the same $g$-orbit (namely,
in $v$). Hence, there exists some $k\in\mathbb{N}$ such that $a^{\prime}
=g^{k}b$. Consider this $k$. We have $a<_{1}b$ and thus $g^{k}a<_{1}g^{k}b$
(since $G$ preserves the relation $<_{1}$). Hence,
$g^{k}a<_{1}g^{k}b=a^{\prime}<_{1}b^{\prime}$.
Since $g^{k}a\in u$ (because $a\in u$) and
$b^{\prime}\in w$, this shows that $u<_{1}^{g}w$. We thus have proven that the
relation $<_{1}^{g}$ is transitive.
\par
Now, we know that the relation $<_{1}^{g}$ is irreflexive and transitive, and
thus also antisymmetric (since every irreflexive and transitive binary
relation is antisymmetric). In other words, $<_{1}^{g}$ is a strict partial
order relation. This proves Proposition~\ref{prop.G-poset.quot.poset}.
\end{proof}

\begin{remark}
Proposition~\ref{prop.G-poset.quot.poset} can be generalized:
Let $E$ be a set. Let $<_1$ be a strict partial order relation on $E$.
Let $G$ be a finite group which acts on $E$. Assume that $G$ preserves
the relation $<_1$. Let $H$ be a subgroup of $G$.
Let $E^H$ be the set of all orbits under the action of
$H$ on $E$. Define a binary relation $<_{1}^H$ on $E^H$ by
\[
\left(  u<_{1}^H v\right)  \Longleftrightarrow\left(  \text{there exist }a\in
u\text{ and }b\in v\text{ with }a<_{1}b\right)  .
\]
Then, $<_1^H$ is a strict partial order relation.

This result (whose proof is quite similar to that of
Proposition~\ref{prop.G-poset.quot.poset}) implicitly appears in
\cite[p. 30]{Stanley-Peck}.
\end{remark}

\begin{proposition}
\label{prop.G-poset.quot.double}
Let $\EE = \left(E, <_1, <_2\right)$ be a tertispecial double poset.
Let $G$ be a finite group which acts on $E$. Assume that $G$ preserves
both relations $<_1$ and $<_2$.

Let $g \in G$. Let $E^{g}$ be the set of all orbits under the action of
$g$ on $E$. Define a binary relation $<_{1}^{g}$ on $E^{g}$ by
\[
\left(  u<_{1}^{g}v\right)  \Longleftrightarrow\left(  \text{there exist }a\in
u\text{ and }b\in v\text{ with }a<_{1}b\right)  .
\]
Define a binary relation $<_{2}^{g}$ on $E^{g}$ by
\[
\left(  u<_{2}^{g}v\right)  \Longleftrightarrow\left(  \text{there exist }a\in
u\text{ and }b\in v\text{ with }a<_{2}b\right)  .
\]
Let $\EE^g$ be the triple $\left(E^g, <_1^g, <_2^g\right)$. Then,
$\EE^g$ is a tertispecial double poset.
\end{proposition}

\begin{proof}[Proof of Proposition~\ref{prop.G-poset.quot.double}.]
Both relations $<_1$ and $<_2$ are strict partial order relations
(since $\EE$ is a double poset).
Proposition~\ref{prop.G-poset.quot.poset} shows that $<_{1}^{g}$ is a
strict partial order relation.
Proposition~\ref{prop.G-poset.quot.poset} (applied to $<_2$ and
$<_2^g$ instead of $<_1$ and $<_1^g$) shows that $<_{2}^{g}$ is a
strict partial order relation. Thus, $\EE^g
= \left(E^g, <_1^g, <_2^g\right)$ is a double poset. It remains
to show that this double poset $\EE^g$ is tertispecial.

Let $u$ and $v$ be two elements of
$E^{g}$ such that $u$ is $<_{1}^{g}$-covered by $v$. We shall prove that $u$
and $v$ are $<_{2}^{g}$-comparable.

We have $u<_{1}^{g}v$ (since $u$ is $<_{1}^{g}$-covered by $v$). In other
words, there exist $a\in u$ and $b\in v$ with $a<_{1}b$. Consider these $a$
and $b$.

If there was a $c\in E$ satisfying $a<_{1}c<_{1}b$, then we would have
$u<_{1}^{g}w<_{1}^{g}v$ with $w$ being the $g$-orbit of $c$, and this would
contradict the condition that $u$ is $<_{1}^{g}$-covered by $v$. Hence, no
such $c$ can exist. In other words, $a$ is $<_{1}$-covered by $b$. Thus, $a$
and $b$ are $<_{2}$-comparable (since the double poset $ \EE $ is
tertispecial). Consequently, $u$ and $v$ are $<_{2}^{g}$-comparable.

Now, let us forget that we fixed $u$ and $v$. We thus have shown that if
$u$ and $v$ are two elements of $E^{g}$ such that $u$ is $<_{1}^{g}$-covered
by $v$, then $u$ and $v$ are $<_{2}^{g}$-comparable. In other words, the
double poset $\EE^g = \left(E^g, <_1^g, <_2^g\right)$ is tertispecial. This
proves Proposition~\ref{prop.G-poset.quot.double}.
\end{proof}

\begin{proposition}
\label{prop.G-poset.quot.Phi}
Let $\EE = \left(E, <_1, <_2\right)$ be a tertispecial double poset.
Let $G$ be a finite group which acts on $E$. Assume that $G$ preserves
both relations $<_1$ and $<_2$.

Let $g \in G$. Define the set $E^g$, the relations $<_1^g$ and $<_2^g$
and the triple $\EE^g$ as in Proposition~\ref{prop.G-poset.quot.double}.
Thus, $\EE^g$ is a tertispecial double poset (by
Proposition~\ref{prop.G-poset.quot.double}).

There is a bijection $\Phi$ between

\begin{itemize}
\item the maps $\pi:E\rightarrow\left\{  1,2,3,\ldots\right\}  $ satisfying
$g\pi=\pi$
\end{itemize}

and

\begin{itemize}
\item the maps $\overline{\pi}:E^{g}\rightarrow\left\{  1,2,3,\ldots\right\}
$.
\end{itemize}

Namely, this bijection $\Phi$ sends any map
$\pi:E\rightarrow\left\{  1,2,3,\ldots \right\}  $ satisfying $g\pi=\pi$
to the map $\overline{\pi}:E^{g}
\rightarrow\left\{  1,2,3,\ldots\right\}  $ defined by
\[
\overline{\pi}\left( u \right)
= \pi\left( a \right)
\qquad\text{for every } u \in E^{g} \text{ and } a \in u.
\]
(The well-definedness of this map $\overline{\pi}$ is easy to see:
Indeed, from $g\pi=\pi$, we can conclude that any two elements
$a_1$ and $a_2$ of a given $g$-orbit $u$ satisfy
$\pi\left(a_1\right) = \pi\left(a_2\right)$.)

Consider this bijection $\Phi$. Let
$\pi:E\rightarrow\left\{  1,2,3,\ldots\right\}  $ be a
map satisfying $g\pi=\pi$.

\begin{enumerate}
\item[(a)] If $\pi$ is an $\EE$-partition, then $\Phi\left(
\pi\right)  $ is an ${\mathbf{E}}^{g}$-partition.

\item[(b)] If $\Phi\left(  \pi\right)  $ is an
$\mathbf{E}^{g}$-partition, then $\pi$ is an $\EE$-partition.

\item[(c)] Let $w : E \to \left\{1,2,3,\ldots\right\}$ be
map. Define a map $w^{g}:E^{g}\rightarrow
\left\{1,2,3,\ldots\right\}$
by
\[
w^{g}\left(  u\right)  =\sum\limits_{a\in u}w\left(
a\right)  \qquad \qquad \text{ for every } u \in E^g .
\]
Then, $\mathbf{x}_{\Phi\left(  \pi\right)  ,w^{g}}
=\mathbf{x}_{\pi,w}$.

\end{enumerate}
\end{proposition}

\begin{proof}[Proof of Proposition~\ref{prop.G-poset.quot.Phi} (sketched).]
The definition of $\Phi$ shows that
\begin{equation}
\left(\Phi\left(\pi\right)\right) \left( u \right)
= \pi\left( a \right)
\qquad\text{for every } u \in E^{g} \text{ and } a \in u.
\label{pf.prop.G-poset.quot.phi.main}
\end{equation}

(a) Assume that $\pi$ is an $\mathbf{E}$-partition. We
want to show that $\Phi\left(  \pi\right)  $ is an
${\mathbf{E}}^{g}$-partition. In order to do so, we can
use Lemma \ref{lem.Epartition.cover}
(applied to $\mathbf{E}^{g}$, $\left(  E^{g},<_{1}^{g},<_{2}^{g}\right)  $ and
$\Phi\left(  \pi\right)  $ instead of $\mathbf{E}$,
$\left(  E,<_{1},<_{2}\right)  $ and $\phi$); we only need to
check the following two conditions:

\textit{Condition 1:} If $e\in E^{g}$ and $f\in E^{g}$ are such that $e$ is
$<_{1}^{g}$-covered by $f$, and if we have $e<_{2}^{g}f$, then $\left(
\Phi\left(  \pi\right)  \right)  \left(  e\right)  \leq\left(  \Phi\left(
\pi\right)  \right)  \left(  f\right)  $.

\textit{Condition 2:} If $e\in E^{g}$ and $f\in E^{g}$ are such that $e$ is
$<_{1}^{g}$-covered by $f$, and if we have $f<_{2}^{g}e$, then $\left(
\Phi\left(  \pi\right)  \right)  \left(  e\right)  <\left(  \Phi\left(
\pi\right)  \right)  \left(  f\right)  $.

\textit{Proof of Condition 1:} Let $e\in E^{g}$ and $f\in E^{g}$ be such that
$e$ is $<_{1}^{g}$-covered by $f$. Assume that we have $e<_{2}^{g}f$.

We have $e<_{1}^{g}f$ (because $e$ is $<_{1}^{g}$-covered by $f$). In other
words, there exist $a\in e$ and $b\in f$ satisfying $a<_{1}b$. Consider these
$a$ and $b$. Since $\pi$ is an $\mathbf{E}$-partition, we have $\pi\left(
a\right)  \leq\pi\left(  b\right)  $ (since $a<_{1}b$). But the definition of
$\Phi\left(  \pi\right)  $ shows that $\left(  \Phi\left(  \pi\right)
\right)  \left(  e\right)  =\pi\left(  a\right)  $ (since $a\in e$) and
$\left(  \Phi\left(  \pi\right)  \right)  \left(  f\right)  =\pi\left(
b\right)  $ (since $b\in f$). Thus, $\left(  \Phi\left(  \pi\right)  \right)
\left(  e\right)  =\pi\left(  a\right)  \leq\pi\left(  b\right)  =\left(
\Phi\left(  \pi\right)  \right)  \left(  f\right)  $. Hence, Condition 1 is proven.

\textit{Proof of Condition 2:} Let $e\in E^{g}$ and $f\in E^{g}$ be such that
$e$ is $<_{1}^{g}$-covered by $f$. Assume that we have $f<_{2}^{g}e$.

We have $e<_{1}^{g}f$ (because $e$ is $<_{1}^{g}$-covered by $f$). In other
words, there exist $a\in e$ and $b\in f$ satisfying $a<_{1}b$. Consider these
$a$ and $b$.

If there was a $c\in E$ satisfying $a<_{1}c<_{1}b$, then the $g$-orbit $w$ of
this $c$ would satisfy $e<_{1}^{g}w<_{1}^{g}f$, which would contradict the
fact that $e$ is $<_{1}^{g}$-covered by $f$. Hence, there exists no such $c$.
In other words, $a$ is $<_{1}$-covered by $b$ (since $a<_{1}b$). Therefore,
$a$ and $b$ are $<_{2}$-comparable (since $\mathbf{E}$ is tertispecial). In
other words, we have either $a<_{2}b$ or $a=b$ or $b<_{2}a$. Since $a<_{2}b$
is impossible (because if we had $a<_{2}b$, then we would have $e<_{2}^{g}f$
(since $a\in e$ and $b\in f$), which would contradict $f<_{2}^{g}e$ (since
$<_{2}^{g}$ is a strict partial order relation)), and since $a=b$ is
impossible (because $a<_{1}b$), we therefore must have $b<_{2}a$. But since
$\pi$ is an $\mathbf{E}$-partition, we have $\pi\left(  a\right)  <\pi\left(
b\right)  $ (since $a<_{1}b$ and $b<_{2}a$). But the definition of
$\Phi\left(  \pi\right)  $ shows that $\left(  \Phi\left(  \pi\right)
\right)  \left(  e\right)  =\pi\left(  a\right)  $ (since $a\in e$) and
$\left(  \Phi\left(  \pi\right)  \right)  \left(  f\right)  =\pi\left(
b\right)  $ (since $b\in f$). Thus, $\left(  \Phi\left(  \pi\right)  \right)
\left(  e\right)  =\pi\left(  a\right)  <\pi\left(  b\right)  =\left(
\Phi\left(  \pi\right)  \right)  \left(  f\right)  $. Hence, Condition 2 is proven.

Thus, Condition 1 and Condition 2 are proven. Hence,
Proposition~\ref{prop.G-poset.quot.Phi} (a) is proven.

(b) Assume that $\Phi\left(  \pi\right)  $ is an
$\mathbf{E}^{g}$-partition. We want to show that $\pi$ is an
$\EE$-partition. In order to do so, we can use
Lemma \ref{lem.Epartition.cover}
(applied to $\phi=\pi$); we only need to check the following two conditions:

\textit{Condition 1:} If $e\in E$ and $f\in E$ are such that $e$ is
$<_{1}$-covered by $f$, and if we have $e<_{2}f$, then
$\pi\left(  e\right)  \leq \pi\left(  f\right)  $.

\textit{Condition 2:} If $e\in E$ and $f\in E$ are such that $e$ is
$<_{1}$-covered by $f$, and if we have $f<_{2}e$, then
$\pi\left(  e\right) <\pi\left(  f\right)  $.

\textit{Proof of Condition 1:} Let $e\in E$ and $f\in E$ be such that $e$ is
$<_{1}$-covered by $f$. Assume that we have $e<_{2}f$.

We have $e<_{1}f$ (since $e$ is $<_{1}$-covered by $f$). Let $u$ and $v$ be
the $g$-orbits of $e$ and $f$, respectively. Thus, $u$ and $v$ belong to
$E^{g}$, and satisfy $u<_{1}^g v$ (since $e<_{1}f$). Hence, $\left(  \Phi\left(
\pi\right)  \right)  \left(  u\right)  \leq\left(  \Phi\left(  \pi\right)
\right)  \left(  v\right)  $ (since $\Phi\left(  \pi\right)  $ is an
$\mathbf{E}^{g}$-partition). But the definition of $\Phi\left(  \pi\right)  $
shows that $\left(  \Phi\left(  \pi\right)  \right)  \left(  u\right)
=\pi\left(  e\right)  $ (since $e\in u$) and $\left(  \Phi\left(  \pi\right)
\right)  \left(  v\right)  =\pi\left(  f\right)  $ (since $f\in v$). Thus,
$\pi\left(  e\right)  =\left(  \Phi\left(  \pi\right)  \right)  \left(
u\right)  \leq\left(  \Phi\left(  \pi\right)  \right)  \left(  v\right)
=\pi\left(  f\right)  $. Hence, Condition 1 is proven.

\textit{Proof of Condition 2:} Let $e\in E$ and $f\in E$ be such that $e$ is
$<_{1}$-covered by $f$. Assume that we have $f<_{2}e$.

We have $e<_{1}f$ (since $e$ is $<_{1}$-covered by $f$). Let $u$ and $v$ be
the $g$-orbits of $e$ and $f$, respectively. Thus, $u$ and $v$ belong to
$E^{g}$, and satisfy $u<_{1}^g v$ (since $e<_{1}f$) and $v<_{2}^g u$ (since
$f<_{2}e$). Hence, $\left(  \Phi\left(  \pi\right)  \right)  \left(  u\right)
<\left(  \Phi\left(  \pi\right)  \right)  \left(  v\right)  $ (since
$\Phi\left(  \pi\right)  $ is an $\mathbf{E}^{g}$-partition). But the
definition of $\Phi\left(  \pi\right)  $ shows that $\left(  \Phi\left(
\pi\right)  \right)  \left(  u\right)  =\pi\left(  e\right)  $ (since $e\in
u$) and $\left(  \Phi\left(  \pi\right)  \right)  \left(  v\right)
=\pi\left(  f\right)  $ (since $f\in v$). Thus, $\pi\left(  e\right)  =\left(
\Phi\left(  \pi\right)  \right)  \left(  u\right)  <\left(  \Phi\left(
\pi\right)  \right)  \left(  v\right)  =\pi\left(  f\right)  $. Hence,
Condition 2 is proven.

Thus, Condition 1 and Condition 2 are proven. Hence,
Proposition~\ref{prop.G-poset.quot.Phi} (b) is proven.

(c) 
The definition of $\mathbf{x}_{\Phi\left(  \pi\right)  ,w^{g}}$ shows that%
\begin{align*}
\mathbf{x}_{\Phi\left(  \pi\right)  ,w^{g}}  & =\prod_{e\in E^{g}}x_{\left(
\Phi\left(  \pi\right)  \right)  \left(  e\right)  }^{w^{g}\left(  e\right)
}=\prod_{u\in E^{g}}\underbrace{x_{\left(  \Phi\left(  \pi\right)  \right)
\left(  u\right)  }^{w^{g}\left(  u\right)  }}_{\substack{=\prod_{a\in
u}x_{\left(  \Phi\left(  \pi\right)  \right)  \left(  u\right)  }^{w\left(
a\right)  }\\\text{(since }w^{g}\left(  u\right)  =\sum\limits_{a\in
u}w\left(  a\right)  \text{)}}}=\prod_{u\in E^{g}}\prod_{a\in u}%
\underbrace{x_{\left(  \Phi\left(  \pi\right)  \right)  \left(  u\right)
}^{w\left(  a\right)  }}_{\substack{=x_{\pi\left(  a\right)  }^{w\left(
a\right)  }\\\text{(by (\ref{pf.prop.G-poset.quot.phi.main}))}}}\\
& =\underbrace{\prod_{u\in E^{g}}\prod_{a\in u}}_{=\prod_{a\in E}}%
x_{\pi\left(  a\right)  }^{w\left(  a\right)  }=\prod_{a\in E}x_{\pi\left(
a\right)  }^{w\left(  a\right)  }=\prod_{e\in E}x_{\pi\left(  e\right)
}^{w\left(  e\right)  }=\mathbf{x}_{\pi,w}%
\end{align*}
(by the definition of $\mathbf{x}_{\pi,w}$). This proves Proposition
\ref{prop.G-poset.quot.Phi} (c).
\end{proof}

Our next lemma is a standard argument in P\'olya enumeration theory (compare
it with the proof of Burnside's lemma):

\begin{lemma}
\label{lem.burnside.sums} Let $G$ be a finite group. Let $F$ be a finite
$G$-set. Let $O$ be a $G$-orbit on $F$, and let $\pi\in O$.

\begin{enumerate}
\item[(a)] We have
\begin{equation}
\dfrac{1}{\left\vert O\right\vert }=\dfrac{1}{\left\vert G\right\vert }%
\sum_{\substack{g\in G;\\g\pi=\pi}}1.\label{eq.lem.burnside.sums.a}%
\end{equation}


\item[(b)] Let $E$ be a further finite $G$-set. For every $g\in G$, let
$\operatorname{sign}_{E}g$ denote the sign of the permutation of $E$ that
sends every $e\in E$ to $ge$. (Thus, $g\in G$ is $E$-even if and only if
$\operatorname*{sign}\nolimits_{E}g=1$.) Then,
\begin{equation}%
\begin{cases}
\dfrac{1}{\left\vert O\right\vert }, & \text{if }O\text{ is }E\text{-coeven}%
;\\
0, & \text{if }O\text{ is not }E\text{-coeven}%
\end{cases}
=\dfrac{1}{\left\vert G\right\vert }\sum_{\substack{g\in G;\\g\pi=\pi
}}\operatorname*{sign}\nolimits_{E}g.\label{eq.lem.burnside.sums.b}%
\end{equation}

\end{enumerate}
\end{lemma}

\begin{proof}
[Proof of Lemma \ref{lem.burnside.sums}.] Let $\operatorname{Stab}_{G}\pi$
denote the stabilizer of $\pi$; this is the subgroup $\left\{  g\in
G\ \mid\ g\pi=\pi\right\}  $ of $G$. The $G$-orbit of $\pi$ is $O$ (since $O$
is a $G$-orbit on $F$, and since $\pi\in O$). In other words, $O = G\pi$.
Therefore,
\[
\left\vert O\right\vert =\left\vert G\pi\right\vert =\dfrac{\left\vert
G\right\vert }{\left\vert \operatorname*{Stab}\nolimits_{G}\pi\right\vert }%
\]
(by the orbit-stabilizer theorem) and thus
\begin{equation}
\dfrac{1}{\left\vert O\right\vert }=\dfrac{\left\vert \operatorname{Stab}%
_{G}\pi\right\vert }{\left\vert G\right\vert }%
.\label{pf.thm.antipode.GammawG.os2}%
\end{equation}


(a) We have%
\[
\sum_{\substack{g\in G;\\g\pi=\pi}}1=\left\vert \underbrace{\left\{  g\in
G\ \mid\ g\pi=\pi\right\}  }_{=\operatorname*{Stab}\nolimits_{G}\pi
}\right\vert =\left\vert \operatorname*{Stab}\nolimits_{G}\pi\right\vert .
\]
Hence,%
\[
\dfrac{1}{\left\vert G\right\vert }\underbrace{\sum_{\substack{g\in
G;\\g\pi=\pi}}1}_{=\left\vert \operatorname*{Stab}\nolimits_{G}\pi\right\vert
}=\dfrac{1}{\left\vert G\right\vert }\left\vert \operatorname*{Stab}%
\nolimits_{G}\pi\right\vert =\dfrac{\left\vert \operatorname*{Stab}%
\nolimits_{G}\pi\right\vert }{\left\vert G\right\vert }=\dfrac{1}{\left\vert
O\right\vert }%
\]
(by (\ref{pf.thm.antipode.GammawG.os2})). This proves Lemma
\ref{lem.burnside.sums} (a).

(b) We need to prove (\ref{eq.lem.burnside.sums.b}). Assume first that $O$ is
$E$-coeven. Thus, $\pi$ is $E$-coeven (by the definition of what it means for
$O$ to be $E$-coeven). This means that every $g\in G$ satisfying $g\pi=\pi$ is
$E$-even. Hence, every $g\in G$ satisfying $g\pi=\pi$ satisfies
$\operatorname*{sign}\nolimits_{E}g=1$. Thus,
\begin{align*}
\dfrac{1}{\left\vert G\right\vert }\sum_{\substack{g\in G;\\g\pi=\pi
}}\underbrace{\operatorname*{sign}\nolimits_{E}g}_{=1} &  =\dfrac
{1}{\left\vert G\right\vert }\sum_{\substack{g\in G;\\g\pi=\pi}}1=\dfrac
{1}{\left\vert O\right\vert }\ \ \ \ \ \ \ \ \ \ \left(  \text{by
\eqref{eq.lem.burnside.sums.a}}\right)  \\
&  =%
\begin{cases}
\dfrac{1}{\left\vert O\right\vert }, & \text{if }O\text{ is }E\text{-coeven}%
;\\
0, & \text{if }O\text{ is not }E\text{-coeven}%
\end{cases}
\ \ \ \ \ \ \ \ \ \ \left(  \text{since }O\text{ is }E\text{-coeven}\right)  .
\end{align*}


Thus, we have proven (\ref{eq.lem.burnside.sums.b}) under the assumption that
$O$ is $E$-coeven. We can therefore WLOG assume the opposite now. Thus, assume
that $O$ is not $E$-coeven. Hence, no element of $O$ is $E$-coeven (due to the
contrapositive of Lemma~\ref{lem.coeven.all-one}). In particular, $\pi$ is not
$E$-coeven. In other words, not every $g\in G$ satisfying $g\pi=\pi$ is
$E$-even. In other words, not every $g\in\operatorname*{Stab}\nolimits_{G}\pi$
is $E$-even (since the elements $g\in G$ satisfying $g\pi=\pi$ are exactly the
elements $g\in\operatorname*{Stab}\nolimits_{G}\pi$). In other words, not
every $g\in\operatorname*{Stab}\nolimits_{G}\pi$ satisfies
$\operatorname*{sign}\nolimits_{E}g=1$.

Now, the map
\[
\operatorname*{Stab}\nolimits_{G}\pi\rightarrow\left\{  1,-1\right\}
,\ \ \ \ \ \ \ \ \ \ g\mapsto\operatorname*{sign}\nolimits_{E}g
\]
is a group homomorphism (since the sign of a permutation is multiplicative)
and is not the trivial homomorphism (since not every $g\in\operatorname*{Stab}%
\nolimits_{G}\pi$ satisfies $\operatorname*{sign}\nolimits_{E}g=1$). Hence, it
must send exactly half the elements of $\operatorname*{Stab}\nolimits_{G}\pi$
to $1$ and the other half to $-1$. Therefore, the addends in the sum
$\sum_{g\in\operatorname*{Stab}\nolimits_{G}\pi}\operatorname*{sign}%
\nolimits_{E}g$ cancel each other out (one half of them are $1$, and the
others are $-1$). Therefore, $\sum_{g\in\operatorname*{Stab}\nolimits_{G}\pi
}\operatorname*{sign}\nolimits_{E}g=0$, so that
\[
\dfrac{1}{\left\vert G\right\vert }\underbrace{\sum_{g\in\operatorname*{Stab}%
\nolimits_{G}\pi}\operatorname*{sign}\nolimits_{E}g}_{=0}=0=%
\begin{cases}
\dfrac{1}{\left\vert O\right\vert }, & \text{if }O\text{ is }E\text{-coeven}%
;\\
0, & \text{if }O\text{ is not }E\text{-coeven}%
\end{cases}
\]
(since $O$ is not $E$-coeven).
Thus,%
\[
\dfrac{1}{\left\vert G\right\vert }\underbrace{\sum_{\substack{g\in
G;\\g\pi=\pi}}}_{=\sum_{g\in\operatorname*{Stab}\nolimits_{G}\pi}%
}\operatorname*{sign}\nolimits_{E}g=\dfrac{1}{\left\vert G\right\vert }%
\sum_{g\in\operatorname*{Stab}\nolimits_{G}\pi}\operatorname*{sign}%
\nolimits_{E}g=%
\begin{cases}
\dfrac{1}{\left\vert O\right\vert }, & \text{if }O\text{ is }E\text{-coeven}%
;\\
0, & \text{if }O\text{ is not }E\text{-coeven}%
\end{cases}
.
\]
This proves (\ref{eq.lem.burnside.sums.b}). Lemma \ref{lem.burnside.sums} (b)
is thus proven.
\end{proof}

\begin{proof}
[Proof of Theorem~\ref{thm.antipode.GammawG} (sketched).]For every $g\in G$,
define a tertispecial double poset
$ \EE ^{g}=\left(  E^{g},<_{1}^{g},<_{2}^{g}\right)  $ as follows:

Let $E^{g}$ be the set of all orbits under the action of $g$ on $E$. Define a
binary relation $<_{1}^{g}$ on $E^{g}$ by
\[
\left(  u<_{1}^{g}v\right)  \Longleftrightarrow\left(  \text{there exist }a\in
u\text{ and }b\in v\text{ with }a<_{1}b\right)  .
\]
Similarly, define a strict partial order relation $<_{2}^{g}$ on $E^{g}$ by
\[
\left(  u<_{2}^{g}v\right)  \Longleftrightarrow\left(  \text{there exist }a\in
u\text{ and }b\in v\text{ with }a<_{2}b\right)  .
\]
Finally, set
$ \EE ^{g}=\left(  E^{g},<_{1}^{g},<_{2}^{g}\right)  $.
Proposition~\ref{prop.G-poset.quot.double} shows that this $ \EE ^{g}$ is
a tertispecial double poset.

Furthermore, for every $g\in G$, define a map $w^{g}:E^{g}\rightarrow
\left\{1,2,3,\ldots\right\}$
by $w^{g}\left(  u\right)  =\sum\limits_{a\in u}w\left(
a\right)  $. (Since $G$ preserves $w$, the numbers $w\left(  a\right)  $ for
all $a\in u$ are equal (for given $u$), and thus $\sum\limits_{a\in u}w\left(
a\right)  $ can be rewritten as $\left\vert u\right\vert \cdot w\left(
b\right)  $ for any particular $b\in u$.)
Now,
\begin{equation}
S\left(  \Gamma\left(  \left(  E^{g},<_{1}^{g},<_{2}^{g}\right)
,w^{g}\right)  \right)  =\left(  -1\right)  ^{\left\vert E^{g}\right\vert
}\Gamma\left(  \left(  E^{g},>_{1}^{g},<_{2}^{g}\right)  ,w^{g}\right)
\label{pf.thm.antipode.GammawG.S1}
\end{equation}
(by Theorem \ref{thm.antipode.Gammaw}, applied to
$\left(  \left(  E^{g},<_{1}^{g},<_{2}^{g}\right)  ,w^{g}\right)  $
instead of $\left(  \left( E,<_{1},<_{2}\right)  ,w\right)  $).

For every $g\in G$, we have
\begin{equation}
\sum_{\substack{\pi\text{ is an } \EE \text{-partition;}\\g\pi=\pi
}}\mathbf{x}_{\pi,w}=\Gamma\left(   \EE ^{g},w^{g}\right)
\label{pf.thm.antipode.GammawG.red}
\end{equation}
\footnote{\textit{Proof of \eqref{pf.thm.antipode.GammawG.red}:} Let $g\in G$.
In Proposition~\ref{prop.G-poset.quot.Phi}, we have introduced a
bijection $\Phi$ between
\par
\begin{itemize}
\item the maps $\pi:E\rightarrow\left\{  1,2,3,\ldots\right\}  $ satisfying
$g\pi=\pi$
\end{itemize}
\par
and
\par
\begin{itemize}
\item the maps $\overline{\pi}:E^{g}\rightarrow\left\{  1,2,3,\ldots\right\}
$.
\end{itemize}
\par
Parts (a) and (b) of Proposition~\ref{prop.G-poset.quot.Phi} show that
this bijection $\Phi$ restricts to a bijection between
\par
\begin{itemize}
\item the $ \EE $-partitions $\pi:E\rightarrow\left\{  1,2,3,\ldots
\right\}  $ satisfying $g\pi=\pi$
\end{itemize}
\par
and
\par
\begin{itemize}
\item the $ \EE ^{g}$-partitions $\overline{\pi}:E^{g}\rightarrow\left\{
1,2,3,\ldots\right\}  $.
\end{itemize}
\par
Hence,
\[
\sum_{\pi\text{ is an } \EE ^{g}\text{-partition}}
\mathbf{x}_{\pi,w^{g}}
=\sum_{\substack{\pi\text{ is an } \EE \text{-partition;}\\g\pi=\pi
}}
\underbrace{\xx_{\Phi\left(  \pi\right)  ,w^{g}}}_{
\substack{=\mathbf{x}_{\pi,w} \\
\text{(by Proposition~\eqref{prop.G-poset.quot.Phi} (c))}
}}
= \sum_{\substack{\pi\text{ is an } \EE \text{-partition;}
\\ g\pi=\pi}}\mathbf{x}_{\pi,w},
\]
whence $\sum_{\substack{\pi\text{ is an } \EE \text{-partition;}
\\g\pi=\pi}}\mathbf{x}_{\pi,w}=\sum_{\pi\text{ is an }
\EE ^{g}\text{-partition}}\mathbf{x}_{\pi,w^{g}}
=\Gamma\left(  \EE^{g},w^{g}\right)  $.
This proves \eqref{pf.thm.antipode.GammawG.red}.}.

It is clearly sufficient to prove
Theorem~\ref{thm.antipode.GammawG} for $\kk = \ZZ$ (since all
the power series that we are discussing are defined functorially
in $\kk$, and thus any identity between these series that holds
over $\ZZ$ must hold over any $\kk$). Therefore, it is sufficient
to prove Theorem~\ref{thm.antipode.GammawG} for $\kk = \QQ$ (since
$\QSym_{\ZZ}$ embeds into $\QSym_{\QQ}$\ \ \ \ \footnote{Here, we
are using the notation $\QSym_{\kk}$ for the Hopf algebra $\QSym$
defined over a base ring $\kk$.}).
Thus, we WLOG assume that $\kk = \QQ$.
This will allow us to divide by positive integers.

Every $G$-orbit $O$ on $\Par \EE$ satisfies
\begin{equation}
\dfrac{1}{\left|O\right|} \sum_{\pi \in O}
\underbrace{\xx_{\pi, w}}_{\substack{= \xx_{O, w} \\
                           \text{(since } \xx_{O, w}
                           \text{ is defined} \\
                           \text{to be } \xx_{\pi, w}
                           \text{)}}}
= \dfrac{1}{\left|O\right|} \underbrace{\sum_{\pi \in O} \xx_{O, w}}_{
                                        = \left|O\right| \xx_{O, w}}
= \dfrac{1}{\left|O\right|} \left|O\right| \xx_{O, w}
= \xx_{O, w} .
\label{pf.thm.antipode.GammawG.averaging}
\end{equation}

Now,
\begin{align}
\Gamma\left(  {{\mathbf{E}}},w,G\right)   &  =\sum_{O\text{ is a
}G\text{-orbit on }\operatorname{Par}{\mathbf{E}}}\underbrace{{\mathbf{x}%
}_{O,w}}_{\substack{=\dfrac{1}{\left\vert O\right\vert }\sum\limits_{\pi\in
O}\mathbf{x}_{\pi,w}\\\text{(by \eqref{pf.thm.antipode.GammawG.averaging})}%
}}=\sum_{O\text{ is a }G\text{-orbit on }\operatorname{Par}{{\mathbf{E}}}%
}\dfrac{1}{\left\vert O\right\vert }\sum\limits_{\pi\in O}\mathbf{x}_{\pi
,w}\nonumber\\
&  =\sum_{O\text{ is a }G\text{-orbit on }\operatorname{Par}{{\mathbf{E}}}%
}\sum\limits_{\pi\in O}\underbrace{\dfrac{1}{\left\vert O\right\vert }%
}_{\substack{=\dfrac{1}{\left\vert G\right\vert }\sum_{\substack{g\in
G;\\g\pi=\pi}}1\\\text{(by \eqref{eq.lem.burnside.sums.a}, applied to }
F = \Par\EE {)}}}\mathbf{x}%
_{\pi,w}\nonumber\\
&  =\underbrace{\sum_{O\text{ is a }G\text{-orbit on }\operatorname{Par}%
{\mathbf{E}}}\sum\limits_{\pi\in O}}_{=\sum_{\pi\in\operatorname{Par}%
{\mathbf{E}}}=\sum_{\pi\text{ is an }{\mathbf{E}}\text{-partition}}}\left(
\dfrac{1}{\left\vert G\right\vert }\sum_{\substack{g\in G;\\g\pi=\pi
}}1\right)  \mathbf{x}_{\pi,w}\nonumber\\
&  =\sum_{\pi\text{ is an }{\mathbf{E}}\text{-partition}}\left(  \dfrac
{1}{\left\vert G\right\vert }\sum_{\substack{g\in G;\\g\pi=\pi}}1\right)
\mathbf{x}_{\pi,w}=\dfrac{1}{\left\vert G\right\vert }\underbrace{\sum
_{\pi\text{ is an }{\mathbf{E}}\text{-partition}}\sum_{\substack{g\in
G;\\g\pi=\pi}}}_{=\sum_{g\in G}\sum_{\substack{\pi\text{ is an }{\mathbf{E}%
}\text{-partition;}\\g\pi=\pi}}}\mathbf{x}_{\pi,w}\nonumber\\
&  =\dfrac{1}{\left\vert G\right\vert }\sum_{g\in G}\underbrace{\sum
_{\substack{\pi\text{ is an }{\mathbf{E}}\text{-partition;}\\g\pi=\pi
}}\mathbf{x}_{\pi,w}}_{\substack{=\Gamma\left(  {\mathbf{E}}^{g},w^{g}\right)
\\\text{(by \eqref{pf.thm.antipode.GammawG.red})}}}\nonumber\\
&  =\dfrac{1}{\left\vert G\right\vert }\sum_{g\in G}\Gamma\left(
\underbrace{{\mathbf{E}}^{g}}_{=\left(  E^{g},<_{1}^{g},<_{2}^{g}\right)
},w^{g}\right)  =\dfrac{1}{\left\vert G\right\vert }\sum_{g\in G}\Gamma\left(
\left(  E^{g},<_{1}^{g},<_{2}^{g}\right)  ,w^{g}\right)
.\label{pf.thm.antipode.GammawG.1}%
\end{align}
Hence, $\Gamma\left(  { \EE },w,G\right) \in \QSym$
(by Proposition~\ref{prop.Gammaw.qsym}).

Applying the map $S$ to both sides of the equality
\eqref{pf.thm.antipode.GammawG.1}, we obtain
\begin{align}
S\left(  \Gamma\left(  { \EE },w,G\right)  \right)    & =\dfrac
{1}{\left\vert G\right\vert }\sum_{g\in G}\underbrace{S\left(  \Gamma\left(
\left(  E^{g},<_{1}^{g},<_{2}^{g}\right)  ,w^{g}\right)  \right)
}_{\substack{=\left(  -1\right)  ^{\left\vert E^{g}\right\vert }\Gamma\left(
\left(  E^{g},>_{1}^{g},<_{2}^{g}\right)  ,w^{g}\right)  \\\text{(by
\eqref{pf.thm.antipode.GammawG.S1})}}}\nonumber\\
& =\dfrac{1}{\left\vert G\right\vert }\sum_{g\in G}\left(  -1\right)
^{\left\vert E^{g}\right\vert }
\Gamma\left(  \left(  E^{g},>_{1}^{g},<_{2}^{g}\right)  ,w^{g}\right)
.
\label{pf.thm.antipode.GammawG.1S}
\end{align}

On the other hand, for every $g\in G$, let $\operatorname{sign}_{E}g$
denote the sign of the permutation of $E$ that sends every
$e\in E$ to $ge$. Thus, $g\in G$ is $E$-even if and only if $\operatorname*{sign}%
\nolimits_{E}g=1$. Now, every $G$-orbit $O$ on $\Par \EE $
and every $\pi\in O$ satisfy%
\begin{equation}%
\begin{cases}
\dfrac{1}{\left\vert O\right\vert }, & \text{if }O\text{ is }E\text{-coeven};\\
0, & \text{if }O\text{ is not }E\text{-coeven}%
\end{cases}
=\dfrac{1}{\left\vert G\right\vert }\sum_{\substack{g\in G; \\ g\pi = \pi}}
\operatorname*{sign}\nolimits_{E} g
\label{pf.thm.antipode.GammawG.signed}%
\end{equation}
(by \eqref{eq.lem.burnside.sums.b}, applied to $F = \Par\EE$). Furthermore,%
\begin{equation}
\operatorname*{sign}\nolimits_{E}g=\left(  -1\right)  ^{\left\vert
E\right\vert -\left\vert E^{g}\right\vert }%
\label{pf.thm.antipode.GammawG.sign}%
\end{equation}
for every $g\in G$\ \ \ \ \footnote{\textit{Proof of
\eqref{pf.thm.antipode.GammawG.sign}:} Let $g\in G$. Recall that
$\operatorname*{sign}\nolimits_{E}g$ is the sign of the permutation of $E$
that sends every $e\in E$ to $ge$. But if $\sigma$ is a permutation of a
finite set $X$, then the sign of $\sigma$ is $\left(  -1\right)  ^{\left\vert
X\right\vert -\left\vert X^{\sigma}\right\vert }$, where $X^{\sigma}$ is the
set of all cycles of $\sigma$. Applying this to $X=E$, $\sigma=\left(
\text{the permutation of }E\text{ that sends every }e\in E\text{ to
}ge\right)  $ and $X^{\sigma}=E^{g}$, we see that the sign of the permutation
of $E$ that sends every $e\in E$ to $ge$ is $\left(  -1\right)  ^{\left\vert
E\right\vert -\left\vert E^{g}\right\vert }$. In other words,
$\operatorname*{sign}\nolimits_{E}g=\left(  -1\right)  ^{\left\vert
E\right\vert -\left\vert E^{g}\right\vert }$, qed.}.

Now,%
\begin{align}
&  \Gamma^{+}\left(  {\mathbf{E}},w,G\right)  \nonumber\\
&  =\sum_{O\text{ is an }E\text{-coeven }G\text{-orbit on }\operatorname{Par}%
{\mathbf{E}}}\underbrace{{\mathbf{x}}_{O,w}}_{\substack{=\dfrac{1}{\left\vert
O\right\vert }\sum\limits_{\pi\in O}\mathbf{x}_{\pi,w}\\\text{(by
\eqref{pf.thm.antipode.GammawG.averaging})}}}=\sum_{O\text{ is an
}E\text{-coeven }G\text{-orbit on }\operatorname{Par}{\mathbf{E}}}\dfrac
{1}{\left\vert O\right\vert }\sum\limits_{\pi\in O}\mathbf{x}_{\pi
,w}\nonumber\\
&  =\sum_{O\text{ is a }G\text{-orbit on }\operatorname{Par}{\mathbf{E}}}%
\begin{cases}
\dfrac{1}{\left\vert O\right\vert }, & \text{if }O\text{ is }E\text{-coeven}%
;\\
0, & \text{if }O\text{ is not }E\text{-coeven}%
\end{cases}
\sum\limits_{\pi\in O}\mathbf{x}_{\pi,w}\nonumber\\
&  \qquad\left(
\begin{array}
[c]{c}%
\text{here, we have extended the sum to all }G\text{-orbits}\\
\text{on }\operatorname{Par}{\mathbf{E}}\text{ (not just the }E\text{-coeven
ones); but all new addends are }0\\
\text{and therefore do not influence the value of the sum}%
\end{array}
\right)  \nonumber\\
&  =\sum_{O\text{ is a }G\text{-orbit on }\operatorname{Par}{\mathbf{E}}}%
\sum\limits_{\pi\in O}\underbrace{%
\begin{cases}
\dfrac{1}{\left\vert O\right\vert }, & \text{if }O\text{ is }E\text{-coeven}%
;\\
0, & \text{if }O\text{ is not }E\text{-coeven}%
\end{cases}
}_{\substack{=\dfrac{1}{\left\vert G\right\vert }\sum_{\substack{g\in
G;\\g\pi=\pi}}\operatorname*{sign}\nolimits_{E}g\\\text{(by
\eqref{pf.thm.antipode.GammawG.signed})}}}\mathbf{x}_{\pi,w}\nonumber\\
&  =\underbrace{\sum_{O\text{ is a }G\text{-orbit on }\operatorname{Par}%
{{\mathbf{E}}}}\sum\limits_{\pi\in O}}_{=\sum_{\pi\in\operatorname{Par}%
{\mathbf{E}}}=\sum_{\pi\text{ is an }{\mathbf{E}}\text{-partition}}}\left(
\dfrac{1}{\left\vert G\right\vert }\sum_{\substack{g\in G;\\g\pi=\pi
}}\operatorname*{sign}\nolimits_{E}g\right)  \mathbf{x}_{\pi,w}\nonumber\\
&  =\sum_{\pi\text{ is an }{\mathbf{E}}\text{-partition}}\left(  \dfrac
{1}{\left\vert G\right\vert }\sum_{\substack{g\in G;\\g\pi=\pi}%
}\operatorname*{sign}\nolimits_{E}g\right)  \mathbf{x}_{\pi,w}=\dfrac
{1}{\left\vert G\right\vert }\underbrace{\sum_{\pi\text{ is an }{\mathbf{E}%
}\text{-partition}}\sum_{\substack{g\in G;\\g\pi=\pi}}}_{=\sum_{g\in G}%
\sum_{\substack{\pi\text{ is an }{\mathbf{E}}\text{-partition;}\\g\pi=\pi}%
}}\left(  \operatorname*{sign}\nolimits_{E}g\right)  \mathbf{x}_{\pi
,w}\nonumber\\
&  =\dfrac{1}{\left\vert G\right\vert }\sum_{g\in G}%
\underbrace{\operatorname*{sign}\nolimits_{E}g}_{\substack{=\left(  -1\right)
^{\left\vert E\right\vert -\left\vert E^{g}\right\vert }\\\text{(by
\eqref{pf.thm.antipode.GammawG.sign})}}}\underbrace{\sum_{\substack{\pi\text{
is an }{\mathbf{E}}\text{-partition;}\\g\pi=\pi}}\mathbf{x}_{\pi,w}%
}_{\substack{=\Gamma\left(  {\mathbf{E}}^{g},w^{g}\right)  \\\text{(by
\eqref{pf.thm.antipode.GammawG.red})}}}\nonumber\\
&  =\dfrac{1}{\left\vert G\right\vert }\sum_{g\in G}\left(  -1\right)
^{\left\vert E\right\vert -\left\vert E^{g}\right\vert }\Gamma\left(
{\mathbf{E}}^{g},w^{g}\right)  .
\label{pf.thm.antipode.GammawG.Gammaplus}%
\end{align}
Hence, $\Gamma^+\left(  { \EE },w,G\right) \in \QSym$
(by Proposition~\ref{prop.Gammaw.qsym}).

The group $G$ preserves the relation $>_1$ (since it preserves the
relation $<_1$). Hence, applying
\eqref{pf.thm.antipode.GammawG.Gammaplus}
to $\left(  E,>_{1},<_{2}\right)  $ instead of $ \EE $, we
obtain
\[
\Gamma^{+}\left(  \left(  E,>_{1},<_{2}\right)  ,w,G\right)  =\dfrac
{1}{\left\vert G\right\vert }\sum_{g\in G}\left(  -1\right)  ^{\left\vert
E\right\vert -\left\vert E^{g}\right\vert }\Gamma\left(  \left(  E^{g}%
,>_{1}^{g},<_{2}^{g}\right)  ,w^{g}\right)  .
\]
Multiplying both sides of this equality by $\left(  -1\right)  ^{\left\vert
E\right\vert }$, we transform it into%
\begin{align*}
\left(  -1\right)  ^{\left\vert E\right\vert }\Gamma^{+}\left(  \left(
E,>_{1},<_{2}\right)  ,w,G\right)    & =\dfrac{1}{\left\vert G\right\vert
}\sum_{g\in G}\underbrace{\left(  -1\right)  ^{\left\vert E\right\vert
}\left(  -1\right)  ^{\left\vert E\right\vert -\left\vert E^{g}\right\vert }%
}_{=\left(  -1\right)  ^{\left\vert E^{g}\right\vert }}\Gamma\left(  \left(
E^{g},>_{1}^{g},<_{2}^{g}\right)  ,w^{g}\right)  \\
& =\dfrac{1}{\left\vert G\right\vert }\sum_{g\in G}\left(  -1\right)
^{\left\vert E^{g}\right\vert }\Gamma\left(  \left(  E^{g},>_{1}^{g},<_{2}%
^{g}\right)  ,w^{g}\right)  \\
& =S\left(  \Gamma\left(  { \EE },w,G\right)  \right)
\ \ \ \ \ \ \ \ \ \ 
\left(  \text{by \eqref{pf.thm.antipode.GammawG.1S}}\right)  .
\end{align*}
This proves Theorem~\ref{thm.antipode.GammawG}.
\end{proof}


\section{Application: Jochemko's theorem}
\label{sect.jochemko}

We shall now demonstrate an application of Theorem \ref{thm.antipode.GammawG}:
namely, we will use it to provide an alternative proof of \cite[Theorem
2.13]{Joch}. The way we derive \cite[Theorem 2.13]{Joch} from Theorem
\ref{thm.antipode.GammawG} is classical, and in fact was what originally
motivated the discovery of Theorem \ref{thm.antipode.GammawG} (although, of
course, it cannot be conversely derived from \cite[Theorem 2.13]{Joch}, so it
is an actual generalization).

An intermediate step between \cite[Theorem 2.13]{Joch} and Theorem
\ref{thm.antipode.GammawG} will be the following fact:

\begin{corollary}
\label{cor.reciprocity.GammawG}Let ${\mathbf{E}}=\left(  E,<_{1},<_{2}\right)
$ be a tertispecial double poset. Let $w:E\rightarrow\left\{  1,2,3,\ldots
\right\}  $. Let $G$ be a finite group which acts on $E$. Assume that $G$
preserves both relations $<_{1}$ and $<_{2}$, and also preserves $w$. For
every $q\in\mathbb{N}$, let $\operatorname*{Par}\nolimits_{q}\mathbf{E}$
denote the set of all $\mathbf{E}$-partitions whose image is contained in
$\left\{  1,2,\ldots,q\right\}  $. Then, the group $G$ also acts on
$\operatorname*{Par}\nolimits_{q}\mathbf{E}$; namely, $\operatorname{Par}%
_{q}{\mathbf{E}}$ is a $G$-subset of the $G$-set $\left\{  1,2,\ldots
,q\right\}  ^{E}$ (see Definition~\ref{def.G-sets.terminology} (d) for the
definition of the latter).

\begin{enumerate}
\item[(a)] There exists a unique polynomial $\Omega_{\mathbf{E},G}%
\in\mathbb{Q}\left[  X\right]  $ such that every $q\in\mathbb{N}$ satisfies%
\begin{equation}
\Omega_{\mathbf{E},G}\left(  q\right)  =\left(  \text{the number of all
}G\text{-orbits on }\operatorname*{Par}\nolimits_{q}\mathbf{E}\right)  .
\label{eq.cor.reciprocity.GammawG.a.def}%
\end{equation}


\item[(b)] This polynomial satisfies%
\begin{align}
&  \Omega_{\mathbf{E},G}\left(  -q\right) \nonumber\\
&  =\left(  -1\right)  ^{\left\vert E\right\vert }\left(  \text{the number of
all even }G\text{-orbits on }\operatorname*{Par}\nolimits_{q}\left(
E,>_{1},<_{2}\right)  \right) \nonumber\\
&  =\left(  -1\right)  ^{\left\vert E\right\vert }\left(  \text{the number of
all even }G\text{-orbits on }\operatorname*{Par}\nolimits_{q}\left(
E,<_{1},>_{2}\right)  \right)  \label{eq.cor.reciprocity.GammawG.b.2}%
\end{align}
for all $q\in\mathbb{N}$.
\end{enumerate}
\end{corollary}

\begin{proof}
[Proof of Corollary \ref{cor.reciprocity.GammawG} (sketched).]Set
$\mathbf{k}=\mathbb{Q}$. For any $f\in\operatorname*{QSym}$ and any
$q\in\mathbb{N}$, we define an element $\operatorname*{ps}\nolimits^{1}\left(
f\right)  \left(  q\right)  \in\mathbb{Q}$ by%
\[
\operatorname*{ps}\nolimits^{1}\left(  f\right)  \left(  q\right)  =f\left(
\underbrace{1,1,\ldots,1}_{q\text{ times}},0,0,0,\ldots\right)
\]
(that is, $\operatorname*{ps}\nolimits^{1}\left(  f\right)  \left(  q\right)
$ is the result of substituting $1$ for $x_{1},x_{2},\ldots,x_{q}$ and $0$ for
$x_{q+1},x_{q+2},x_{q+3},\ldots$ in the power series $f$).

(a) Consider the elements $\Gamma\left(  {\mathbf{E}},w,G\right)  $ and
$\Gamma^{+}\left(  {\mathbf{E}},w,G\right)  $ of ${\operatorname{QSym}}$
defined in Theorem \ref{thm.antipode.GammawG}. Observe that
$\operatorname*{Par}\nolimits_{q}\mathbf{E}$ is a $G$-subset of
$\operatorname*{Par}\mathbf{E}$.

\begin{noncompile}
Clearly, there exists \textbf{at most} one polynomial $\Omega_{\mathbf{E}%
,G}\in\mathbb{Q}\left[  X\right]  $ such that every $q\in\mathbb{N}$ satisfies
(\ref{eq.cor.reciprocity.GammawG.a.def}) (because a polynomial in
$\mathbb{Q}\left[  X\right]  $ is uniquely determined by its values at all
nonnegative integers). It remains to show that there exists \textbf{at least}
one such polynomial.
\end{noncompile}

Now, \cite[Proposition 7.7 (i)]{Reiner} shows that, for any given
$f\in\operatorname*{QSym}$, there exists a unique polynomial in $\mathbb{Q}%
\left[  X\right]  $ whose value on each $q\in\mathbb{N}$ equals
$\operatorname*{ps}\nolimits^{1}\left(  f\right)  \left(  q\right)  $.
Applying this to $f=\Gamma\left(  {\mathbf{E}},w,G\right)  $, we conclude that
there exists a unique polynomial in $\mathbb{Q}\left[  X\right]  $ whose value
on each $q\in\mathbb{N}$ equals $\operatorname*{ps}\nolimits^{1}\left(
\Gamma\left(  {\mathbf{E}},w,G\right)  \right)  \left(  q\right)  $. But since
every $q\in\mathbb{N}$ satisfies
\begin{align}
\operatorname*{ps}\nolimits^{1}\left(  \Gamma\left(  {\mathbf{E}},w,G\right)
\right)  \left(  q\right)   &  =\underbrace{\left(  \Gamma\left(  {\mathbf{E}%
},w,G\right)  \right)  }_{=\sum_{O\text{ is a }G\text{-orbit on }%
\operatorname{Par}{\mathbf{E}}}{\mathbf{x}}_{O,w}}\left(
\underbrace{1,1,\ldots,1}_{q\text{ times}},0,0,0,\ldots\right) \nonumber\\
&  =\sum_{O\text{ is a }G\text{-orbit on }\operatorname{Par}{\mathbf{E}}%
}\underbrace{{\mathbf{x}}_{O,w}\left(  \underbrace{1,1,\ldots,1}_{q\text{
times}},0,0,0,\ldots\right)  }_{=%
\begin{cases}
1, & \text{if }O\subseteq\operatorname*{Par}\nolimits_{q}\mathbf{E};\\
0, & \text{if }O\not \subseteq \operatorname*{Par}\nolimits_{q}\mathbf{E}%
\end{cases}
}\nonumber\\
&  =\sum_{O\text{ is a }G\text{-orbit on }\operatorname{Par}{\mathbf{E}}}%
\begin{cases}
1, & \text{if }O\subseteq\operatorname*{Par}\nolimits_{q}\mathbf{E};\\
0, & \text{if }O\not \subseteq \operatorname*{Par}\nolimits_{q}\mathbf{E}%
\end{cases}
\nonumber\\
&  =\sum_{O\text{ is a }G\text{-orbit on }\operatorname{Par}_{q}{\mathbf{E}}%
}1=\left(  \text{the number of all }G\text{-orbits on }\operatorname*{Par}%
\nolimits_{q}\mathbf{E}\right)  , \label{pf.cor.reciprocity.GammawG.a.1}%
\end{align}
this rewrites as follows: There exists a unique polynomial in $\mathbb{Q}%
\left[  X\right]  $ whose value on each $q\in\mathbb{N}$ equals $\left(
\text{the number of all }G\text{-orbits on }\operatorname*{Par}\nolimits_{q}%
\mathbf{E}\right)  $. This proves Corollary \ref{cor.reciprocity.GammawG} (a).

(b) \cite[Proposition 7.7 (i)]{Reiner} shows that, for any given
$f\in\operatorname*{QSym}$, there exists a unique polynomial in $\mathbb{Q}%
\left[  X\right]  $ whose value on each $q\in\mathbb{N}$ equals
$\operatorname*{ps}\nolimits^{1}\left(  f\right)  \left(  q\right)  $. This
polynomial is denoted by $\operatorname*{ps}\nolimits^{1}\left(  f\right)  $
in \cite[Proposition 7.7]{Reiner}. From our above proof of Corollary
\ref{cor.reciprocity.GammawG} (a), we see that
\[
\Omega_{\mathbf{E},G}=\operatorname*{ps}\nolimits^{1}\left(  \Gamma\left(
{\mathbf{E}},w,G\right)  \right)  .
\]


But \cite[Proposition 7.7 (iii)]{Reiner} shows that, for any $f\in
\operatorname*{QSym}$ and $m\in\mathbb{N}$, we have $\operatorname*{ps}%
\nolimits^{1}\left(  S\left(  f\right)  \right)  \left(  m\right)
=\operatorname*{ps}\nolimits^{1}\left(  f\right)  \left(  -m\right)  $.
Applying this to $f=\Gamma\left(  {\mathbf{E}},w,G\right)  $, we obtain%
\[
\operatorname*{ps}\nolimits^{1}\left(  S\left(  \Gamma\left(  {\mathbf{E}%
},w,G\right)  \right)  \right)  \left(  m\right)
=\underbrace{\operatorname*{ps}\nolimits^{1}\left(  \Gamma\left(  {\mathbf{E}%
},w,G\right)  \right)  }_{=\Omega_{\mathbf{E},G}}\left(  -m\right)
=\Omega_{\mathbf{E},G}\left(  -m\right)
\]
for any $m\in\mathbb{N}$. Thus, any $m\in\mathbb{N}$ satsfies%
\begin{align*}
\Omega_{\mathbf{E},G}\left(  -m\right)   &  =\operatorname*{ps}\nolimits^{1}%
\left(  \underbrace{S\left(  \Gamma\left(  {\mathbf{E}},w,G\right)  \right)
}_{\substack{=\left(  -1\right)  ^{\left\vert E\right\vert }\Gamma^{+}\left(
\left(  E,>_{1},<_{2}\right)  ,w,G\right)  \\\text{(by Theorem
\ref{thm.antipode.GammawG})}}}\right)  \left(  m\right) \\
&  =\operatorname*{ps}\nolimits^{1}\left(  \left(  -1\right)  ^{\left\vert
E\right\vert }\Gamma^{+}\left(  \left(  E,>_{1},<_{2}\right)  ,w,G\right)
\right)  \left(  m\right) \\
&  =\left(  -1\right)  ^{\left\vert E\right\vert }\operatorname*{ps}%
\nolimits^{1}\left(  \Gamma^{+}\left(  \left(  E,>_{1},<_{2}\right)
,w,G\right)  \right)  \left(  m\right)  .
\end{align*}
Renaming $m$ as $q$ in this equality, we see that every $q\in\mathbb{N}$
satisfies%
\begin{equation}
\Omega_{\mathbf{E},G}\left(  -q\right)  =\left(  -1\right)  ^{\left\vert
E\right\vert }\operatorname*{ps}\nolimits^{1}\left(  \Gamma^{+}\left(  \left(
E,>_{1},<_{2}\right)  ,w,G\right)  \right)  \left(  q\right)  .
\label{pf.cor.reciprocity.GammawG.b.2}%
\end{equation}


But just as we proved (\ref{pf.cor.reciprocity.GammawG.a.1}), we can show that
every $q\in\mathbb{N}$ satisfies
\[
\operatorname*{ps}\nolimits^{1}\left(  \Gamma^{+}\left(  {\mathbf{E}%
},w,G\right)  \right)  \left(  q\right)  =\left(  \text{the number of all even
}G\text{-orbits on }\operatorname*{Par}\nolimits_{q}\mathbf{E}\right)  .
\]
Applying this to $\left(  E,>_{1},<_{2}\right)  $ instead of $\mathbf{E}$, we
obtain%
\begin{align*}
&  \operatorname*{ps}\nolimits^{1}\left(  \Gamma^{+}\left(  \left(
E,>_{1},<_{2}\right)  ,w,G\right)  \right)  \left(  q\right) \\
&  =\left(  \text{the number of all even }G\text{-orbits on }%
\operatorname*{Par}\nolimits_{q}\left(  E,>_{1},<_{2}\right)  \right)  .
\end{align*}
Now, (\ref{pf.cor.reciprocity.GammawG.b.2}) becomes%
\begin{align*}
\Omega_{\mathbf{E},G}\left(  -q\right)   &  =\left(  -1\right)  ^{\left\vert
E\right\vert }\underbrace{\operatorname*{ps}\nolimits^{1}\left(  \Gamma
^{+}\left(  \left(  E,>_{1},<_{2}\right)  ,w,G\right)  \right)  \left(
q\right)  }_{=\left(  \text{the number of all even }G\text{-orbits on
}\operatorname*{Par}\nolimits_{q}\left(  E,>_{1},<_{2}\right)  \right)  }\\
&  =\left(  -1\right)  ^{\left\vert E\right\vert }\left(  \text{the number of
all even }G\text{-orbits on }\operatorname*{Par}\nolimits_{q}\left(
E,>_{1},<_{2}\right)  \right)  .
\end{align*}


In order to prove Corollary \ref{cor.reciprocity.GammawG} (b), it thus remains
to show that%
\begin{align}
&  \left(  \text{the number of all even }G\text{-orbits on }%
\operatorname*{Par}\nolimits_{q}\left(  E,>_{1},<_{2}\right)  \right)
\nonumber\\
&  =\left(  \text{the number of all even }G\text{-orbits on }%
\operatorname*{Par}\nolimits_{q}\left(  E,<_{1},>_{2}\right)  \right)
\label{pf.cor.reciprocity.GammawG.b.goal5}%
\end{align}
for every $q\in\mathbb{N}$.

\textit{Proof of (\ref{pf.cor.reciprocity.GammawG.b.goal5}):} Let
$q\in\mathbb{N}$. Let $w_{0}:\left\{  1,2,\ldots,q\right\}  \rightarrow
\left\{  1,2,\ldots,q\right\}  $ be the map sending each $i\in\left\{
1,2,\ldots,q\right\}  $ to $q+1-i$. Then, the map%
\[
\operatorname*{Par}\nolimits_{q}\left(  E,>_{1},<_{2}\right)  \rightarrow
\operatorname*{Par}\nolimits_{q}\left(  E,<_{1},>_{2}\right)
,\ \ \ \ \ \ \ \ \ \ \pi\mapsto w_{0}\circ\pi
\]
is an isomorphism of $G$-sets (this is easy to check). Thus,
$\operatorname*{Par}\nolimits_{q}\left(  E,>_{1},<_{2}\right)  \cong%
\operatorname*{Par}\nolimits_{q}\left(  E,<_{1},>_{2}\right)  $ as $G$-sets.
From this, (\ref{pf.cor.reciprocity.GammawG.b.goal5}) follows (by
functoriality, if one wishes).

The proof of Corollary \ref{cor.reciprocity.GammawG} (b) is now complete.
\end{proof}

Now, the second formula of \cite[Theorem 2.13]{Joch} follows from our
(\ref{eq.cor.reciprocity.GammawG.b.2}), applied to $\mathbf{E}=\left(
P,\prec,<_{\omega}\right)  $ (where $<_{\omega}$ is the partial order on $P$
given by $\left(  p<_{\omega}q\right)  \Longleftrightarrow\left(
\omega\left(  p\right)  <\omega\left(  q\right)  \right)  $). The first
formula of \cite[Theorem 2.13]{Joch} can also be derived from our above
arguments. We leave the details to the reader.

\nocite{*}
\bibliographystyle{abbrvnat}
% use the following instead if you encounter problems 
%\bibliographystyle{alpha}
\bibliography{sample}
\label{sec:biblio}

\begin{thebibliography}{99999999}                                                                                         %

\bibitem[Abe77]{Abe-HA}Eiichi Abe, \textit{Hopf Algebras}, CUP 1977.

%\bibitem[ABS03]{ABS}Marcelo Aguiar, Nantel Bergeron, Frank Sottile,
%\textit{Combinatorial Hopf algebras and generalized Dehn-Sommerville
%relations}, Compositio Mathematica, vol. 142, Issue 01, January 2006, pp.
%1--30.\newline Also available as arXiv:math/0310016.\newline Newer version at
%\url{http://www.math.tamu.edu/~maguiar/CHalgebra.pdf}

\bibitem[BenSag14]{BenSag}Carolina Benedetti, Bruce Sagan, \textit{Antipodes
and involutions}, arXiv:1410.5023v3.\newline
\url{http://arxiv.org/abs/1410.5023v3}

\bibitem[BBSSZ13]{BBSSZ}Chris Berg, Nantel Bergeron, Franco Saliola, Luis
Serrano, Mike Zabrocki, \textit{A lift of the Schur and Hall-Littlewood bases
to non-commutative symmetric functions}, Canadian Journal of Mathematics 66
(2014), pp. 525--565.\newline
\url{http://dx.doi.org/10.4153/CJM-2013-013-0}
\newline Also available as arXiv:1208.5191v3.\newline
\url{http://arxiv.org/abs/1208.5191v3}

\bibitem[DNR01]{Dasca-HA}Sorin D\u{a}sc\u{a}lescu, Constantin
N\u{a}st\u{a}sescu, \c{S}erban Raianu, \textit{Hopf Algebras}, Marcel Dekker 2001.

\bibitem[Fresse14]{Fresse-Op}Benoit Fresse, \textit{Homotopy of operads \&
Grothendieck-Teichm\"{u}ller groups, First Volume},
preprint, 30 January 2016.\newline
\url{http://math.univ-lille1.fr/~fresse/OperadHomotopyBook/OperadHomotopy-FirstVolume.pdf}

\bibitem[Gessel84]{Gessel}Ira M. Gessel, \textit{Multipartite
P-partitions and Inner Products of Skew Schur Functions}, Contemporary
Mathematics, vol. 34, 1984, pp. 289--301.\newline%
\url{http://people.brandeis.edu/~gessel/homepage/papers/multipartite.pdf}

\bibitem[Gessel15]{Gessel-Ppar}Ira M. Gessel, \textit{A Historical Survey of
$P$-Partitions}, to be published in Richard Stanley's 70th Birthday Festschrift,
arXiv:1506.03508v1.\newline
\url{http://arxiv.org/abs/1506.03508v1}

%\bibitem[GesReu93]{Gessel-Reutenauer}Ira M. Gessel, Christophe Reutenauer,
%\textit{Counting Permutations with Given Cycle Structure
%and Descent Set},
%Journal of Combinatorial Theory, Series A 64, pp. 189--215 (1993).

\bibitem[Grin14]{Gri-dimm}Darij Grinberg,
\textit{Dual immaculate creation operators and a dendriform algebra structure on the quasisymmetric functions},
\href{http://arxiv.org/abs/1410.0079v6}{arXiv preprint arXiv:1410.0079v6}.

\bibitem[GriRei14]{Reiner}Darij Grinberg, Victor Reiner, \textit{Hopf algebras
in Combinatorics}, August 22, 2016, arXiv:1409.8356v4.\newline
\url{http://www.math.umn.edu/~reiner/Classes/HopfComb.pdf}

\bibitem[HaGuKi10]{HGK}Michiel Hazewinkel, Nadiya Gubareni, V. V. Kirichenko,
\textit{Algebras, Rings and Modules: Lie Algebras and Hopf Algebras},
AMS 2010.

%\bibitem[Hivert99]{Hivert-CQS}Florent Hivert, \textit{Combinatoire des
%fonctions quasi-sym\'{e}triques}, PhD thesis, defended 1999, January the
%15.\newline
%\url{https://www.lri.fr/~hivert/PAPER/these.ps}

\bibitem[Joch13]{Joch}Katharina Jochemko, \textit{Order polynomials and
P\'{o}lya's enumeration theorem},
The Electronic Journal of Combinatorics 21(2) (2014), P2.52.
See also
\texttt{\href{http://arxiv.org/abs/1310.0838v2}{arXiv:1310.0838v2}}
for a preprint.

%\bibitem[LMvW13]{LMvW}Kurt Luoto, Stefan Mykytiuk and Stephanie van
%Willigenburg, \textit{An introduction to quasisymmetric Schur functions --
%Hopf algebras, quasisymmetric functions, and Young composition tableaux}, May
%23, 2013.\newline
%\url{http://www.math.ubc.ca/~steph/papers/QuasiSchurBook.pdf}
%% [todo: get rid of this if not enough space, else expand it]

\bibitem[Malve93]{Malve-Thesis}Claudia Malvenuto, \textit{Produits et
coproduits des fonctions quasi-sym\'{e}triques et de l'alg\`{e}bre des
descentes}, thesis, defended November 1993.\newline
\url{http://www1.mat.uniroma1.it/people/malvenuto/Thesis.pdf}

\bibitem[MalReu98]{Mal-Reu}Claudia Malvenuto, Christophe Reutenauer,
\textit{Plethysm and conjugation of quasi-symmetric functions}, Discrete
Mathematics, Volume 193, Issues 1--3, 28 November 1998, Pages
225--233.\newline
\url{http://www.sciencedirect.com/science/article/pii/S0012365X98001423}

\bibitem[MalReu09]{Mal-Reu-DP}Claudia Malvenuto, Christophe Reutenauer,
\textit{A self paired Hopf algebra on double posets and
a Littlewood-Richardson rule},
Journal of Combinatorial Theory, Series A 118 (2011) 1322-1333.
A preprint version appeared as
\href{http://arxiv.org/abs/0905.3508v1}{arXiv:0905.3508v1}.

\bibitem[Manchon04]{Manchon-HA}Dominique Manchon, \textit{Hopf algebras, from
basics to applications to renormalization}, Comptes Rendus des Rencontres
Mathematiques de Glanon 2001 (published in 2003).\newline%
\url{http://arxiv.org/abs/math/0408405v2}

\bibitem[Montg93]{Montg-Hopf}Susan Montgomery, \textit{Hopf Algebras and their
Actions on Rings}, Regional Conference Series in Mathematics Nr. 82, AMS 1993.

\bibitem[NovThi05]{Nov-Thi} Jean-Christophe Novelli, Jean-Yves Thibon,
\textit{Hopf algebras and dendriform structures arising from parking functions},
Fundamenta Mathematicae 193 (2007), 189--241. A preprint also appears
on arXiv as \href{http://arxiv.org/abs/math/0511200v1}{arXiv:math/0511200v1}.

\bibitem[Stan11]{Stanley-EC1}Richard P. Stanley, \textit{Enumerative
Combinatorics, volume 1}, Cambridge University Press, 2011. \newline%
\url{http://math.mit.edu/~rstan/ec/ec1/}

\bibitem[Stan99]{Stanley-EC2}Richard P. Stanley, \textit{Enumerative
Combinatorics, volume 2}, Cambridge University Press, 1999.

\bibitem[Stan71]{Stanley-Thes}Richard P. Stanley, \textit{Ordered Structures and
Partitions}, Memoirs of the American Mathematical Society, No. 119, American
Mathematical Society, Providence, R.I., 1972. \newline
\url{http://www-math.mit.edu/~rstan/pubs/pubfiles/9.pdf}

\bibitem[Stan84]{Stanley-Peck}Richard P. Stanley,
\textit{Quotients of Peck posets}, Order, 1 (1984), pp. 29--34. \newline
\url{http://dedekind.mit.edu/~rstan/pubs/pubfiles/60.pdf}

\bibitem[Sweed69]{Sweedler-HA}Moss E. Sweedler, \textit{Hopf Algebras},
W. A. Benjamin 1969.

\end{thebibliography}


\end{document}
